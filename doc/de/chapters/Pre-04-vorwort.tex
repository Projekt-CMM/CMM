%!TEX root=../Vorlage_DA.tex
%	########################################################
% 							Vorwort
%	########################################################


%	--------------------------------------------------------
% 	Überschrift, Inhaltsverzeichnis
%	--------------------------------------------------------
\chapter*{Vorwort}
\addcontentsline{toc}{chapter}{Vorwort}


%	--------------------------------------------------------
% 	Inhalt
%	--------------------------------------------------------

Wir möchten uns bei unseren Projektbetreuern am Institut für Systemsoftware der Johannes Kepler Universität Linz bedanken, die uns bei unserem Praktikum zu Beginn des Projektes und während des Jahres unterstützt haben. Wir konnten einen interessanten Einblick in den Alltag und die Arbeitsweise an der Universität erhalten.

Besonderer Dank gilt unserem Projektbetreuer, Herrn Matejka, der uns während des Projektes stets mit Rat und Tat zur Seite gestanden ist und uns mit seiner Erfahrung als Lehrer einen anderen Blickwinkel auf die Entwicklung von C Compact eingebracht hat.

Einen wertvollen Beitrag haben die Schülerinnen und Schüler geleistet, die bei den Versuchen beteiligt waren und uns somit laufend auf die Schwächen und Probleme unserer Entwicklungsumgebung aufmerksam gemacht, uns aber auch das Potential und die Vorteile von C Compact gezeigt haben.

Schließlich möchten wir noch den Lehrern danken, die uns jeweils mehrere Unterrichtsstunden zur Verfügung gestellt haben, um diese Tests zu ermöglichen.

Wir hoffen, dass C Compact seine Anwendung im Unterricht nicht nur während dieser Versuche gefunden hat. Die Schülerinnen und Schüler konnten von von den Features und der anschaulichen Oberfläche von C Compact profitieren und gaben uns durchwegs positive Rückmeldungen. Damit C Compact jederzeit und von jedem verwendet werden kann, haben wir beschlossen, diese Entwicklungsumgebung unter der GNU General Public License öffentlich zur Verfügung zu stellen.