%!TEX root = "../../../DA_GUI.tex"

%	--------------------------------------------------------
% 		Wissenschaftliche Analyse: Einleitung
%	--------------------------------------------------------

\section{Einleitung}

Bei der Entwicklung eines Programms und der Gestaltung der Benutzeroberfläche ist es wichtig, die Anforderungen beziehungsweise die Anwendung zu kennen und den Fortschritt des Programms regelmäßig zu überprüfen.

Um sicherzustellen, dass C Compact für den Unterricht tatsächlich geeignet ist und die Schülerinnen und Schüler auch gut mit der Entwicklungsumgebung umgehen können, haben wir beschlossen, Versuche mit Schülerinnen und Schülern der ersten und zweiten Klassen durchzuführen.

\section{Versuche mit SchülerInnen}
\label{sec:sci-trial-intro}
Während unseres Projektes hatten wir insgesamt vier mal die Möglichkeit, unser Programm mit einer FSST-Gruppe (14-16 SchülerInnen) zu testen.

\def\arraystretch{1.6}
\begin{table}[h!]
\begin{tabular}{|l|l|l|l||l|l|}
\hline
Datum & Klasse & \parbox{1.3cm}{Anzahl Schüler} & Lehrer & Version & Testgegenstand \\
\hline
4. 11. 2014 & 2AHELS & 14 & Franz Matejka & Alpha 1.1 & Benutzeroberfläche \\
3. 12. 2014 & 2BHELS & 16 & Kurt Kreilinger & Alpha 1.2 & Benutzeroberfläche \\
18. 3. 2015 & 2BHELS & 15 & Christian Hanl & Alpha 1.4.2 & Benutzeroberfläche \\
22. 4. 2015 & 1AHELS & 16 & Reinhard Pfoser & Alpha 1.4.5 & Questsystem \\
\hline
\end{tabular}
\caption{Versuche mit Schülern der ersten und zweiten Klassen}
\end{table}
Siehe auch Kapitel \ref{sec:versions}, Versionsgeschichte.

Bei den ersten drei Versuchen sollte vor allem die Benutzeroberfläche und der Debugger verbessert werden, beim letzten Versuch sollte hingegen das Questsystem getestet und optimiert werden. Dieser Versuch wurde dementsprechend anders gestaltet (beispielsweise durch andere Aufgabenstellungen).

\subsection{Ziel der Versuche}
Die Versuche sollten nicht nur die grundsätzliche Anwendbarkeit von C Compact unter Beweis stellen, sondern auch Programmfehler, Bugs und Probleme bei der Bedienung aufzeigen. Als Entwickler ist es oft schwer, die Ansätze und Bedienungsvorgänge anderer Benutzer nachzuvollziehen. So entstehen zwangsläufig Fehler, die der Programmierer selbst nicht bemerkt, da er das Programm immer so bedient, wie er es für logisch erachtet.

%TODO kein Widerspruch mit Pointhi
Die Benutzeroberfläche kann, im Gegensatz zum Compiler oder Interpreter, nur beschränkt durch statische Codeanalysen und Testroutinen verbessert werden, da ein wichtiger Faktor für eine reibungslose Funktionalität der Benutzer ist. Wir konnten mit den Versuchen viele neue Ansätze sammeln und sahen, welche Funktionen und Features für die Schüler tatsächlich von Bedeutung sind.

\subsection{Ablauf eines Versuches}
\label{sec:sci-trials-gui-sched}
Die Versuche begannen immer mit einer Einführungsphase. Zuerst stellten wir uns vor und erklärten den Versuch und seinen Zweck. Dann folgte eine Einführung in C Compact, einerseits in die verwendete Sprache, andererseits auch in die Funktionsweise der Benutzeroberfläche. Im zweiten Versuch fanden wir heraus, dass es in der Einführungsphase besonders wichtig ist, auf den Debugger hinzuweisen. Besonders Schüler mit schlechteren Programmierkenntnissen waren mit der Funktionsweise eines Debuggers nicht vertraut und benutzten C Compact wie einen einfachen Texteditor. Im dritten und vierten Versuch führten wir zu Beginn eine einfache Übung zum Debugger mit der Klasse gemeinsam durch. Dafür konnten wir einen zuvor noch umfangreicheren Teil der Einführung zu Spracheigenheiten von C Compact kürzen, da Compiler und Präprozessor im Laufe des Projektes große Fortschritte erzielen konnten und die Sprache C immer ähnlicher wurde.

Den größte Teil der Versuchszeit arbeiteten die Schüler selbstständig an von uns gestellten Aufgaben. Bei den ersten drei Versuchen wurden Übungsblätter in Form eines PDF-Dokumentes gemeinsam mit C Compact verteilt; beim letzten Test wurden die Aufgaben mit dem Questsystem gestellt. In dieser Übungszeit beobachteten wir die Schüler, notierten Probleme und Programmfehler und halfen bei Schwierigkeiten. Auf diese Weise erhielten wir einen guten Eindruck von Problemen sowie Vor- und Nachteilen bei der Anwendung.

Im letztlich vergleichbare Ergebnisse zu erzielen und einen objektiven Gesamteindruck zu erhalten, wurden in den letzten 10 bis 15 Minuten der verwendeten Unterrichtszeit Fragebögen mit Fragen zum jeweiligen Versuchsthema ausgeteilt. Während die in der Übungszeit aufgenommenen Informationen vor allem einzelne Probleme aufzeigen, hatten wir durch die Fragebögen die Möglichkeit, die weitere Arbeit an C Compact nach einem allgemeinen Trend zu orientieren.
%TODO ref Anhang-> Versuchsunterlagen, FRAGEBÖGEN

\subsection{Persönliche Erfahrungen}
Bei den Versuchen mit Schülergruppen konnten wir auch außerhalb des Projekts viele hilfreiche Erfahrungen sammeln. Wir mussten die für den Versuch verwendeten Unterrichtsstunden selbst vorbereiten und ein Thema aus der Informatik, wie etwa den Sortieralgorithmus \glqq{}Bubblesort\grqq{} und die zugehörigen Übungen erklären. Anschließend unterstützten wir die Schülerinnen und Schüler der Unterrichtsgruppe beim Lösen der Aufgaben.

Dadurch konnten wir einerseits ein besseres Verständnis für den Programmierunterricht und die Aufgabe des Lehrers erreichen, was auch für weitere Entwicklungen des Projektes hilfreich war. Andererseits ist es auch eine interessante persönliche Erfahrung, nicht unter den Schülern zu sitzen, sondern vor der Klasse zu stehen.
