%!TEX root=../../DA_GUI.tex
%	########################################################
% 				Allgemeiner Teil (Theorie)
%	########################################################


%	--------------------------------------------------------
% 	Versionsgeschichte
%	--------------------------------------------------------
\chapter{Versionsgeschichte}
\label{sec:versions}
Dieses Kapitel enthält eine Auflistung der Versionen von C Compact. Die Versionen werden wie folgt unterteilt:
\begin{itemize}
\item \textbf{Pre-Alpha:} Die Versionsunterteilung begann erst im Herbst 2014. Alle Versionen, die zuvor entstanden sind, sind nicht nummeriert. Das betrifft im Wesentlichen unsere Arbeit am Institut für Systemsoftware der Johannes Kepler Universität im Juli 2014.
\item \textbf{Alpha:} Versionen mit dem Namen Alpha umfassen die Entwicklungen vom Herbst 2014 bis Anfang 2015.
\end{itemize}

Nachfolgend werden alle bisher benannten Versionen aufgelistet und beschrieben. Es ist zu beachten, dass Versuche mit Schülern immer am Beginn der aktuellen Version durchgeführt wurden. Die Entwicklung einer durch einen Versuch getestete Version wurde also durch den Versuch selbst wesentlich beeinflusst und optimiert.

%	--------------------------------------------------------
% 	Kein Versionsname
%	--------------------------------------------------------
\subsubsection*{Pre-Alpha}
7. Juli 2014 bis 30.September 2014
\begin{itemize}
\item Diese Version umfasst die Arbeiten in unserer Zeit an der JKU sowie die grundlegende Implementierung der drei Hauptkomponenten des Projektes:
\begin{itemize}
\item Compiler
\item Interpreter
\item Benutzeroberfläche
\end{itemize}
\item Dateimanagement: Neue Datei - Öffnen - Speichern
\item Datentyp \glqq string\grqq (ähnlich zu verwenden wie in Java)
\item Datentyp \glqq bool\grqq
\item Minimaler Präprozessor
\end{itemize}

%	--------------------------------------------------------
% 	Alpha 1.0
%	--------------------------------------------------------
\subsubsection*{Alpha 1.0}
1. Oktober 2014 bis 3. November 2014
\begin{itemize}
\item Variablen können sowohl als Baumstruktur, als auch in Tabellen dargestellt werden
\item Beim Schließen wird überprüft, ob die aktuelle Datei verändert wurde
\item Implementierung einer Standardbibliothek
\item Breakpoints
\item Operatoren für Inkrementation und Dekrementation
\item switch/case
\item Schlüsselwort \glqq nop\grqq
\item Verbesserte Tests mit Travis-CI
Zuweisung der Arraygrößen mit Konstanten
\item Elementarfunktion \glqq printf\grqq
\end{itemize}

%	--------------------------------------------------------
% 	Alpha 1.1
%	--------------------------------------------------------
\subsubsection*{Alpha 1.1}
4. November 2014 bis 23. November 2014,\newline
Getestet: 5. November 2014 mit der 2AHELS
%%TODO Referenz auf 1. Versuch
\begin{itemize}
\item Variablenanzeige mit Tabellen wurde entfernt
\item Variablenanzeige als Baumstruktur wurde wesentlich optimiert
\item Step-over- und Step-out-buttons immer sichtbar
\item Erkennen von Buffer Overflows und Variablenüberlauf
\item Erkennen uninitialisierter Variablen
\item Schlüsselwort \glqq library\grqq
\item Übergeben von Arrays als Funktionsparameter
\item Speichern von Laufzeitinformationen von Variablen
\end{itemize}

%	--------------------------------------------------------
% 	Alpha 1.2 (Build 1/4)
%	--------------------------------------------------------
\subsubsection*{Alpha 1.2}
24. November 2014 bis 14. Dezember 2014,\newline
Getestet: 3. Dezember 2014 mit der 2BHELS
\begin{itemize}
%%TODO Referenz auf Ausführungsmodus
\item Ausführungsmodus (Texteditor, Fehler, Automatisches Debuggen, Schritt-für-Schritt-Debuggen) wird sichtbar durch ein farbiges Panel über dem Textfeld gekennzeichnet.
\item Step-over und Step-out Funktionen vorläufig entfernt
\item HTML-Dokumente für unterschiedliche Fehler im Debugmodus (Compiler, Preprozessor, Laufzeitfehler)
\item Tooltips für die wichtigsten Bedienelemente (Buttons, etc.)
\item Logarithmische Skalierung der Zeitschritte beim automatischen Debuggen
\item Die größten Elemente der Benutzeroberfläche sind verschiebbar und damit besser anzupassen
\item Eigener Syntax Highlighter für die Sprache von C Compact 
\item Ein neuer Präprozessor sorgt für reibungslosen Ablauf und ermöglicht Bedingungen wie etwa \glqq \#ifdef\grqq
\item Die Zeile von Laufzeitfehlern wird korrekt erkannt und ausgegeben
\item Hinzufügen spezieller Elementarfunktionen wie \glqq time\grqq und \glqq \_\_assert\_\_\grqq für Bibliotheken und fortgeschrittene Benutzer
\end{itemize}

%	--------------------------------------------------------
% 	Alpha 1.3
%	--------------------------------------------------------
\subsubsection*{Alpha 1.3}
Ab 15. Dezember 2014
\begin{itemize}
\item Arrays werden jetzt wie in C deklariert
\item Präprozessor kann Fehler werfen, die dem Benutzer angezeigt werden (wenn beispielsweise eine Bibliothek nicht gefunden wurde)
\item Das Projekt wurde offiziell unter der GNU General Public License 3 lizenziert\footnote{http://www.gnu.org/copyleft/gpl.html}
\item Das Questsystem wurde in die Benutzeroberfläche integriert. Dazu gehören:
\begin{itemize}
\item Ein Launcher, um ein Benutzerprofil auszuwählen
\item Ein neues Panel im Hauptfenster, in dem der Benutzer Informationen über die aktuelle Quest erhält und überprüfen kann, ob sein Sourcecode die Anforderungen der Quest erfüllt
\item Ein Auswahlfenster zum Selektieren und Starten einer Quest
\item Ein an das Questsystem angepasstes System für globale Einstellungen (um zum Beispiel für Schriftgröße oder die zuletzt verwendeten Benutzerprofile zu speichern)
\end{itemize}
\end{itemize}

\subsection*{Alpha 1.4.1}
Ab 14. März 2015
\begin{itemize}
\item Diese wurde erstmals als Pre-Release veröffentlicht\footnote{https://github.com/Projekt-CMM/CMM/releases}
\item Seit der letzten Version wurden sehr viele Bugs und Fehler behoben, die Benutzeroberfläche läuft sehr stabil
\item Es wurden einige kleinere Features hinzugefügt, wie etwa
\begin{itemize}
\item Ein Popup zum Anzeigen des Inhaltes eines Arrays im Debugger
\item Ein Popup zum Anzeigen von Rückgabewerten von Funktionen während der Laufzeit
\item Optionsmenü für die Benutzeroberfläche
\end{itemize}
\end{itemize}

\subsection*{Alpha 1.4.2}
Ab 17. März 2015\\
Getestet: 18. März 2014 mit der 2BHELS
\begin{itemize}
\item Bugs im Popup-System wurden behoben
\item Popups wurden verbessert (Event handling)
\item Bugs in der Benutzeroberfläche behoben
\begin{itemize}
\item Dateimanagement
\item Zeilenmarkierung im Debugger
\end{itemize}
\item Beispieldateien hinzugefügt
\end{itemize}

\subsection*{Alpha 1.4.4}
Ab 14. April 2015
\begin{itemize}
\item Das Questsystem wurde verbessert und vollständig in die Benutzeroberfläche integriert.
\item Alpha 1.4.4 ist die erste Pre-Release, bei der das Questsystem offiziell unterstützt wird.
%TODO ref Versuch 4
\item Die bei dem Versuch am 22. 4. verwendeten Aufgaben (Quests) sind hier bereits als Beispielaufgaben beigelegt
\end{itemize}

\subsection*{Alpha 1.4.5}
Ab 21. April 2015\\
Getestet: 22. April mit der 1AHELS\\
Enthält leichte optimierungen gegenüber Version 1.4.5 


