%!TEX root=../Vorlage_DA.tex
%	########################################################
% 				Allgemeiner Teil (Theorie)
%	########################################################


%	--------------------------------------------------------
% 	Versionsgeschichte
%	--------------------------------------------------------
\chapter{Versionsgeschichte}
Dieses Kapitel enthält eine Auflistung der Versionen von C Compact. Die Versionen werden wie folgt unterteilt:
\begin{itemize}
\item \textbf{Kein Versionsname:} Die Versionsnummerierung begann erst im Herbst 2014. Alle Versionen, die zuvor entstanden sind, sind nicht nummeriert. Das betrifft im wesentlichen unsere Arbeit am Institut für Systemsoftware der Johannes Kepler Universität im Juli 2014. Zu dieser Zeit war eine Unterteilung in Versionen noch nicht unbedingt nötig.
\item \textbf{Alpha:} Versionen mit dem Namen Alpha umfassen die Entwicklungen vom Herbst 2014 bis Anfang 2015. Unter die Bezeichnung Alpha fallen alle Versionen, in denen das Questsystem noch nicht vollständig implementiert ist.
%%TODO Referenz auf Questsystem
\end{itemize}

Buildnummern wurden nicht nach der tatsächlichen Anzahl der Builds vergeben, sondern nach kleinen Erweiterungen, die noch in die aktuelle Version eingegliedert wurden.

Nachfolgend werden alle bisher benannten Versionen aufgelistet und beschrieben. Es sind nicht alle Builds vollständig aufgelistet, da die Änderungen teilweise minimal waren.

%	--------------------------------------------------------
% 	Kein Versionsname
%	--------------------------------------------------------
\subsubsection*{Pre-Alpha}
Entwicklungsstart: 7. Juli 2014
\begin{itemize}
\item Umfasst alle grundlegenden Funktionen: eine einfache, C-ähnliche Sprache und die Grundversion der aktuellen Benutzeroberfläche
\item Dateimanagement: Neue Datei - Öffnen - Speichern
\item String als Datentyp (ähnlich wie in Java)
\item Minimaler Präprozessor
\end{itemize}

%	--------------------------------------------------------
% 	Alpha 1.0
%	--------------------------------------------------------
\subsubsection*{Alpha 1.0}
Gültig ab: 1. Oktober 2014
\begin{itemize}
\item Variablen können sowohl als Baumstruktur, als auch in Tabellen dargestellt werden
\item Beim Schließen wird überprüft, ob die aktuelle Datei verändert wurde
\item Implementierung einer Standardbibliothek
\item Breakpoints
\end{itemize}

%	--------------------------------------------------------
% 	Alpha 1.1 (Build 3)
%	--------------------------------------------------------
\subsubsection*{Alpha 1.1}
Gültig ab: 4. November 2014,
Getestet: 5. November 2014 mit der 2AHELS
%%TODO Referenz auf 1. Versuch
\begin{itemize}
\item Variablenanzeige mit Tabellen wurde entfernt
\item Variablenanzeige als Baumstruktur wurde wesentlich optimiert
\item Step-over- und Step-out-buttons immer sichtbar
\end{itemize}

%	--------------------------------------------------------
% 	Alpha 1.2 (Build 1)
%	--------------------------------------------------------
\subsubsection*{Alpha 1.2 (Build 1)}
Gültig ab: 24. November 2014
\begin{itemize}
%%TODO Referenz auf Ausführungsmodus
\item Ausführungsmodus (Texteditor, Fehler, Automatisches Debuggen, Schritt-für-Schritt-Debuggen) wird sichtbar durch ein farbiges Panel über dem Textfeld gekennzeichnet.
\item Arrays können an Funktionen als Referenz übergeben werden
\item Entfernen der Step-over und Step-out Funktionen
\item Schlüsselwort "library" für Funktionen, die im Debugger übersprungen werden sollen
\item HTML-Dokumente für unterschiedliche Fehler im Debugmodus (Compiler, Preprozessor, Laufzeitfehler)
\item Tooltips für die wichtigsten Bedienelemente (Buttons, etc.)
\item Logarithmische Skalierung der Zeitschritte beim automatischen Debuggen
\item Die größten Elemente der Benutzeroberfläche sind verschiebbar und damit besser anzupassen
\end{itemize}

%	--------------------------------------------------------
% 	Alpha 1.2 (Build 4)
%	--------------------------------------------------------
\subsubsection*{Alpha 1.2 (build 4)}
Gültig ab: 2. Dezember 2014,
Getestet: 3. Dezember 2014 mit der 2BHELS
%%TODO Referenz auf 2. Versuch
\begin{itemize}
\item Verbessertes Design bei Fehlermeldungen
\item Eigener Syntax Highlighter für die Sprache von C Compact
\end{itemize}


