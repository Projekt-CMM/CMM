%!TEX root=../Vorlage_DA.tex
%	########################################################
% 				Lexikalische Struktur
%	########################################################

\subsection{Standardbibliotek}

Der C-Compact Standard besitzt zur Kommunikation zwischen Programm und Oberfläche nur wenige Funktionen. Funktionalitäten wie Ein/Ausgabe von Strings, das Umwandeln von int nach string,... müssen daher direkt in C-Compact implementiert werden.

In C-Compact sind die wichtigsten Funktionen, welche in der C-Standardbibliotek vorkommen nachimplementiert.

\subsubsection{math.h}

\subsubsection{stdio.h}

\begin{itemize}
  \item void putc(char ch);
  \item void prints(string s);
  \item string scanf();
\end{itemize}

\subsubsection{stdlib.h}

\begin{itemize}
  \item string itoa(int x);
  \item string ftoa(float f);
  \item int atoi(string s);
%  \item float atof(string s);
  \item int rand();
  \item void srand (int seed);
\end{itemize}

%\subsubsection{string.h}

%\subsubsection{time.h}