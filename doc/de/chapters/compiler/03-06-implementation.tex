%!TEX root=../Vorlage_DA.tex
%	########################################################
% 				Allgemeiner Teil (Theorie)
%	########################################################


%	--------------------------------------------------------
% 	Implementierung
%	--------------------------------------------------------
\subsection{Implementierung}

\subsubsection{CMM.atg}

In CMM.atg ist die gesamte Attributierte Grammatik von C-Compact enthalten. Mit Hilfe der darin enthaltenen Information kann Coco/R eine Scanner und einen Parser generieren\footnote{\url{http://www.ssw.uni-linz.ac.at/Coco/Doc/UserManual.pdf}}, welche nach der definierten EBNF-Syntax arbeitet.

Die Datei enth\"alt unter anderem die EBNF-Syntax, welche den grundlegenden Aufbau der Programmiersprache definiert. Des Weiteren kann Java-Code eingebettet werden, welcher ausgef\"uhrt wird wenn bestimmte Teile der EBNF durchlaufen werden. So ist es m\"oglich einen abstrakten Syntaxbaum und eine Symboltabelle aufzubauen, und den durchlaufenen Code auf Syntaxbedingungen zu \"uberpr\"ufen, welche von der EBNF nicht erkannt werden k\"onnen. Da Coco/R einen LL(1) Parser generiert, gewisse Teile der Programmiersprache aber nicht mithilfe eines LL(1) Parsers implementiert werden k\"onnen gibt es LL(1) conflict resolvers, welche den Parser an bestimmten Stellen mitteilen k\"onnen, wie dieser weiter zu Verfahren hat.

\htlParagraph{Beispiel}

Es ist in C-Compact m\"oglich explizite Typenumwandlung durchzuf\"uhren. Dies wird wie in C realisiert, indem der gew\"unschte Typ in Klammern angegeben wird. Es ist aber genauso m\"oglich in Gleichungen Klammern einzusetzen, wodurch Coco/R nicht in der Lage ist zwischen einer Typumwandlung, und einer Gleichung (oder auch einer geklammerten Variable) zu unterscheiden.

Um dieses Problem zu l\"osen wird ein LL(1) conflict resolvers eingesetzt. Dieser kann einerseits mehrere Tokens nach vorne schauen, und anderseits den Token auf bereits bekannten Informationen \"uberpr\"ufen (z.B. ist dieser Identifier ein Typ oder eine Variable). Auf Basis dieser Informationen wird dann der weitere Parservorgang gesteuert.

\begin{lstlisting}[language=Java]
boolean isCast() {
	// get next token
	Token x = scanner.Peek();

	// if it is not an identifier, it cannot be a cast
	if (x.kind != _ident) 
		return false;

	// get the identifier
	Obj obj = tab.find(x.val);

	// check if the identfier declare a type
	return obj.kind == Obj.TYPE;
}
\end{lstlisting}

Um zu \"uberpr\"ufen ob eine Typumwandlung oder ein normaler Klammerausdruck vorliegt wird zuerst der n\"achste Token ermittelt. Dann wird \"uberpr\"uft ob \"uberhaupt ein korrekter Token vorliegt (ident), und falls dies zutrifft wird mit hilfe der Symboltabelle \"uberpr\"uft ob der vorliegende Identifier einen g\"ultigen Typen darstellt.

Es kann dann dieser LL(1) conflict resolvers genutzt werden, um auf Basis der vorliegenden Informationen zu entscheiden, wie bei dem weiteren Parservorgang vorgegangen werden soll. In diesem Fall wird bei vorliegen einer Typkonvertierung eine Explizite Typumwandlung durchgef\"uhrt, welche wenn n\"otig den Abstrakten Syntaxbaum erweitert.

\begin{lstlisting}[language=EBNF]
//...
| IF (isCast())                        
  "(" 
  Type<out type>
  ")"
  Factor<out n>    (.  n = tab.expliciteTypeCon(n, type); .)
| "("
    BinExpr<out n>
 ")"
\end{lstlisting}

\subsubsection{Scanner.java}

Diese Datei wird auf Grundlage von CMM.atg und Scanner.frame durch den Parsergenerator Coco/R generiert. Diese Datei ist ein Scanner basierend auf einem endlichen Automaten\footnote{\url{https://de.wikipedia.org/wiki/Coco/R}}, welcher die in CMM.atg definierten Tokens extrahiert.

Diese Tokens k\"onnen einerseits explizit definiert sein, oder werden ansonsten aus der definierten EBNF extrahiert.

\subsubsection{Parser.java}

Der Parser wird auch auf Grundlagen von CMM.atg und Parser.frame generiert. Dieser Parser arbeitet dabei auf Basis des rekursiven Abstiegs\footnote{\url{https://de.wikipedia.org/wiki/Rekursiver_Abstieg}}, in der jede Produktionsregel eine eigene Funktion darstellt, welche rekursiv aufgerufen werden k\"onnen.

\htlParagraph{WEAK Token}

Um bessere Fehlermeldungen ausgeben zu k\"onnen ist es m\"oglich sogenannte schwache Token zu definieren. Diese befinden sich meist am Anfang von Iterationen, und werden gerne vergessen oder falsch geschrieben. Das definieren eines sogenannten schwachen Token kann f\"ur bessere Fehlermeldungen sorgen, da der Parser dieses wenn n\"otig ignoriert und dann versucht sich neu zu synchronisieren.

\htlParagraph{Synchronisation}

Wenn im Parser ein Fehler auftritt, da ein unbekanntes Token im Tokenstrom auftritt, muss sich dieser neu synchronisieren, um weiter Code Parsen zu k\"onnen. Dass Programm ist nicht l\"anger valide, aber so ist es m\"oglich weitere Fehler im Code zu erkennen.

Dies wird durch das Keyword SYNC in Coco/R realisiert, welches sichere Tokens spezifiziert, welche normalerweise daf\"ur sorgen dass das Programm von einem sicheren Zustand weiter geparst werden kann. Solch ein Token stellt z.B. ein Semikolon dar.

\htlParagraph{Beispiel}

Eine Condition besteht jeweils aus mindestens einem CondTerm, und kann auch eines oder mehrere bin\"are ODER beinhalten welche wiederum eine CondTerm folgt.

\begin{lstlisting}[language=EBNF]
Condition<out Node con>   (.  Node newCon; .)
= 
CondTerm<out con>
{ "||"                                
  CondTerm<out newCon>    (.  con = new Node(Node.OR, con, newCon, Tab.boolType); .)
}.
\end{lstlisting}

Der generierte Code ist eine Funktion welcher die obige Attributierte Grammatik repr\"sentiert. Dabei wird zuerst die Variable f\"ur den R\"uckgabewert definiert, und darauf folgend kommt noch die Variablendefinition welche in der Attributierten Grammatik angegeben wurde.

Die EBNF beginnt zwingend mit einem CondTerm, welcher in einer eigenen Funktion definiert wurde (rekursiver Aufruf von Produktionsregeln), und einen R\"uckgabewert besitzt. Wenn der CondTerm durchlaufen ist wird \"uberpr\"uft ob das n\"achste Token ein bin\"ares ODER darstellt, welches \"uber die Tokennummer eindeutig identifiziert werden kann.

Falls das folgende Token dem gesuchtem entspricht, wird die Schleife durchlaufen, wobei zuerst das n\"achste Token abgerufen wird, und danach wiederum ein CondTerm folgen muss. Zuletzt wird der Abstrakte Syntaxbaum erweitert, wie es bereits in der Attributierten Grammatik angegeben wurde.

\begin{lstlisting}[language=Java]
Node  Condition() {
	Node con;
	Node newCon; 
	con = CondTerm();
	while (la.kind == 32) {
		Get();
		newCon = CondTerm();
		con = new Node(Node.OR, con, newCon, Tab.boolType); 
	}
	return con;
}
\end{lstlisting}

\subsubsection{Node.java}

Der Abstrakte Syntaxbaum wird aus einzelnen Nodes aufgebaut, welche sich Baumartig verkn\"upfen lassen. In unserer Implementierung verwenden wir einen homogenen Syntaxbaum, in welchem je nach Knotentyp unterschiedliche Daten gespeichert werden.

Jeder Knoten ist von einer speziellen Art, welcher mithilfe einer Nummer definiert wurde. So gibt es einen eigenen Knoten f\"ur eine Variable, ein Bitweises UND oder eines Index. Im Interpreter werden die gleichen Arten von Knoten verwendet, um dann das Programm korrekt zu durchlaufen. Die Werte von Konstanten werden direkt in einem Node gespeichert, Variablen dagegen erhalten einen Verweis auf das dementsprechende Objekt wo diese definiert wurde.

\begin{lstlisting}[language=Java]
public int kind;        // STATSEQ, ASSIGN, ...
public Struct type;     // only used in expressions
public int line;        // only used in statement nodes
public int col;         // only used in statement nodes
public int colLength;

public Node left;       // left son
public Node right;      // right son
public Node next;       // for linking statements, parameters,.

public Obj obj;         // object node of an IDENT
public int val;         // value of an INTCON or CHARCON
public float fVal;      // value of a FLOATCON
\end{lstlisting}

\subsubsection{Scope.java}

Die Symboltabelle basiert auf mehreren Ebenen, welche unterschiedliche logische Ebenen eines Programmes repr\"asentieren, aus dessen es aufgebaut ist.

Als unterste Ebene gibt es den Universe-Scope, in welchem alle Standardtypen wie auch Standardfunktionen definiert sind. Diese stellen elementare Elemente einer Programmiersprache dar, welche immer verf\"ugbar sind.

Dar\"uber gibt es den Standardscope, in welchem globale Variablen wie auch Prozeduren deklariert sind. Wenn eine Prozedur deklariert wird, wird tempor\"ar ein neuer Scope erstellt, in dem die Variablen deklariert werden. Wenn die Funktion fertig deklariert wurde, wird dessen Inhalt in das Funktionsobjekt kopiert und der Scope geschlossen.

Bei der Namensaufl\"osung von Variablen wird immer zuerst der aktuelle Scope durchsucht, und wenn die suche erfolglos ist der darunterliegende Scope, und wenn die Suche wieder erfolglos ist wieder der n\"achste scope. Dadurch ist es m\"oglich Variablen mit gleichem Namen als globale Variable wie auch als Funktionsvariable zu deklarieren.

\begin{lstlisting}[language=Java]
public class Scope {
	public Scope outer;  // to outer scope
	public Obj   locals; // to local variables of this scope
	public int   size;   // total size of variables in this scope
}
\end{lstlisting}

\subsubsection{Obj.java}

Ein Objektnode wird dazu verwendet definierte Konstanten, Variablen, Typen und Funktionen in die Symboltabelle einzutragen. Dabei wird bei Variablen zum Beispiel gespeichert welchen Offset diese zum FramePointer haben, oder von welchem Typ diese ist.

Bei Prozeduren gibt es eine Referenz zu dem dahinter liegenden Abstrakten Syntaxbaum, welcher durchlaufen wird wenn die Funktion aufgerufen wird. Auch wird vermerkt wie viel Speicher allokiert werden muss, um alle Variablen im Stack speichern zu k\"onnen, oder wie viele Variablen Funktionsargumente darstellen.

\begin{lstlisting}[language=Java]
public int     kind;      // CON, VAR, TYPE, PROC
public String  name;      // object name
public Struct  type;      // object type
public Obj     next;      // next local object in this scope
public boolean library;   // is this Obj a library function?

public int     val;       // CON: int or char value
public float   fVal;      // CON: float value

public int     adr;       // VAR: address
public int     level;     // VAR: declaration level
public boolean isRef;     // VAR: ref parameter
	
public int     line;      // VAR, CON: line of declaration

public Node    ast;       // PROC: AST of this procedure
public int     size;      // PROC: frame size in bytes
public int     nPars;     // PROC: number of formal parameters
public Obj     locals;    // PROC: parameters and local objects
public boolean isForward; // PROC: is it a forward declaration
\end{lstlisting}

\subsubsection{Compiler.java}

\subsubsection{Error.java}

\subsubsection{NodeList.java}

\subsubsection{Strings.java}

\subsubsection{Struct.java}

\subsubsection{Tab.java}