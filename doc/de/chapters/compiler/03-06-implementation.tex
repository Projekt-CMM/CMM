%!TEX root=../Vorlage_DA.tex
%	########################################################
% 				Allgemeiner Teil (Theorie)
%	########################################################


%	--------------------------------------------------------
% 	Implementierung
%	--------------------------------------------------------
\subsection{Implementierung}

\subsubsection{CMM.atg}

In CMM.atg ist die gesamte attributierte Grammatik von C Compact enthalten. Mit Hilfe der darin enthaltenen Information kann Coco/R einen Scanner und einen Parser generieren\footnote{\url{http://www.ssw.uni-linz.ac.at/Coco/Doc/UserManual.pdf}}, welche nach der definierten EBNF-Syntax arbeitet.

Die Datei enth\"alt unter anderem die EBNF-Syntax, welche den grundlegenden Aufbau der Programmiersprache definiert. Des Weiteren kann Java-Code eingebettet werden, welcher ausgef\"uhrt wird wenn bestimmte Teile der EBNF durchlaufen werden. So ist es m\"oglich, einen abstrakten Syntaxbaum und eine Symboltabelle aufzubauen, und den durchlaufenen Code auf Syntaxbedingungen zu \"uberpr\"ufen, welche von der EBNF nicht erkannt werden k\"onnen. Da Coco/R einen LL(1) Parser generiert, gewisse Teile der Programmiersprache aber nicht mithilfe eines LL(1) Parsers implementiert werden k\"onnen gibt es LL(1) conflict resolvers, welche den Parser an bestimmten Stellen mitteilen k\"onnen, wie dieser weiter zu verfahren hat.

\htlParagraph{Beispiel}

Es ist in C Compact m\"oglich explizite Typenumwandlung durchzuf\"uhren. Dies wird wie in C realisiert, indem der gew\"unschte Typ in Klammern angegeben wird. Es ist aber genauso m\"oglich, in Gleichungen Klammern einzusetzen, wodurch Coco/R nicht in der Lage ist, zwischen einer Typumwandlung, und einer Gleichung (oder auch einer geklammerten Variable) zu unterscheiden.

Um dieses Problem zu l\"osen wird ein LL(1) conflict resolvers eingesetzt. Dieser kann einerseits mehrere Tokens nach vorne schauen, und anderseits den Token auf bereits bekannte Informationen \"uberpr\"ufen (z.B. ist dieser Identifier ein Typ oder eine Variable). Auf Basis dieser Informationen wird dann der weitere Parservorgang gesteuert.

\begin{lstlisting}[language=Java]
boolean isCast() {
	// get next token
	Token x = scanner.Peek();

	// if it is not an identifier, it cannot be a cast
	if (x.kind != _ident) 
		return false;

	// get the identifier
	Obj obj = tab.find(x.val);

	// check if the identfier declare a type
	return obj.kind == Obj.TYPE;
}
\end{lstlisting}

Um zu \"uberpr\"ufen, ob eine Typumwandlung oder ein normaler Klammerausdruck vorliegt, wird zuerst der n\"achste Token ermittelt. Dann wird \"uberpr\"uft, ob \"uberhaupt ein korrekter Token vorliegt (ident) und falls dies zutrifft wird mit Hilfe der Symboltabelle \"uberpr\"uft, ob der vorliegende Identifier einen g\"ultigen Typen darstellt.

Es kann dann der LL(1) conflict resolver genutzt werden, um auf Basis der vorliegenden Informationen zu entscheiden, wie bei dem weiteren Parservorgang vorgegangen werden soll. In diesem Fall wird bei vorliegen einer Typkonvertierung eine explizite Typumwandlung durchgef\"uhrt, welche wenn n\"otig den abstrakten Syntaxbaum erweitert.

\begin{lstlisting}[language=EBNF]
//...
| IF (isCast())                        
  "(" 
  Type<out type>
  ")"
  Factor<out n>    (.  n = tab.expliciteTypeCon(n, type); .)
| "("
    BinExpr<out n>
 ")"
\end{lstlisting}

\subsubsection{Scanner.java}

Diese Datei wird auf Grundlage von CMM.atg und Scanner.frame durch den Parsergenerator Coco/R generiert. Diese Datei ist ein Scanner basierend auf einem endlichen Automaten\footnote{\url{https://de.wikipedia.org/wiki/Coco/R}}, welcher die in CMM.atg definierten Tokens extrahiert.

Diese Tokens k\"onnen einerseits explizit definiert sein, oder werden ansonsten aus der definierten EBNF extrahiert.

\subsubsection{Parser.java}

Der Parser wird auch auf Grundlage von CMM.atg und Parser.frame generiert. Dieser Parser arbeitet dabei auf Basis des rekursiven Abstiegs\footnote{\url{https://de.wikipedia.org/wiki/Rekursiver_Abstieg}}, in der jede Produktionsregel eine eigene Funktion darstellt, welche rekursiv aufgerufen werden k\"onnen.

\htlParagraph{WEAK Token}

Um bessere Fehlermeldungen ausgeben zu k\"onnen ist es m\"oglich sogenannte schwache Token zu definieren. Diese befinden sich meist am Anfang von Iterationen, und werden gerne vergessen oder falsch geschrieben. Das Definieren eines sogenannten schwachen Token kann f\"ur bessere Fehlermeldungen sorgen, da der Parser dieses wenn n\"otig ignoriert und dann versucht sich neu zu synchronisieren.

\htlParagraph{Synchronisation}

Wenn im Parser ein Fehler auftritt, da ein unbekanntes Token im Tokenstrom auftritt, muss sich dieser neu synchronisieren, um weiter Code Parsen zu k\"onnen. Das Programm ist nicht l\"anger valide, aber so ist es m\"oglich, weitere Fehler im Code zu erkennen.

Dies wird durch das Keyword SYNC in Coco/R realisiert, welches sichere Tokens spezifiziert, welche normalerweise daf\"ur sorgen dass das Programm von einem sicheren Zustand weiter geparst werden kann. Solch ein Token stellt z.B. ein Semikolon dar.

\htlParagraph{Beispiel}

Eine Condition besteht jeweils aus mindestens einem CondTerm, und kann auch eines oder mehrere bin\"are ODER beinhalten welche wiederum eine CondTerm folgt.

\begin{lstlisting}[language=EBNF]
Condition<out Node con>   (.  Node newCon; .)
= 
CondTerm<out con>
{ "||"                                
  CondTerm<out newCon>    (.  con = new Node(Node.OR, con, newCon, Tab.boolType); .)
}.
\end{lstlisting}

Der generierte Code ist eine Funktion welcher die obige attributierte Grammatik repr\"asentiert. Dabei wird zuerst die Variable f\"ur den R\"uckgabewert definiert und darauf folgend kommt noch die Variablendefinition welche in der attributierten Grammatik angegeben wurde.

Die EBNF beginnt zwingend mit einem CondTerm, welcher in einer eigenen Funktion definiert wurde (rekursiver Aufruf von Produktionsregeln) und einen R\"uckgabewert besitzt. Wenn der CondTerm durchlaufen ist wird \"uberpr\"uft, ob das n\"achste Token ein bin\"ares ODER darstellt, welches \"uber die Tokennummer eindeutig identifiziert werden kann.

Falls das folgende Token dem Gesuchtem entspricht, wird die Schleife durchlaufen, wobei zuerst das n\"achste Token abgerufen wird und danach wiederum ein CondTerm folgen muss. Zuletzt wird der abstrakte Syntaxbaum erweitert, wie es bereits in der attributierten Grammatik angegeben wurde.

\begin{lstlisting}[language=Java]
Node  Condition() {
	Node con;
	Node newCon; 
	con = CondTerm();
	while (la.kind == 32) {
		Get();
		newCon = CondTerm();
		con = new Node(Node.OR, con, newCon, Tab.boolType); 
	}
	return con;
}
\end{lstlisting}

\subsubsection{Node.java}

Der abstrakte Syntaxbaum wird aus einzelnen Nodes aufgebaut, welche sich baumartig verkn\"upfen lassen. In unserer Implementierung verwenden wir einen homogenen Syntaxbaum, in welchem je nach Knotentyp unterschiedliche Daten gespeichert werden.

Jeder Knoten ist von einer speziellen Art, welcher mithilfe einer Nummer definiert wurde. So gibt es einen eigenen Knoten f\"ur eine Variable, ein Bitweises UND oder eines Index. Im Interpreter werden die gleichen Arten von Knoten verwendet, um dann das Programm korrekt zu durchlaufen. Die Werte von Konstanten werden direkt in einem Node gespeichert, Variablen dagegen erhalten einen Verweis auf das dementsprechende Objekt wo diese definiert wurde.

\begin{lstlisting}[language=Java]
public int kind;        // STATSEQ, ASSIGN, ...
public Struct type;     // only used in expressions
public int line;        // only used in statement nodes
public int col;         // only used in statement nodes
public int colLength;

public Node left;       // left son
public Node right;      // right son
public Node next;       // for linking statements, parameters,.

public Obj obj;         // object node of an IDENT
public int val;         // value of an INTCON or CHARCON
public float fVal;      // value of a FLOATCON
\end{lstlisting}

\subsubsection{Scope.java}

Die Symboltabelle basiert auf mehreren Ebenen, welche unterschiedliche logische Ebenen eines Programmes repr\"asentieren, aus dessen es aufgebaut ist.

Als unterste Ebene gibt es den Universe-Scope, in welchem alle Standardtypen wie auch Standardfunktionen definiert sind. Diese stellen elementare Elemente einer Programmiersprache dar, welche immer verf\"ugbar sind.

Dar\"uber gibt es den Global-Scope, in welchem globale Variablen wie auch Prozeduren deklariert sind. Wenn eine Prozedur deklariert wird, wird tempor\"ar ein Local-Scope erstellt, in dem die Variablen deklariert werden. Wenn die Funktion vollst\"andig deklariert wurde, wird dessen Inhalt in das Funktionsobjekt kopiert und der Scope geschlossen.

Bei der Namensaufl\"osung von Variablen wird immer zuerst der aktuelle Scope durchsucht, und wenn die suche erfolglos ist der darunterliegende Scope, und wenn die Suche wieder erfolglos ist wieder der n\"achste scope. Dadurch ist es m\"oglich, Variablen mit gleichem Namen als globale Variable wie auch als Funktionsvariable zu deklarieren.

\begin{lstlisting}[language=Java]
public class Scope {
	public Scope outer;  // to outer scope
	public Obj   locals; // to local variables of this scope
	public int   size;   // total size of variables in this scope
}
\end{lstlisting}

\subsubsection{Obj.java}

Ein Objektnode wird dazu verwendet, definierte Konstanten, Variablen, Typen und Funktionen in die Symboltabelle einzutragen. Dabei wird bei Variablen zum Beispiel gespeichert, welchen Offset diese zum FramePointer besitzen, oder von welchem Typ diese sind.

Bei Prozeduren gibt es eine Referenz zu dem dahinter liegenden abstrakten Syntaxbaum, welcher durchlaufen wird wenn die Funktion aufgerufen wird. Auch wird vermerkt, wie viel Speicher allokiert werden muss, um alle Variablen im Stack speichern zu k\"onnen, oder wie viele Variablen Funktionsargumente darstellen.

\begin{lstlisting}[language=Java]
public int     kind;      // CON, VAR, TYPE, PROC
public String  name;      // object name
public Struct  type;      // object type
public Obj     next;      // next local object in this scope
public boolean library;   // is this Obj a library function?

public int     val;       // CON: int or char value
public float   fVal;      // CON: float value

public int     adr;       // VAR: address
public int     level;     // VAR: declaration level
public boolean isRef;     // VAR: ref parameter
	
public int     line;      // VAR, CON: line of declaration

public Node    ast;       // PROC: AST of this procedure
public int     size;      // PROC: frame size in bytes
public int     nPars;     // PROC: number of formal parameters
public Obj     locals;    // PROC: parameters and local objects
public boolean isForward; // PROC: is it a forward declaration
\end{lstlisting}


\subsubsection{Struct.java}

Die grundlegenden Eigenschaften von Datentypen werden in dem Objekt Struct definiert. Dabei handelt es sich um Informationen wie der Art des Typs, dessen Gr\"o\ss{}e, und bei zusammengesetzte Datentypen auch weiterf\"urende Informationen wie alle definierten Variablen einer Struktur oder den grundlegenden Datentyp eines Arrays (welcher wiederum ein Array sein kann).

\begin{lstlisting}[language=Java]
public int    kind;       // NONE, INT, FLOAT, CHAR, ARR, ...
public int    size;       // size of this type in bytes
public int    elements;   // ARR: number of elements
public Struct elemType;   // ARR: element type
public Obj    fields;     // STRUCT: fields
\end{lstlisting}

\subsubsection{Strings.java}

Strings sind in C Compact konstant. Das hei\ss{}t dass ein neuer String angelegt werden muss, falls ein bestehender ver\"andert wird. Gespeichert wird ein null-terminierter String in einem char Array, welches wenn n\"otig automatisch vergr\"o\ss{}ert wird. Wenn ein String abgerufen werden soll, muss die eindeutige id des Strings angegeben werden. Es ist auch m\"oglich, einzelne Zeichen eines Strings abzufragen, wobei die Anfangsposition ben\"otigt wird.

Um gleiche Strings nicht doppelt anzulegen gibt es eine HashMap, in der alle verf\"ugbaren Strings und dessen Anfangsposition vermerkt sind. So ist es m\"oglich, den Speicherverbrauch zu verringern indem Duplikate verhindert werden.

\begin{lstlisting}[language=Java]
static private char[] data = new char[4096];
static private int top = 0;
static private Map<String, Integer> map = new HashMap<>();
\end{lstlisting}

\subsubsection{Tab.java}

Diese Datei ist die zentrale Datei welche f\"ur viele Aktionen im Compiler verwendet werden. Hier wird der Universe-Scope mit allen Standard-Typen und Funktionen zu Beginn initialisiert. Des Weiteren sind hier die Methoden deklariert, welche f\"ur Aktionen wie Typkonvertierung, hinzuf\"ugen neuer Objekte in den aktuellen Scope oder das Suchen von Variablen welche im Aktuellen oder einem darunterliegenden Scope deklariert wurde.

\begin{lstlisting}[language=Java]
public Obj find(String name) {
	for(Scope scp = curScope; scp != null; scp = scp.outer) {
		for (Obj p = scp.locals; p != null ; p = p.next) {
			if (p.name.equals(name))
				return p;
		}
	}
	parser.SemErr(name + " not found");
	return noObj;
}
\end{lstlisting}

\htlParagraph{Beispiel}

Wenn ein Objekt in die Symboltabelle eingef\"ugt wird, muss einerseits das Objekt erstellt werden, anderseits ist es notwendig, je nach Art des Objekts zus\"atzliche Informationen einzutragen, welche sp\"ater unter anderem vom Compiler oder vom Interpreter ben\"otigt werden.

Auch kann es passieren, dass versehentlich der gleiche Identifier verwendet wurde, weshalb \"uberpr\"uft werden muss ob eine Variable, Funktion oder Typ bereits definiert worden ist. Es gibt jedoch auch bei Funktionen Vorw\"artsdeklarationen, welche jedoch pro Funktion nur 1 mal zugelassen sind.

Dabei muss wiederum beachtet werden, das Variablen im globalen wie auch im lokalen Skope definiert werden k\"onnen, ohne sich gegenseitig zu beeinflussen. Dagegen darf es nicht m\"oglich sein, Funktionen oder Typen zu \"uberschreiben.

\begin{lstlisting}[language=Java]
public Obj insert(int kind, String name, Struct type, int line) {
	//--- create
	Obj obj = new Obj(kind, name, type, line);
	if (kind == Obj.VAR) {
		obj.adr = curScope.size;
		curScope.size += type.size;
		obj.level = curLevel;
	}
	
	//--- check if function or type already exists in universe
	if(kind == Obj.PROC || kind == Obj.TYPE) {
		Scope checkScope = curScope;
		while(checkScope.outer != null) {
			checkScope = checkScope.outer;
			Obj p =  checkScope.locals;
			while (p != null) {
				if (p.name.equals(name)) {
					if(p.isForward) {
						return p;
					}
					parser.SemErr(name + " declared twice");
				}
				p = p.next;
			}
		}
	}
	//--- insert
	Obj p = curScope.locals, last = null;
	while (p != null) {
		if (p.name.equals(name)) {
			if(p.isForward) {
				return p;
			}
			parser.SemErr(name + " declared twice");
		}
		last = p;
		p = p.next;
	}
	if (last == null)
		curScope.locals = obj;
	else
		last.next = obj;
	return obj;
}
\end{lstlisting}

%\subsubsection{Compiler.java}

%\subsubsection{Error.java}

%\subsubsection{NodeList.java}

\subsubsection{Preprocessor.java}

Der Compiler wurde nur dazu entwickelt, einzelne Dateien in einen abstrakten Syntaxbaum zu \"ubersetzen. Um eine mehrere Dateien einzubinden (z.B. Standardbiblioteken) ist es aber notwendig, Dateien vor dem kompilieren zusammenzuf\"ugen, und Mehrfachincludes korrekt zu verarbeiten.

Um dies umzusetzen, wurde ein Pr\"aprozessor implementiert welcher alle grundlegenden Funktionen des C-Pr\"aprozessors nachbildet. Bewusst wurde bei dieser Implementierung die urspr\"ungliche Funktionsweise von \#define ver\"andert, um die Verwendung der sprach-internen Konstanten zu erzwingen.
%Eines der Hauptprobleme von \#define sind die Fehlenden Typinformationen bei der Verwendung als Konstante, welche bei einer Programmierumgebung f\"ur Anf\"anger am besten von Anfang an vermieden 

Der Pr\"aprozessor umfasst einen einfachen Scanner und Parser. Er ist in der Lage, Kommentare wie auch Kommentare in Strings korrekt auszuwerten und gesteuert mithilfe von Pr\"aprozessorfunktionen eine Quellcodedatei aufzubauen, die dann vom Compiler \"ubersetzt werden kann.

Um Fehlermeldungen wieder in eine korrekte Form zu bringen, ist es notwendig, die Position im generierten Quelltext auf den urspr\"unglichen Quelltext zur\"ukzurechnen. Dazu wird eine Datenstruktur erstellt welche angibt, welche Quelldatei an einer bestimmten Position im generierten Quelltext genutzt wird. Diese Information reicht aus, um die urspr\"ungliche Zeile im Quelltext zur\"uckzurechnen, da der Pr\"aprozessor keine Zeilen einf\"ugt oder l\"oscht, sondern diese nur ver\"andert.