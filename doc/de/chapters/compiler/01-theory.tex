%!TEX root=../Vorlage_DA.tex
%	########################################################
% 				Allgemeiner Teil (Theorie)
%	########################################################


%	--------------------------------------------------------
% 	Allgmeine Hinweise
%	--------------------------------------------------------
\section{Allgemeiner Teil (Theorie)}

\subsection{Erweiterte Backus-Naur-Form (EBNF)}

Die sogenannte Erweiterte Backus-Naur-Form (EBNF)\footnote{\url{https://de.wikipedia.org/wiki/Erweiterte_Backus-Naur-Form}} ist eine weiterentwicklung der Backus-Naur-Form (BNF)\footnote{\url{https://de.wikipedia.org/wiki/Backus-Naur-Form}}.

Vorteile der EBNF sind unter anderem:

\begin{itemize}
  \item einfacheres erstellen von Rekursionen
  \item bessere Lesbarkeit
  \item ...
\end{itemize}

\subsubsection{Terminalsymbol}

Ein sogenanntes Terminalsymbol\footnote{\url{https://de.wikipedia.org/wiki/Terminalsymbol}} wird nicht weiter zerlegt. Es stellt z.B. eine Zahl, eine Variable oder ein Schlüsselwort (if, while, ...) dar.

\subsubsection{Nichtterminalsymbol}

Ein Nichtterminalsymbol\footnote{\url{https://de.wikipedia.org/wiki/Nichtterminalsymbol}}

\subsubsection{Produktionsregel}

\footnote{\url{https://de.wikipedia.org/wiki/Produktionsregel}}

\newpage

\subsubsection{Eindeutigkeit}

Die Gramatik muss so aufgebaut sein, dass es für einen Tokenstrom nur eine eindeutige Möglichkeit gibt diesen zu parsen.\footnote{\url{http://www.iai.uni-bonn.de/III//lehre/vorlesungen/Informatik_I/WS05/Folien/VLWS0506-10.pdf}}

\htlParagraph{Beispiel}

Es soll eine EBNF-Regel erstellt werden welche einen Mathematischen Ausdruck, aufgebaut aus Additionen und Subtraktionen korrekt parsen.

\subhtlParagraph{Mehrdeutige EBNF Regel}

\includegraphics[scale=0.5]{./media/images/compiler/ambiguity_wrong.png}

\begin{lstlisting}[language=EBNF]
Expr = intCon | Expr ( "+" | "-" ) Expr.
\end{lstlisting}

Der Ausdruck $1-2+5$ kann mit dieser EBNF Regel auf mehreren Arten gelöst werden. Es liegt also eine Mehrdeutigkeit vor. Der Ausdruck $1-2+5$ kann dabei sowohl als $(1-2)+5$, aber auch als $1-(2+5)$ geparst werden.

\begin{tabular}{ c | c }
  $(1-2)+5=4$ & 
  $1-(2+5)=-6$ \\
  \includegraphics[scale=0.5]{./media/images/compiler/ambiguity_tree_correct.png} & 
  \includegraphics[scale=0.5]{./media/images/compiler/ambiguity_tree_wrong.png} \\
\end{tabular}

\subhtlParagraph{Eindeutige EBNF Regel}

\includegraphics[scale=0.5]{./media/images/compiler/ambiguity_correct.png}

\begin{lstlisting}[language=EBNF]
Expr = intCon [ ( "+" | "-" ) Expr ].
\end{lstlisting}

Der Ausdruck $1-2+5$ kann mit dieser EBNF Regel nur mehr auf eine Art gelöst werden. Dabei werden die Operatoren links-assoziativ behandelt. Der Ausdruck $1-2+5$ wird als $(1-2)+5$ ausgewertet, was Mathematisch den korrekte Weg darstellt.

\subsubsection{Beispiele}

\includegraphics[scale=0.5]{./media/images/compiler/ebnf_expr.png}

\begin{lstlisting}[language=EBNF]
Expr = Term { ( "+" | "-" ) Term }.
\end{lstlisting}

Eine Expression besteht dabei aus einem Term, und dann optional wiederum aus einem ''+'' oder ''-'' gefolgt von einem weiterem Term.

\includegraphics[scale=0.5]{./media/images/compiler/ebnf_term.png}
\begin{lstlisting}[language=EBNF]
Term = Factor { ( "*" | "/" ) Factor }.
\end{lstlisting}

Das Nichtterminalsymbol Term stellt wiederum eine Produktionsregel dar, welche aus einem Faktor, und dann optional wiederum aus einem ''*'' oder ''/'' gefolgt von einem weiterem Faktor besteht.

Durch solch einfache Regeln können z.B.: Mathematische Regeln wie Punkt vor Strich eindeutig und korrekt in einen Syntaxbaum übersetzt werden.

\includegraphics[scale=0.5]{./media/images/compiler/ebnf_designator.png}
\begin{lstlisting}[language=EBNF]
Designator = ident { "." ident | "[" BinExpr "]" }.
\end{lstlisting}


%\includegraphics[scale=0.5]{./media/images/compiler/ebnf_factor.png}

\subsection{Lexikalische Analyse - Scanner}

Bei der Lexikalischen Analyse zerlegt der sogenannt Scanner den Zeichenstrom in einen sogenannten Tokenstrom. Ein sogenannter Token\footnote{\url{https://de.wikipedia.org/wiki/Token_(\%C3\%9Cbersetzerbau)}} stellt dabei eine logische Einheit dar (z.B.: eine variable, oder ein operatorzeichen), und wird nicht mehr weiter zerlegt (Terminalsymbol).

\subsection{Syntaxanalyse - Parser}

Der Parser wandelt den Tokenstrom in einen Syntaxbaum um, welcher das Programm repräsentiert.

\subsection{Attributierte Gramatik}

\subsection{Abstrakter Syntaxbaum}

\subsection{Symboltabelle}

\subsection{Scope}