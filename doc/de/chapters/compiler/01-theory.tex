%!TEX root=../Vorlage_DA.tex
%	########################################################
% 				Allgemeiner Teil (Theorie)
%	########################################################


%	--------------------------------------------------------
% 	Allgmeine Hinweise
%	--------------------------------------------------------
\section{Allgemeiner Teil (Theorie)}

\subsection{Erweiterte Backus-Naur-Form (EBNF)}

Die sogenannte Erweiterte Backus-Naur-Form (EBNF)\footnote{\url{https://de.wikipedia.org/wiki/Erweiterte_Backus-Naur-Form}} ist eine weiterentwicklung der Backus-Naur-Form (BNF)\footnote{\url{https://de.wikipedia.org/wiki/Backus-Naur-Form}}. Das Anwendungsgebiet der BNF und EBNF ist das definieren von  kontextfreie Grammatiken\footnote{\url{https://de.wikipedia.org/wiki/Kontextfreie_Grammatik}}, welche untere anderem f\"ur Programmiersprachen benutzt werden.

Bei kontextfreier Gramatik wird jeweils Ein Nichtterminalsymbol auf eine beliebige Folge von Terminal- bzw. Nichtterminalsymbolen abgeleitet ($V \rightarrow w$). Wenn das Nichtterminalsymbol $V$ in einem anderem Nichtterminalsymbol verwendet wird, wird nicht auf den Kontext geachtet, in welcher das Nichtterminalsymbol $V$ eingebettet ist. Die Regel ist daher kontextfrei.

Vorteile der EBNF gegen\"uber der BNF sind unter anderem:

\begin{itemize}
  \item einfacheres erstellen von Rekursionen und optionalen Ausdr\"ucken
  \item bessere Lesbarkeit
  \item einfaches und eindeutiges definieren von Terminalsymbolen
\end{itemize}

\newpage

\subsubsection{Terminalsymbol}

Ein sogenanntes Terminalsymbol\footnote{\url{https://de.wikipedia.org/wiki/Terminalsymbol}} wird nicht weiter zerlegt. Es stellt z.B. eine Zahl, eine Variable oder ein Schlüsselwort (''if'', ''while'', ...) dar.

\begin{lstlisting}[language=EBNF]
Expr = Term { ( "+" | "-" ) Term }.
Term = intCon { ( "*" | "/" ) intCon }.
\end{lstlisting}

Terminalsymbole werden normalerweise entweder klein geschrieben (z.B. intCon), oder direkt in der Gramatik definiert (z.B. ''+'').

\subsubsection{Nichtterminalsymbol}

Ein Nichtterminalsymbol\footnote{\url{https://de.wikipedia.org/wiki/Nichtterminalsymbol}} besteht aus einen oder mehreren Terminalsymbolen bzw. Nichtterminalsymbolen.

\begin{lstlisting}[language=EBNF]
Expr = Term { ( "+" | "-" ) Term }.
Term = intCon { ( "*" | "/" ) intCon }.
\end{lstlisting}

Nichtterminalsymbole werden normalerweise gro\ss{} geschrieben. In diesem Beispiel stellen ''Expr'' und ''Term'' Nichtterminalsymbole dar, welche wiederum von anderen Nichtterminalsymbolen verwendet werden k\"onnen.

\subsubsection{Produktionsregel}

Die Produktionsregel\footnote{\url{https://de.wikipedia.org/wiki/Produktionsregel}} ist eine Regel, welche angibt wie aus W\"ortern neue W\"orter produziert werden.

\begin{lstlisting}[language=EBNF]
Expr = Term { ( "+" | "-" ) Term }.
Term = intCon { ( "*" | "/" ) intCon }.
\end{lstlisting}

Die Produktionsregel gibt an wie z.B. das Nichtterminalsymbol ''Expr'' aufgebaut ist. Eine Produktionsregel besteht jeweils aus einem Nichtterminalsymbol auf der linken Seite, und der jeweiligen Regel auf der rechten Seite welches aus Terminal- und Nichtterminalsymbolen aufgebaut ist.

\newpage

\subsubsection{Eindeutigkeit}

Die Gramatik muss so aufgebaut sein, dass es für einen Tokenstrom nur eine eindeutige Möglichkeit gibt diesen zu parsen.\footnote{\url{http://www.iai.uni-bonn.de/III//lehre/vorlesungen/Informatik_I/WS05/Folien/VLWS0506-10.pdf}}

\htlParagraph{Beispiel}

Es soll eine EBNF-Regel erstellt werden welche einen Mathematischen Ausdruck, aufgebaut aus Additionen und Subtraktionen korrekt parsen.
\subhtlParagraph{Mehrdeutige EBNF Regel}

\includegraphics[scale=0.5]{./media/images/compiler/ambiguity_wrong.png}

\begin{lstlisting}[language=EBNF]
Expr = intCon | Expr ( "+" | "-" ) Expr.
\end{lstlisting}

Der Ausdruck $1-2+5$ kann mit dieser EBNF Regel auf mehreren Arten gelöst werden. Es liegt also eine Mehrdeutigkeit vor. Der Ausdruck $1-2+5$ kann dabei sowohl als $(1-2)+5$, aber auch als $1-(2+5)$ geparst werden.

\begin{tabular}{ c | c }
  $(1-2)+5=4$ & 
  $1-(2+5)=-6$ \\
  \includegraphics[scale=0.5]{./media/images/compiler/ambiguity_tree_correct.png} & 
  \includegraphics[scale=0.5]{./media/images/compiler/ambiguity_tree_wrong.png} \\
\end{tabular}
\subhtlParagraph{Eindeutige EBNF Regel}

\includegraphics[scale=0.5]{./media/images/compiler/ambiguity_correct.png}

\begin{lstlisting}[language=EBNF]
Expr = intCon [ ( "+" | "-" ) Expr ].
\end{lstlisting}

Der Ausdruck $1-2+5$ kann mit dieser EBNF Regel nur mehr auf eine Art gelöst werden. Dabei werden die Operatoren links-assoziativ behandelt. Der Ausdruck $1-2+5$ wird als $(1-2)+5$ ausgewertet, was Mathematisch den korrekte Weg darstellt.

\subsubsection{EBNF-Beispiele}

\includegraphics[scale=0.5]{./media/images/compiler/ebnf_expr.png}

\begin{lstlisting}[language=EBNF]
Expr = Term { ( "+" | "-" ) Term }.
\end{lstlisting}

Eine Expression\footnote{\url{https://de.wikipedia.org/wiki/Ausdruck_(Programmierung)}} besteht dabei aus einem Term, und dann optional wiederum aus einem ''+'' oder ''-'' gefolgt von einem weiterem Term.

\includegraphics[scale=0.5]{./media/images/compiler/ebnf_term.png}
\begin{lstlisting}[language=EBNF]
Term = Factor { ( "*" | "/" ) Factor }.
\end{lstlisting}

Das Nichtterminalsymbol Term stellt wiederum eine Produktionsregel dar, welche aus einem Faktor, und dann optional wiederum aus einem ''*'' oder ''/'' gefolgt von einem weiterem Faktor besteht.

Durch solche einfachen Regeln können z.B.: Mathematische Regeln wie Punkt vor Strich eindeutig und korrekt in einen Syntaxbaum übersetzt werden.

\includegraphics[scale=0.5]{./media/images/compiler/ebnf_designator.png}
\begin{lstlisting}[language=EBNF]
Designator = ident { "." ident | "[" BinExpr "]" }.
\end{lstlisting}


%\includegraphics[scale=0.5]{./media/images/compiler/ebnf_factor.png}

\newpage

\subsection{Lexikalische Analyse - Scanner}

Bei der Lexikalischen Analyse zerlegt der sogenannt Scanner\footnote{\url{https://de.wikipedia.org/wiki/Tokenizer}} den Zeichenstrom in einen sogenannten Tokenstrom. Ein sogenannter Token\footnote{\url{https://de.wikipedia.org/wiki/Token_(\%C3\%9Cbersetzerbau)}} stellt dabei eine logische Einheit dar (z.B.: eine variable, oder ein Operatorzeichen), und wird nicht mehr weiter zerlegt (Terminalsymbol).

\htlParagraph{Zeichenstrom}

Der Scanner bekommt einen Zeichenstrom, welcher aus einzelnen Zeichen besteht. Ein Zeichen ist z.B. ein Buchstabe, eine Zahl oder Sonderzeichen wie ''='', oder ''+''.

\includegraphics[width=0.6\textwidth]{./media/images/compiler/input_characterstream.png}

\htlParagraph{Resultierender Tokenstrom}

Aus diesem Zeichenstrom werden dann die Tokens extrahiert, welche dann die sogenannten Terminalsymbole in der EBNF bzw. BNF darstellen. Tokens k\"onnen dabei aus 1. oder mehreren Zeichen bestehen.

Manche Tokens wie z.B. Variablen oder Konstanten besitzen au\ss{}er der Tokennummer desweiteren noch einen Tokenwert, mit dem sp\"ater z.B. der Variablenname, oder der Wert einer Variable ausgelesen werden kann.

\includegraphics[width=\textwidth]{./media/images/compiler/scanner_tokenstream.png}

\subsection{Syntaxanalyse - Parser}

Der Parser\footnote{\url{https://de.wikipedia.org/wiki/Parser}} wandelt danach den Tokenstrom in einen Syntaxbaum um, welcher das Programm repräsentiert.

\newpage

\htlParagraph{Syntaxbaum}

\includegraphics[width=0.6\textwidth]{./media/images/compiler/parser_syntaxtree.png}

\subsection{Abstrakter Syntaxbaum}

\subsection{Attributierte Gramatik}

\subsection{Symboltabelle}

\subsection{Scope}