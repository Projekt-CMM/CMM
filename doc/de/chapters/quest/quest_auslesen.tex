\section{Auslesen der Quests}
\subsection{Auslesen der Packages}

\begin{figure}[h] 
  \centering
     \includegraphics[width=1\textwidth]{./media/images/gui/package-tree-rekursion.png}
  \caption{Schematische Darstellung vom Auslesen der Packages}
  \label{fig:Package_Tree_Rekursion}
\end{figure}

Ob ein Ordner ein Package ist, wird mithilfe einer Rekursion ausgelesen. Der Ausgangspunkt dieser Rekursion bildet der Ordner \textbf{"`packages"'}. Dieser bildet auch den Haupt-Knoten. Nun wird \textit{\textbf{getFolderView()}} aufgerufen, und der relative Pfad und der Hauptknoten werden übergeben.

Von diesem Punkt aus wird in die nächsten Unterordner gewechselt. Dort wird nun zuerst überprüft, ob der gewählte Ordner Subordner enthält. Ist dies der Fall, wird durch die Liste der Unterordner mithilfe einer ForEach-Schleife durchiteriert.

Es wird nun überprüft, ob eines dieser Elemente ein Package ist. Trifft dies zu, wird das gerade iterierende Element als Unterknoten hinzugefügt und dieselbe Methode (\textit{\textbf{getFolderView()}}) wird nochmals aufgerufen.

Wenn das aktuelle Element kein Package ist, erfolgt erneut eine Überprüfung nach weiteren Quests im Unterordner. Werden solche gefunden, wird ein neuer Unterknoten erstellt und wiederum die eigene Methode aufgerufen.

Wenn diese Bedingung nicht zutrifft, wird der Hauptknoten wieder zurückgegeben.

\subsection{Überprüfungsiterationen und Rekursion}
\htlParagraph{isPackage()-Methode:}\\
Mithilfe von isPackage(), wird eine Funktion aufgerufen welche eine Iteration durchführt und alle Ordner mithilfe der \textit{\textbf{isPathQuest(path)}} Methode überprüft. Somit kann festgestellt werden ob es sich bei dem geprüften Ordner um ein Package handelt.

\htlParagraph{isPathQuest()-Methode:}\\
Da eine Quest eine \textit{ref.cmm}, \textit{description.html} und eine \textit{input.cmm} haben muss, ist es möglich, anhand einer Iteration festzustellen, ob es sich bei diesen Ordnern um Quests handelt.

\htlParagraph{containsQuests()-Methode:}\\
Durch eine einfache Rekursion kann der gesamte Ordner mit seinen Unterordnern aufgeschlüsselt beziehungsweise auf Quests überprüft werden.

\begin{lstlisting}[language=JAVA]
	public static boolean containsQuests(String path){
		//Checking if there is a Quest in the Current Folder
		if(isPathQuest(path))
			return true;
		//Iterate Through all other Folders
		else{
			List<String> subFolders = Quest.ReadFolderNames(path);
			if(subFolders != null){
				for(String subFolder : subFolders){
						return containsQuests(path + File.separator + subFolder);
				}
			}
		}
		return false;
	}
\end{lstlisting}



\begin{figure}[h] 
  \centering
     \includegraphics[width=0.5\textwidth]{./media/images/gui/quest-control-rekursion.png}
  \caption{Schematische Darstellung der Rekursion}
  \label{fig:JTree_Control_Rekursion}
\end{figure}

In der Abbildung \ref{fig:JTree_Control_Rekursion} wird ein beispielhafter Ordnerpfad aufgeschlüsselt. Der Ordner oberhalb des Questordners kann nun als Package erkannt werden.

\subsection{Auslesen einer Quest:}
Damit eine Quest ausgelesen werden kann, ist es notwendig, dass alle benötigten Pfade verfügbar sind. Dazu zählen: der Package-Pfad, und der Ordnername der Quest.

Zuerst werden die Pfade welche übergeben werden gesetzt. Dann werden die enthaltenen Files auf Verfügbarkeit geprüft und der Status dementsprechend in den Variablen angepasst. Wenn die Dateien unvollständig sind, oder der Ofad nicht verfügbar ist, wird die Quest als nicht funktionstüchtig erklärt.

Jetzt kann wenn eine quest.xml vorhanden ist, diese eingelesen werden. 

Wenn Variablen keinen Wert zugewiesen bekommen haben bekommen diese nun einen vordefinierten zugewiesen:
\begin{itemize}
	\item Der Titel wird auf den Ordnernamen gesetzt.
	\item Der Status wird auf selectable gesetzt.
	\item Das Attribut wird auf \textit{exercise} gesetzt.
\end{itemize}