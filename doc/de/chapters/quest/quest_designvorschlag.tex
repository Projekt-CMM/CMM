\section{Erstellung einer Quest}
Zuerst sollten die Dateien erstellt werden welche vom Programm benötigt werden. Diese sind: \textbf{ref.cmm,input.cmm} und \textbf{description.html}.

Hierbei ist zu beachten dass \textbf{input.cmm} Inputdaten für die \textbf{ref.cmm} und die zu testende Datei erstellt.
\begin{lstlisting}[language=C]
//input.cmm
#include <stdio.h>
#include <stdlib.h>

void main(){
	srand(time());
	printf("%d ", rand());	
}

\end{lstlisting}
In der hier gegebenen \textbf{input.cmm} wird eine Zufallszahl erzeugt, diese kann nun in der \textbf{ref.cmm} eingelesen und weiterverarbeitet werden. 
\begin{lstlisting}[language=C]
	//Initialisierung des Zufallsgenerators
	srand(atoi(scanf()));
\end{lstlisting}
In der hier gezeigten Codezeile wird von der \textbf{input.cmm} die Zufallszahl übernommen und im \textbf{ref.cmm} weiterverarbeitet. Die hier gezeigte Codezeile muss nun auch im Programm des Benutzers vorhanden sein. Damit es dadurch für den Benutzer zu keiner Verwirrung kommt, kann man eine \textbf{default.cmm} erstellen. Diese beinhaltet einen vordefinierten Quelltext. Darin sollte auch die vorher gezeigte Zeile enthalten sein.

Der Aufbau einer guten Beschreibung einer Quest ist oftmals nicht einfach. Hierfür sollten mehrere Punkte beachtet werden. 

\begin{itemize}
\item Die Beschreibung sollte einfach und verständlich sein. 
\item Zuerst sollte ein theoretischer Teil beschrieben werden, danach die Aufgabenstellung.
\item Es sollte versucht werden, dass für die Textblöcke die richtigen Klassen des Stylesheets verwendet werden, sodass ein durchgängiges Design entstehen kann.
\item Überschriften mit dem HTML-Tag \textbf{"<h1">} oder \textbf{"<h2">} kennzeichnen.
\end{itemize}