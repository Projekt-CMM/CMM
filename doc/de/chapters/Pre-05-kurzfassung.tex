%!TEX root=../Vorlage_DA.tex
%	########################################################
% 							Zusammenfassung
%	########################################################


%	--------------------------------------------------------
% 	Überschrift, Inhaltsverzeichnis
%	--------------------------------------------------------
\chapter*{Zusammenfassung}
\addcontentsline{toc}{chapter}{Zusammenfassung}



%	--------------------------------------------------------
% 	Inhalt
%	--------------------------------------------------------

C Compact ist eine Entwicklungsumgebung, die besonders auf die Bedürfnisse von angehenden Programmierern ausgelegt ist. Diese Umgebung sollte vor allem in den ersten und zweiten Klassen der HTL Anwendung finden, wo Schülerinnen und Schüler erste und oft schwierige Erfahrungen mit dem Programmieren machen. Während professionelle Entwicklungsumgebungen einen oftmals unübersichtlich breiten Funktionsumfang haben, setzt C Compact auf intuitive und einfache Bedienung.

Die Schüler arbeiten dabei mit einer Sprache, die C sehr ähnlich ist.
Trotz des äußerlich simplen Aufbaus liefert sie jedoch den vollen Umfang einer anspruchsvollen Programmiersprache und den damit verbundenen Programmiertechniken. C Compact ist sowohl zum Lösen einfacher Aufgaben, als auch zum Entwickeln komplexer Algorithmen geeignet und kann bis in die dritte Klasse verwendet werden.

Die Entwicklungsumgebung enthält einen integrierten Debugger, der von Anfang an ein fundiertes Verständnis für die Logik und den Ablauf eines Computerprogramms vermitteln soll. Viele Irrtümer und Verständnisprobleme können bereits durch grafische Veranschaulichungen ausgeschlossen werden. Tritt ein Fehler im Quelltext auf, erhält der Benutzer eine leicht verständliche Fehlerbeschreibung und Hinweise zum Suchen der Fehlerstelle.

