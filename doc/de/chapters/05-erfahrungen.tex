%!TEX root=../Vorlage_DA.tex
%	########################################################
% 					Persönliche Erfahrungen
%	########################################################


%	--------------------------------------------------------
% 	Überschrift, Inhaltsverzeichnis
%	--------------------------------------------------------
\chapter{Persönliche Erfahrungen}

\section{Zusammenarbeit mit dem Institut für\\ Systemsoftware}
\label{sec:coop-jku}
Wir begannen bereits bei einem Pratkikum am Institut im Sommer 2014 mit unserem Projekt. Besonders in der Anfangsphase waren unsere Projektbetreuer an der Universität eine große Hilfe. Wir erlernten schnell die Grundlagen von Compiler- und Interpreterbau und entwarfen das grundsätzliche Aussehen der Benutzeroberfläche.

Während des Schuljahres schicken wir regelmäßig Fortschrittsberichte und aktuelle Ergebnisse an unsere Betreuer und erhalten dann Feedback. Wir führen unser Projekt weitgehend selbstständig durch, können uns bei Problemen aber jederzeit an unsere Ansprechpartner in der Universität wenden.

Kurz vor Weihnachten, am 22. 12. 2014, Verbrachten wir noch einmal einen Vormittag an der JKU, wo wir den Fortschritt unseres Projektes vorstellten und die aktuelle Version von C Compact präsentierten. Wir erhielten positive Rückmeldungen zu unserem Arbeitsfortschritt und hatten anschließend eine interessante Diskussion mit Herrn Professor Balschek, bei der wir neue Ideen zur Implementierung des Aufgabensystems sammelten.

\section{Zusammenarbeit mit unserem Projektbetreuer}
In der Schule wird unser Projekt von Herrn Dipl.-Ing. Franz Matejka betreut. Wir haben jede Woche acht Laboreinheiten, bei denen wir weitgehend selbstständig an unserem Projekt arbeiten. Herr Matejka unterstützt uns wenn notwendig mit Tips und Ideen zum Projekt.

\section{Teamarbeit bei der Softwareentwicklung}
Bei der gemeinsamen Entwicklung eines Softwareprojektes ist es wichtig, zusammenzuarbeiten und sich mit den Teamkollegen abzustimmen. Da wir uns schon vor Beginn des Projektes gut kannten, war die Zusammenarbeit an sich kein großes Problem. Da sich das manuelle Zusammenfügen unterschiedlicher Programmkomponenten jedoch als mühsam und umständlich erwies, begannen wir bald, die Versionsverwaltungssoftware git zu verwenden (siehe Kapitel \ref{sec:dev-git}).

Dadurch wurde die Teamarbeit hinsichtlich Softwareentwicklung wesentlich vereinfacht. Weil wir am Anfang Schwierigkeiten mit git und seinem weiten Umfang an Funktionen hatten, ist dieser Teil unserer Zusammenarbeit eine besonders prägende Erfahrung. Die Verwendung von Versionsverwaltungssoftware ist in der Softwareentwicklung unerlässlich.

Professionelle Zusammenarbeit und die richtige Kommunikation in einem Entwicklerteam sind mit einem Lernprozess verbunden, der in einem großen Projekt dauernd stattfindet.

\section{Versuche mit SchülerInnen}
Bei den Versuchen mit Schülergruppen konnten wir auch außerhalb des Projekts viele hilfreiche Erfahrungen sammeln. Wir mussten die für den Versuch verwendeten Unterrichtsstunden selbst vorbereiten und ein Thema aus der Informatik, wie etwa den Sortieralgorithmus \glqq{}Bubblesort\grqq{} und die zugehörigen Übungen erklären. Anschließend unterstützten wir die Schülerinnen und Schüler der Unterrichtsgruppe beim Lösen der Aufgaben.

Dadurch konnten wir einerseits ein besseres Verständnis für den Programmierunterricht und die Aufgabe des Lehrers erreichen, was auch für weitere Entwicklungen des Projektes hilfreich war. Andererseits ist es auch eine interessante persönliche Erfahrung, nicht unter den Schülern zu sitzen, sondern vor der Klasse zu stehen.