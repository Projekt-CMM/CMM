\documentclass[11pt, oneside]{book}   		% Allgemeine Schriftgröße, Dokumententyp
\input{./includes/htl_definintions.tex}		% Formatierung der Dokuments, Diverse Befehle
\usepackage{multirow}
\usepackage{rotating}
\usepackage{pifont}
\usepackage{changepage}
\usepackage{wrapfig}
\usepackage{pdfpages}
\usepackage{amsmath}
\usepackage{ifthen}

%	########################################################
% 					Autoren		
%	########################################################
\newboolean{fabian}
\newboolean{thomas}
\newboolean{peter}

\setboolean{fabian}{true}
\setboolean{thomas}{true}
\setboolean{peter}{true}

%	########################################################
% 					Allgemeine Informationen			
%	########################################################
\def\htlArbeit{Diplomarbeit }
\def\htlArbeitsthema{C Compact }	% Thema oder Titel der Arbeit
\def\htlArbeitstitel{C Compact }			% Arbeitstitel

%\lstset{basicstyle=\small}


\begin{document}

%	########################################################
% 						Einleitung			
%	########################################################
\pagenumbering{roman}	% Beginn mit römischen Seitenzahlen

%	--------------------------------------------------------
% 	Deckblatt
%	--------------------------------------------------------			
%
%!TEX root=../Vorlage_DA.tex
%	########################################################
% 							Deckblatt
%	########################################################

\newgeometry{top=1cm}
\begin{titlepage}
\title{
%
{\includegraphics[width=0.15\textwidth]{./includes/htl_c_cmyk_rein.pdf}}
%
\\{\normalsize \textbf{Höhere Technische Bundeslehranstalt}}
\\[-0.4em] {\normalsize \textbf{und Bundesfachschule}}
\\[-0.4em] {\normalsize im Hermann Fuchs Bundesschulzentrum}
%
\\\vspace{3cm}{ \textbf{\htlArbeitsthema}}
%
\\\vspace{0.5cm} {\Large \textsc{\htlArbeit}}
%
\\\vspace{5cm}
%
{\normalsize \emph{Ausgeführt im Schuljahr 2014/2015 von:}}
\\{\normalsize Thomas Pointhuber, 5CHELS}
\\[-0.4em] {\normalsize Peter Wassermair, 5CHELS}
\\[-0.4em] {\normalsize Fabian Hummer, 5CHELS}
\\[1.0em] {\normalsize \emph{Betreuer:}}
\\{\normalsize Dipl.-Ing. Franz Matejka}
}		
%
\maketitle
\end{titlepage}
\restoregeometry
\pagebreak


%	--------------------------------------------------------
% 	Arbeitstitel
%	--------------------------------------------------------		
%!TEX root=../Vorlage_DA.tex
%	########################################################
% 							Arbeitstitel
%	########################################################


%	--------------------------------------------------------
% 	Überschrift, Inhaltsverzeichnis
%	--------------------------------------------------------
\chapter*{Arbeitstitel: \newline \htlArbeitstitel}



%	--------------------------------------------------------
% 	Bearbeiter
%	--------------------------------------------------------
\htlParagraph{Bearbeiter:}


%	--------------------------------------------------------
% 	Beteiligte Firmen
%	--------------------------------------------------------
\htlParagraph{An der \htlArbeit beteiligte Firmen:}

\renewcommand{\arraystretch}{1.5}
\begin{tabularx}{1\textwidth}{@{} l X @{}}

\emph{Firma:} & Johannes Kepler Universit\"at\\
\emph{Adresse:} & Altenberger Straße, 69\\
\emph{Plz, Ort:} & 4040, Linz\\
\emph{Kontaktperson:} & Prof. Dr. Dr. h.c. Hanspeter Mössenböck\\
\emph{Telefon:} & + 43-732-2468-4340\\
%\emph{E-Mail:} & max.mustermann@firmenname.com\\

\end{tabularx}


%	--------------------------------------------------------
% 	Eidesstattliche Erklärung
%	--------------------------------------------------------			
\input{./chapters/Pre-03-erklaerung.tex}

%	--------------------------------------------------------
% 	Inhaltsverzeichnis
%	--------------------------------------------------------			
\tableofcontents

%	--------------------------------------------------------
% 	Vorwort
%	--------------------------------------------------------	
%!TEX root=../Vorlage_DA.tex
%	########################################################
% 							Vorwort
%	########################################################


%	--------------------------------------------------------
% 	Überschrift, Inhaltsverzeichnis
%	--------------------------------------------------------
\chapter*{Vorwort}
\addcontentsline{toc}{chapter}{Vorwort}


%	--------------------------------------------------------
% 	Inhalt
%	--------------------------------------------------------

Wir möchten uns bei unseren Projektbetreuern am Institut für Systemsoftware der Johannes Kepler Universität Linz bedanken, die uns bei unserem Praktikum zu Beginn des Projektes und während des Jahres unterstützt haben. Wir konnten einen interessanten Einblick in den Alltag und die Arbeitsweise an der Universität erhalten.

Besonderer Dank gilt unserem Projektbetreuer, Herrn Matejka, der uns während des Projektes stets mit Rat und Tat zur Seite gestanden ist und uns mit seiner Erfahrung als Lehrer einen anderen Blickwinkel auf die Entwicklung von C Compact eingebracht hat.

Einen wertvollen Beitrag haben die Schülerinnen und Schüler geleistet, die bei den Versuchen beteiligt waren und uns somit laufend auf die Schwächen und Probleme unserer Entwicklungsumgebung aufmerksam gemacht, uns aber auch das Potential und die Vorteile von C Compact gezeigt haben.

Schließlich möchten wir noch den Lehrern danken, die uns jeweils mehrere Unterrichtsstunden zur Verfügung gestellt haben, um diese Tests zu ermöglichen.

Wir hoffen, dass C Compact seine Anwendung im Unterricht nicht nur während dieser Versuche gefunden hat. Die Schülerinnen und Schüler konnten von von den Features und der anschaulichen Oberfläche von C Compact profitieren und gaben uns durchwegs positive Rückmeldungen. Damit C Compact jederzeit und von jedem verwendet werden kann, haben wir beschlossen, diese Entwicklungsumgebung unter der GNU General Public License öffentlich zur Verfügung zu stellen.

%	--------------------------------------------------------
% 	Zusammenfassung
%	--------------------------------------------------------		 	  
%!TEX root=../Vorlage_DA.tex
%	########################################################
% 							Zusammenfassung
%	########################################################


%	--------------------------------------------------------
% 	Überschrift, Inhaltsverzeichnis
%	--------------------------------------------------------
\chapter*{Zusammenfassung}
\addcontentsline{toc}{chapter}{Zusammenfassung}



%	--------------------------------------------------------
% 	Inhalt
%	--------------------------------------------------------

C Compact ist eine Entwicklungsumgebung, die besonders auf die Bedürfnisse von angehenden Programmierern ausgelegt ist. Diese Umgebung sollte vor allem in den ersten und zweiten Klassen der HTL Anwendung finden, wo Schülerinnen und Schüler erste und oft schwierige Erfahrungen mit dem Programmieren machen. Während professionelle Entwicklungsumgebungen einen oftmals unübersichtlich breiten Funktionsumfang haben, setzt C Compact auf intuitive und einfache Bedienung.

Die Schüler arbeiten dabei mit einer Sprache, die C sehr ähnlich ist.
Trotz des äußerlich simplen Aufbaus liefert sie jedoch den vollen Umfang einer anspruchsvollen Programmiersprache und den damit verbundenen Programmiertechniken. C Compact ist sowohl zum Lösen einfacher Aufgaben, als auch zum Entwickeln komplexer Algorithmen geeignet und kann bis in die dritte Klasse verwendet werden.

Die Entwicklungsumgebung enthält einen integrierten Debugger, der von Anfang an ein fundiertes Verständnis für die Logik und den Ablauf eines Computerprogramms vermitteln soll. Viele Irrtümer und Verständnisprobleme können bereits durch grafische Veranschaulichungen ausgeschlossen werden. Tritt ein Fehler im Quelltext auf, erhält der Benutzer eine leicht verständliche Fehlerbeschreibung und Hinweise zum Suchen der Fehlerstelle.



%	--------------------------------------------------------
% 	Abstract
%	--------------------------------------------------------		
%!TEX root=../Vorlage_DA.tex
%	########################################################
% 							Abstract
%	########################################################


%	--------------------------------------------------------
% 	Überschrift, Inhaltsverzeichnis
%	--------------------------------------------------------
\chapter*{Abstract}
\addcontentsline{toc}{chapter}{Abstract}



%	--------------------------------------------------------
% 	Inhalt
%	--------------------------------------------------------
C Compact is a simple but powerful development environment for a programming language similar to C. It is especially adjusted to the needs of programming beginners who often have troubles understanding the way a program works and the logic it is based on.

In contrast to professional IDEs, C Compact can be used easily and in an intuitive way. However, it also provides the full range of features that are needed for programming lessons at technical schools. C Compact can be used to aquire simple tasks as well as to develop complex algorithms.

This IDE is intended for the use in programming lessons in the first two grades of Austrian technical colleges (HTL). Though, C Compact is not limited to this application. It can also be used for advanced training or programming courses as well as by hobby programmers.

C Compact is distributed under the terms of GNU General Public License and is therefore free software. 



%	########################################################
% 						Arbeit			
%	########################################################

%	--------------------------------------------------------
% 	INFO: Zitieren, Abbildungen, Listing
%	-------------------------------------------------------	
%\input{./chapters/Pre-XX-beispiele.tex} 	% Nur zur Info

%	--------------------------------------------------------
% 	Aufgabenstellung / Pflichtenheft
%	--------------------------------------------------------		
%\input{./chapters/01-aufgabenstellung.tex}
\pagenumbering{arabic}	% Beginn mit arabischen Seitenzahlen

\iffabian
%!TEX root = "../../DA_GUI.tex"

\chapter{Einleitung}

\section{Die Projektidee}
Für Firmen ist es schwierig, gut ausgebildete Programmierer zu finden, eine Reihe von Stellen bleiben unbesetzt. Etliche Firmen werden dadurch daran gehindert, zu expandieren oder wichtige Aufträge anzunehmen.
Ein wesentlicher Grund für diese Misere ist, dass das Erlernen einer Programmiersprache eine anspruchsvolle Tätigkeit ist. Besonders Anfänger stoßen oft auf Schwierigkeiten.

Im Rahmen des Projektes C Compact wurde eine speziell für die Ausbildung konzipierte Entwicklungsumgebung erarbeitet. Diese soll den Einstieg in C durch intuitive Bedienung und anschauliche Darstellung von Programmabläufen erleichtern.

\section{Projektorganisation}

Das Projekt C Compact wurde in Zusammenarbeit mit dem Institut für Systemsoftware\footnote{http://ssw.jku.at/} der Johannes Kepler Universität Linz durchgeführt. Jeder Projektschüler ist für einen Bereich des Projektes verantwortlich und wird dabei von einem Projektbetreuer der Universität unterstützt. Das Diplomprojekt selbst wird von Herrn Dipl.Ing. Franz Matejka betreut. Tabelle \ref{tab:intro-work-1} zeigt die Aufgabenverteilung, wie sie zu Beginn des Projektes festgelegt wurde.

\def\arraystretch{1.4}
\begin{table}
\begin{tabular}{|l|l|l|}
\hline
\textbf{Projektschüler}&\textbf{Projektbetreuer}&\textbf{Aufgabenbereich}\\
\hline
Thomas Pointhuber & Univ.Prof. Hanspeter Mössenböck & Compiler \\
Peter Wassermair & Dipl.Ing. Matthias Grimmer & Interpreter \\
Fabian Hummer & Univ.Prof. Günther Blaschek & Benutzeroberfläche\\
\hline
\end{tabular}
\caption{Ursprüngliche Aufgabeneinteilung}
\label{tab:intro-work-1}
\end{table}

\section{Terminübersicht}

Der Großteil der Projektarbeit wurde als Freizeitleistung durchgeführt. Eine genaue Aufstellung der Arbeitszeiten ist in Anhang \ref{} zu finden.

Die Entwicklung von C Compact wurde in den Sommerferien 2014 --- in Form eines Praktikums der Projektschüler am Institut für Systemsoftware --- begonnen. Bereits im Vorfeld wurden einige Projektanforderungen festgelegt (siehe Kapitel \ref{sec:intro.anf}). Das Praktikum begann am 7. Juli 2014 und dauerte 2 bzw. 4 Wochen\footnote{Praktikum an der JKU von 7.7.2014 bis 18.7.2014, bzw. Peter Wassermair von 7.7.2014 bis 1.8.2014}. Wir erlernten zu Beginn die Grundlagen und Techniken von Compiler und Interpreter und konnten schnell mit der Entwicklung starten.

Am 22. Dezember 2014 wurde den Projektbetreuern an der JKU der Projektfortschritt vorgestellt, daraufhin wurden weitere Entwicklungsschritte zum Questsystem und zum Compiler besprochen.

Eine abschließende Besprechung am Institut für Systemsoftware ist für Juni 2015 vorgesehen.

Im Laufe des Schuljahres wurde C Compact mit unterschiedlichen Funktionen und Bestandteilen erweitert. Tabelle \ref{tab:intro-milestones} zeigt eine grobe Einteilung der Projektarbeit in einzelne Meilensteine.

\begin{table}[h!]
\def\arraystretch{1.6}
\begin{tabularx}{\columnwidth}{|l|XXX|}
\hline
  \textbf{Zeitraum}&\textbf{Thomas Pointhuber}&\textbf{Peter Wassermair}&\textbf{Fabian Hummer}\\
  \hline
  Sommerferien 2014\footnote{Praktikum an der JKU von 7.7.2014 bis 18.7.2014, bzw. Peter Wassermair von 7.7.2014 bis 1.8.2014} & Compiler & Interpreter & Benutzeroberfläche, Entwicklugnsumgebung \\
  Sept. - Dez. 2014 & Implementierung weiterer Sprachfunktionen & Grundgerüst des Aufgabensystens, Dateistruktur & Verbesserung der Benutzeroberfläche, Versuche mit Schülergruppen \\
  Januar - März 2015 & Testen und Dokumentieren von Sprachfunktionen & Integration des Aufgabensystems in die Entwicklungsumgebung & Abschließen der Entwicklungsumgebung, Hilfe bei Questsystem\\
  April - Mai 2015 & \multicolumn{3}{c}{Fehlerbehebung, kleine Erweiterungen und Dokumentation}\\
  \hline
\end{tabularx}
\caption{Meilensteine in der Entwicklung von C Compact}
\label{tab:intro-milestones}
\end{table}


\section{C Compact im Überblick}

\subsection*{Die Sprache}
Die Benutzer arbeiten bei C Compact mit einer Programmiersprache, die C sehr ähnlich ist. Trotz des äußerlich simplen Aufbaus kann sie den vollen Umfang einer anspruchsvollen Programmiersprache und die damit verbundenen Programmiertechniken vermitteln. Siehe Kapitel \ref{} und \ref{}.
%TODO add ref

Bekannte C-Bibliotheken wie \glqq{}stdlib.h\grqq{}sind in C Compact ebenfalls vorhanden, um auch komplexere Programme einfach umsetzen zu können. Außerdem wurde ein einfacher Präprozessor implementiert, um Projekte mit mehreren Dateien zu ermöglichen. Siehe Kapitel \ref{}
%TODO add ref

\subsection*{Die Umgebung}
Viele Irrtümer und Verständnisprobleme, die bei angehenden Programmierern immer wieder auftreten, sollen bereits durch die Entwicklungsumgebung selbst ausgeschlossen werden. Deshalb ist die Benutzeroberfläche ein essenzieller Bereich von C Compact. %Wir haben häufig auftretende Missverständnisse und Denkfehler analysiert und Features erarbeitet, die eben jene Probleme ausschließen sollen.

%TODO add ref to common mistakes

\begin{figure}[h!]
	\centering
	\includegraphics[width=0.7\textwidth]{./media/images/gui/main/CCompactAlpha1-4-5-guimain.png}
	\caption{Die Benutzeroberfläche von C Compact}
	\label{fig:intro-over-gui}
\end{figure}

Ein wichtiger Bestandteil der Benutzeroberfläche ist der integrierte Debugger (in Abbildung \ref{fig:intro-over-gui} rechts), der speziell für die Bedürfnisse von Anfängern ausgelegt ist. Während das Programm Schritt für Schritt abgearbeitet wird, sind alle Variablen und der Call Stack in übersichtlicher Form dargestellt. In der Variablentabelle werden Wertänderungen farbig hervorgehoben. Dadurch soll ein fundiertes Verständnis für den Programmablauf und die Logik dahinter entstehen.
Ein weiterer daraus resultierender Vorteil ist, dass Schüler von Anfang an mit einem Debugger arbeiten und ähnliche Tools später auch in anderen Entwicklungsumgebungen verwenden werden. Der Debugger wird in Kapitel \ref{sec:deb} beschrieben.

Um sicherzustellen, dass die Entwicklungsumgebung wirkungsvoll ist und zuverlässig funktioniert, wurde sie regelmäßig mit Gruppen der zweiten Klassen der HTL Braunau getestet. Bei diesen Versuchen lösten die Schülerinnen und Schüler komplexe Aufgaben mithilfe von C Compact und bewerten Benutzerfreundlichkeit und Bedienbarkeit nachher in einem Fragebogen. Siehe Kapitel \ref{sec:sci-trial-intro}.

\subsection*{Integration im Unterricht}
Um den Programmierunterricht zu verbessern und effektiver zu gestalten, wurde auf aktuellen und bewährten Unterrichtsmethoden aufgebaut werden. C Compact soll Lehrer wie Schüler unterstützen. Zu diesem Zweck wurden einige FSST-Lehrer (Fachspezifische Softwaretechnik) unserer Schule zum Aufbau und Ablauf ihrer Stunden befragt, um Einsatzmöglichkeiten für C Compact zu erkennen. Gewonnene Erkenntnisse flossen laufend in den Entwicklungsprozess von C Compact ein. Siehe Kapitel \ref{}.
%TODO add ref

C Compact enthält außerdem ein Aufgaben- und Belohnungssystem (auch genannt Questsystem). Mit diesem System können Schüler kleine und größere Aufgaben, die in Themenblöcke gegliedert sind, lösen. Eine ausgewählte Aufgabe wird bei der Fertigstellung automatisch auf Richtigkeit überprüft. Durch den individuellen Fortschritt soll selbstständiges Lernen gefördert werden. Um kleine Erfolgserlebnisse zu vermitteln, erhalten die Benutzer für gelöste Aufgaben Auszeichnungen. Siehe Kapitel \ref{}.
%TODO add ref

\subsection*{Verwendung und Erhältlichkeit}
Unseren Recherchen zufolge gibt es derzeit keine vergleichbare Entwicklungsumgebung für die Ausbildung. Projekte mit ähnlichen Zielen besitzen oft einen rein spielerischen Charakter oder unterstützen nur einen sehr grundlegenden Befehlssatz. In C Compact dagegen ist es möglich, komplexere Algorithmen zu entwickeln, zu visualisieren, und diese auch in späteren Projekten einzusetzen.

Wir hoffen zwar, dass C Compact vor Allem im Unterricht Verwendung findet, sind aber auch für viele andere Anwendungsbereiche offen, wie etwa bei Kursen und Schulungen oder für Hobbyprogrammierer. Deshalb haben wir beschlossen, C Compact unter der GNU General Public License\footnote{http://www.gnu.org/copyleft/gpl.html} öffentlich zur Verfügung zu stellen. Das gesamte Projekt kann auf GitHub\footnote{https://github.com/Projekt-CMM/CMM} angesehen werden.

\begin{figure}[h!]
	\centering
	\includegraphics[width=0.3\textwidth]{./media/images/intro/GPLv3-Logo.png}
	\caption{Logo der GNU General Public License}
\end{figure}

\section{Anforderungen an C Compact}
\label{sec:intro.anf}
Für die Anwendung im Unterricht und insbesondere für die Anwendung von unerfahrenen Programmierern muss die Entwicklungsumgebung C Compact einige besondere Eigenschaften aufweisen. Die Anforderungen an diese Entwicklungsumgebung unterscheiden sich von den Anforderungen an professionelle Tools.

Die Anforderungen für die erste Version von C Compact --- die bereits im Sommer 2014 entwickelt wurde --- wurden von den Projektbetreuern an der JKU gestellt. Herr Professor Mössenböck erstellte ein Anforderungsdokument für Compiler und Interpreter, das im Laufe des Projektes erweitert wurde (siehe Anhang \ref{app:anf-comp}). Herr Professor Blaschek definierte vor Projektbeginn die grundlegenden Anforderungen an die Benutzeroberfläche von C Compact und legte ihren grundlegenden Aufbau fest (siehe Anhang \ref{app:anf-gui}). Im Laufe des Projektes wurde die Benutzeroberfläche ausgebaut.

\subsection*{Anforderungen an das Programm an sich}
Für die Anwendung im Unterricht --- insbesondere als Hilfestellung für unerfahrene Benutzer --- sind folgende Punkte besonders wichtig:
\begin{itemize}
\item \textbf{Das Programm soll auf möglichst vielen Betriebssystemen funktionieren}\\
Läuft ohne Probleme auf Windows 7, Windows 8, Mac OSX, Ubuntu 14.04, Ubuntu 14.10, Linux Mint 17.1\\
Leichte Probleme auf Windows 10 (Technical Preview)
\item \textbf{Das Programm soll einfach zu installieren sein}\\
Eine Installation ist nicht erforderlich.
\item \textbf{Das Programm soll einfach erhältlich sein}\\
C Compact ist freie Software und kann kostenlos heruntergeladen werden.
\end{itemize}

\subsection*{Anforderungen an den Compiler und den Interpreter}
Compiler und Interpreter sind grundlegende Bausteine von C Compact. Hier sind besonders Stabilität und Effizienz gefragt.
\begin{itemize}
\item \textbf{Der Compiler muss stabil laufen}\\
Durch ausführiche Tests wurden sehr viele Fehler behoben. Der Compiler läuft praktisch ohne Probleme.
\item \textbf{Die verwendete Sprache soll C möglichst ähnlich sein}\\
Es gibt leichte Unterschiede zu C, auf die im Unterricht eventuell hingewiesen werden muss. Einige der Änderungen bringen aber auch Voreteile mit sich.
\item \textbf{Die verwendete Sprache soll gegenüber C keine wesentlichen Einschränkungen enthalten}\\
C Compact kann bis in die dritte Klasse problemlos verwendet werden. Nur bei einigen sehr fortgeschrittenen Themen muss auf professionelle Compiler zurückgegriffen werden.
\end{itemize}

\subsection*{Anforderungen an die Benutzeroberfläche}
Die Benutzeroberfläche ist die Schnittstelle zwischen Mensch und Maschine. Um Programmieranfänger nicht zu überfordern, muss die Oberfläche möglichst einfach und intuitiv zu bedienen sein.
\begin{itemize}
\item \textbf{C Comapct muss einfach zu Bedienen sein}\\
Durch eine Reihe von Versuchen mit Schülerinnen und Schülern der ersten und zweiten Klassen konnten wir die Benutzeroberfläche immer wieder optimieren, sodass C Compact nun sehr übersichtlich und angenehm zu Bedienen ist.
\item \textbf{C Compact soll den Schülerinnen und Schülern helfen, Programmabläufe besser zu verstehen}\\
Der intuitiv zu bedienende Debugger von C Compact zeigt den Ablauf und die Logik eines Programms sehr anschaulich. In unseren Versuchen haben wir herausgefunden, dass C Compact tatsächlich eine Verbesserung des FSST-Unterrichts darstellt.
\item \textbf{Die Fehlermeldungen des Compilers müssen verständlich und ersichtlich sein}\\
Wenn ein Fehler auftritt erhält der Benutzer sowohl eine Angabe, wo der Fehler aufgetreten ist, als auch eine ausführliche Beschreibung des Fehlers und möglicher Fehlerursachen.
\end{itemize}

\subsection*{Anforderungen an das Questsystem}
Mit dem Questsystem können die Schülerinnen und Schüler Probleme zu einem bestimmten Thema selbstständig erarbeiten. Die Schülerinnen und Schüler lösen Aufgaben, die von C Compact gestellt werden. Das Programm überprüft auch, ob die Lösungen der Schülerinnen und Schüler korrekt ist. Für korrekte Lösungen werden Auszeichnungen vergeben.
\begin{itemize}
\item \textbf{Aufgaben müssen in Themenbereiche strukturierbar sein}\\
Aufgaben sind in beliebig viele Pakete und Bereiche --- auch mit mehreren Ebenen --- strukturierbar. Lehrer können Aufgabensammlunglen erstellen, die an ihren Unterricht angepasst sind
\item \textbf{Die Aufgaben sollten eine Verbesserung für den Unterricht darstellen}\\
Bei einem Versuch mit einer Gruppe der 1AHLES hat sich gezeigt, dass die Schüler gerne Aufgaben aus dem Questsystem erarbeiten. Durch Auszeichnungen und Anzeige des Fortschritts in einem bestimmten Themenbereich können die Schüler gezielt motiviert werden.
\end{itemize}

%!TEX root = "../../DA_GUI.tex"

%	--------------------------------------------------------
% 	Aufbau der Benutzeroberfläche
%	--------------------------------------------------------

\chapter{Elemente der Benutzeroberfläche}
In diesem Kapitel wird der grundlegende Aufbau aller Fenster und Benutzeroberflächen von C Compact beschrieben - sowohl die konzeptuelle Konstruktion als auch die Implementierung.
\section{Das Konzept}
Die Erscheinung und Bedienung von C Compact folgt durchgehend der grundlegenden Idee, eine einfach und intuitiv zu bedienende Entwichlungsumgebung zu schaffen. Das bedeutet, Features möglichst Zielgruppenorientiert zu integrieren und überflüssige Funktionen zu vermeiden. Durch besondere Rücksicht auf Vollständigkeit und logische Bedienvorgänge haben wir ein durchgängig logisches Bedienerlebnis angestrebt.
Konkret haben wir uns dabei an folgende Ŕegeln gehalten:
\begin{itemize}
\item Das Erscheinen von Elementen in der Benutzeroberfläche soll logisch begründbar und verständlich vermittelbar sein
\item Funktionen, die im FSST-Unterricht der ersten und zweiten Klassen wahrscheinlich nicht benötigt werden, werden vermieden oder sind standardmäßig deaktiviert; Bedienelemente werden also auf das wesentlichste reduziert
\item Wir sind der Meinung, dass eine intuitiv zu bedinenede Oberfläche nicht durch Animationen und große Symbole erzeilt werden kann, sondern durch einfache, bereits bekannte Konzepte. So sind in C Compact bekannte Elemente wie beispielsweise eine Menüleiste mit gewohnten Datei- und Dokumentoperationen zu finden (Siehe Kapitel \ref{sec:gui-main-menu}).
%TODO reference GUImain
\end{itemize}

\subsection{Daraus resultierende Einschränkungen}
Durch diese Reduktion der gesamten Oberfläche ist C Compact auf einen bestimmten Zweck, also die Ausbildung, beschränkt. Das sehen wir allerdings nicht als Nachteil, sondern als besondere Stärke. Mit C Compact wollen wir eine Entwicklungsumgebung schaffen, die Anfänger nicht überfordert und ihnen hilft, sich auf das wesentliche zu konzentrieren.
Das Konzept einer anfängerfreundlichen Entwichlungsumgebung ist mit dem Aufbau einer Umgebung für die professionelle Entwicklung nicht vereinbar. C Compact soll aber auf späteres Arbeiten mit komplexeren Programmen vorbereiten. Schüler, die ihre erste Programmiersprache mit C Compact erlernen, kennen bereits den Umgang mit einem Debugger und haben ein tieferes verständnis für den Ablauf von Programmen.

%TODO add window desc. for launcher, quest seleter, package selecter
\section{Elemente und Fenster Der Benutzeroberfläche}
Neben dem Hauptfenster, das den Kern der Benutzeroberfläche darstellt, gibt es eine Reihe von kleineren Aktionsfenster, Dialogen und Einstellungsfenstern, die zum reibungslosen Ablauf bei der Bedienung von C Comapct beitragen. Das Hauptfenster selbst wird in Kapitel \ref{sec:gui-main} beschrieben.

\subsection{Launcher}
...

\subsection{Einstellung der Sprache}
Wenn C Compact zum ersten Mal gestartet wird, wird zu allererst dieses fenster angezeigt. Der Benutzer wird gebeten, eine Sprache zu wählen. Später kann die Sprache über das Menü ,,Datei'' im Hauptfenster (Siehe Kapitel \ref{sec:gui-main-menu-file}) geändert werden. Die Änderungen werden erst wirksam, wenn der Benutzer die neue Sprache mit ,,OK'' bestätigt. Dann wird C Compact neu gestartet.
%TODO ref translations, Sprache

\begin{figure}[htp]
\centering
\includegraphics[width=0.3\textwidth]{./media/images/gui/elements/Bildschirmfoto-Sprache.png}
\caption{Auswahl der Sprache}
\label{fig:win-lang}
\end{figure}

Dieses Fenster ist in der Klasse \textbf{GUILanguage} im Package \textbf{at.jku.ssw.cmm.gui.properties} implementiert. 

\subsection{Allgemeine Einstellungen}
Dieses fenster enthält allgemeine Optionen zum Haupfenster der Benutzeroberfläche.
%TODO TODO TODO hier weitermachen

%TODO ref GUImainSettings



%!TEX root = "../../DA_GUI.tex"
%	########################################################
% 				Allgemeiner Teil (Theorie)
%	########################################################


%	--------------------------------------------------------
% 	Aufbau der Benutzeroberfläche
%	--------------------------------------------------------


\chapter{Die Benutzeroberfläche der Entwicklungsumgebung}
\label{sec:gui-main}
Zu diesem Bereich zählen alle Elemente des Hauptfensters. Der Grundlegende Aufbau wurde bereits vor dem Projektstart entworfen und in unserem Praktikum im Sommer 2014 implementiert. Vom Herbst 2014 bis Ende Januar 2015 wurde der Aufbau und die Bedienung dieser Oberfläche erweitert und verbessert. Eine besondere Hilfestellung dabei waren die Versuche mit SchülerInnen der ersten und zweiten Klassen, die an unseren Versuchen teilgenommen haben.
%TODO reference versuche

%TODO list parts of GUI

%TODO refer to package "at.jku.ssw.cmm.gui.*"

%TODO describe how to launch C Compact (from class...)

%TODO add desc for GUIExecutable

\section{Basispanel des Hauptfensters}
\ref{sec:gui-main}
Die Entwicklungsumgebung selbst ist in zwei Hälften geteilt. Im Linken Teil befinden sich die Textfelder für den Quelltext und für die Ein- und Augabe des CMM-Programmes. Im rechten Teil befinden sich Registerkarten mit unterschiedlichen Aktionsfeldern.

Die beiden Teile sind durch einen Balken getrennt, der mit der Maus verschoben werden kann. Diese Funktionen wurden mit dem Swing-Element JSplitPane\footnote{http://docs.oracle.com/javase/7/docs/api/javax/swing/JSplitPane.html} implementiert.

\begin{figure}[htbp] 
  \centering
     \includegraphics[width=0.7\textwidth]{./media/images/gui/main/CCompactAlpha1-4-5-guimain.png}
  \caption{Aufbau des Hauptfensters}
  \label{fig:gui-main-1}
\end{figure}

\subsection{Implementierung}
\label{sec:gui-main-impl}
Das Hauptfenster selbst ist in der Klasse \textbf{GUImain} implementiert. Diese Klasse implementiert, da sie ein Hauptfenster regelt, das Interface \textbf{GUIExecutable}. Zum Starten der Entwicklungsumgebung muss immer ein Objekt dieser Klasse angelegt werden. Im Konstruktor muss eine Referenz auf ein Einstellungsobjekt der Klasse \textbf{GUImainSettings} übergeben werden. In diesem Objekt werden alle Einstellungen des Benutzers gespeichert.

Wenn C Compact ohne Benutzerprofil gestartet werden soll, wird als Parameter \textbf{null} übergeben. In diesem Fall können keine Quests gestartet werden, alle anderen Funktionen wie etwa der Debugger sind aber verfügbar.
%TODO ref GUImainSettings

\begin{lstlisting}[language=JAVA]
	/**
	 * Constructor requires specific configuration for the window (settings)
	 * 
	 * @param settings Configuration object for the main GUI.
	 */
	public GUImain(GUImainSettings settings) {
		this.settings = settings;
	}
\end{lstlisting}

Um das Hauptfenster zu initialisieren und C Compact zu starten, muss danach die Methode \textbf{void start(boolean test)} aufgerufen werden. Diese Methode wurde vom Interface GUIExecutable übernommen. Als Parameter sollte normalerweise \textbf{false} übergeben werden. Wenn \textbf{true} übergeben wird, wird das Fenster gleich nach dem Initialisieren geschlossen. Diese Funktion wird für automatisierte Tests mit ANT verwendet.
%TODO ref ant tests

\begin{lstlisting}[language=JAVA]
	/**
	 * Initializes and launches the main GUI and therefore the main part of the
	 * program. <b>This is not the static main function!</b> Running this method
	 * requires calling a constructor with configuration data before (see above
	 * in code).
	 * 
	 * @param test
	 *            TRUE if program shall exit after init (for GUI test)
	 */
	@Override
	public void start(boolean test) {
		.
		.
		.
			
		// Initialize the window
		this.jFrame = new JFrame(VERSION);
		this.jFrame.setDefaultCloseOperation(JFrame.DO_NOTHING_ON_CLOSE);
		this.jFrame.getContentPane().setPreferredSize(new Dimension(800, 500));
		this.jFrame.setMinimumSize(new Dimension(600, 400));
		
		.
		.
		.
		
		// Exit if this was just a test
		if (test)
			System.exit(0);
	}
\end{lstlisting}

Außerdem implementiert die Klasse GUImain die Methode \textbf{saveAndDispose()} des Interfaces \textbf{GUIExecutable}. Diese Methode speichert alle Einstellungen und Daten des Benutzers und beendet das Programm.

\begin{lstlisting}[language=JAVA]
	@Override
	public void saveAndDispose() {
		this.getSaveManager().directSave();
		this.getSettings().writeXMLsettings();
		this.dispose();
	}
\end{lstlisting}

\section{Linker Teil der Benutzeroberfläche}
Zum linken Bereich der Benutzeroberfläche gehören Textfelder für Ein- und Ausgabedaten des CMM-Programmes, das Textfeld für den Source Code, sowie ein farbiges Panel, das den aktuellen Zustand der Benutzeroberfläche visualisiert.

Die größe der Elemente kann angepasst werden, indem die Grenzbalken mit der Maus verschoben werden. In Kapitel \ref{sec:gui-main} wurde das JSplitPane bereits erwähnt. Ein SplitPane enthält immer zwei Elemente\footnote{https://docs.oracle.com/javase/tutorial/uiswing/components/splitpane.html}, die durch den verschiebbaren Balken getrennt werden. In diesem Fall werden zwei verschachtelte SplitPanes verwendet. Das äußere Panel enthält einerseits das Zustandspanel (siehe Kapitel \ref{sec:gui-main-left-zust}) und das Textfeld für den Source Code (siehe Kapitel \ref{sec:gui-main-left-code}), andererseits auch ein zweites SplitPane, das Ein- und Ausgabefeld trennt (Kapitel \ref{sec:gui-main-left-io}).
%TODO add doc for JSplitPane

\subsection{Allgemeine Implementierung}
Der gesamte linke Teil der Benutzeroberfläche ist in der Klasse \textbf{GUIleftPanel} im Packet \textbf{at.jku.ssw.cmm.gui} implementiert. Ein Objekt dieser Klasse wird beim Starten von C Compact von \textbf{GUImain} initialisiert.

Für die Kommunikation mit anderen Klassen wurden einige Methoden implementiert, die wichtigsten werden in Tabelle \ref{tab:gui-main-left-methods}

\def\arraystretch{1.6}
\begin{table}
\begin{tabularx}{\columnwidth}{l|p{9cm}}
\textbf{Name}&\textbf{Beschreibung}\\
\hline
\hline
public JSplitPane init(...)&Initialisiert diesen Teil der Benutzeroberfläche und gibt eine Referenz auf das Hauptelement zurück\\
\hline
public void setReadyMode()&\multirow{4}{*}{\parbox{9cm}{Ändert den Modus des Zustandpanels, Siehe Kapitel \ref{sec:gui-main-left-zust}}}\\
public void setErrorMode(...)&\\
public void setRunMode()&\\
public void setPauseMode()&\\
\hline
public void lockInput()&\multirow{2}{*}{\parbox{9cm}{Sperrt und ermöglicht Eingabe in eines der Textfelder, Siehe Kapitel \ref{sec:gui-main-left-zust}}}\\
public void unlockInput()&\\
\hline
public void setOrientation(...)&Ändert die Anordnung der Elemente in diesem Teil der Benutzeroberfläche, Siehe Kapitel \ref{sec:gui-main-left-ord} und ...\\
\hline
public void updateFontSize()&Ändert die Schriftgröße in allen Textfeldern, Siehe Kapitel ...
\end{tabularx}
\caption{Die wichtigsten Methoden der Klasse \textbf{GUIleftPanel}}\label{tab:gui-main-left-methods}
\end{table}
%TODO ref GUIProperties

Die Textfelder in diesem Teil der Benutzeroberfläche werden mit statischen Methoden initialisiert, die in der Klasse \textbf{InitLeftPanel} im Package \textbf{at.jku.ssw.cmm.gui.init} zu finden sind. Im Folgenden wird bei dem jeweiligen Element erwähnt, in welcher Methode es initialisiert wird.

\subsection{Anordnung der Elemente}
\label{sec:gui-main-left-ord}

Um C Compact für alle Bildschirmformate optimal anzupassen, sind in den Einstellungen unterschiedliche Optionen zur Anordnung der Elemente im linken Panel zu finden. Die Abbildungen \ref{fig:gui-main-left-o1} bis \ref{fig:gui-main-left-o4} zeigen die möglichen Konfigurationen.
%TODO ref GUIProperties

\begin{figure}
\centering
	\begin{minipage}{0.45\textwidth}
		\centering
		\includegraphics[width=1.0\textwidth]{./media/images/gui/main/orientations/CCompact-gui-1.png}
		\caption{Classic Vertical}\label{fig:gui-main-left-o1}
	\end{minipage}\hfill
	\begin{minipage}{0.45\textwidth}
		\centering
		\includegraphics[width=1.0\textwidth]{./media/images/gui/main/orientations/CCompact-gui-2.png}
		\caption{Classic Horizontal}\label{fig:gui-main-left-o2}
	\end{minipage}
\end{figure}

\begin{figure}
\centering
	\begin{minipage}{0.45\textwidth}
		\centering
		\includegraphics[width=1.0\textwidth]{./media/images/gui/main/orientations/CCompact-gui-3.png}
		\caption{Widescreen Central}\label{fig:gui-main-left-o3}
	\end{minipage}\hfill
	\begin{minipage}{0.45\textwidth}
		\centering
		\includegraphics[width=1.0\textwidth]{./media/images/gui/main/orientations/CCompact-gui-4.png}
		\caption{Widescreen Left}\label{fig:gui-main-left-o4}
	\end{minipage}
\end{figure}

Wie zu Beginn dieses Kapitels bereits beschrieben wurde, ist der linke Teil der Benutzeroberfläche mit zwei verschachtelten JSplitPanes organisiert. Für einfache Änderungen am Aufbau der benutzeroberfläche reicht es, die Orientierung eines SplitPanes zu verändern:

\begin{lstlisting}[language=JAVA]
	// Trennbalken ist vertikal
	splitPanel.setOrientation(JSplitPane.VERTICAL_SPLIT);
	
	// Trennbalken ist horizontal
	splitPanel.setOrientation(JSplitPane.HORIZONTAL_SPLIT);
\end{lstlisting}

Mit der Methode \textbf{setOrientation} kann die Anordnung der Elemente im linken Panel verändert werden. Für die dritte mögliche Anordnung (Abbildung \ref{fig:gui-main-left-o3}, ,,Widescreen Central'') müssen die Elemente des äußeren Panels allerdings vertauscht werden. Deshalb wird bei jedem Wechsel auf diese Konfiguration die private Methode ,,swap'' aufgerufen. Wenn als Parameter \textbf{true} übergeben wird, werden die Elemente ausgetauscht, ansonsten werden die Komponenten wieder auf ihre Ausgangsposition geracht.

\begin{lstlisting}[language=JAVA]
	private void swap(boolean def) {
		
		// Reset splitPane
		this.outerPane.setTopComponent(null);
		this.outerPane.setBottomComponent(null);
		
		// Decide what to do
		if( def ) {
			this.outerPane.setTopComponent(this.jSourceCodeContainer);
			this.outerPane.setBottomComponent(this.innerPane);
		}
		else {
			this.outerPane.setBottomComponent(this.jSourceCodeContainer);
			this.outerPane.setTopComponent(this.innerPane);
		}
	}
\end{lstlisting}

\subsection{Zustandspanel}
\label{sec:gui-main-left-zust}
%TODO ref GUI states
%TODO ref right state panel

\begin{figure}[htbp] 
  \centering
     \includegraphics[width=0.7\textwidth]{./media/images/gui/main/CCompact-gui-left-panel.png}
  \caption{Das Zustandspanel während des Debuggens}
  \label{fig:gui-main-left-panel}
\end{figure}

Die Benutzeroberfläche der Entwicklungsumgebung kann vier unterschiedliche Zustände annehmen. Diese Zustände beziehen sich nur auf die Basisfunktionen der Entwicklungsumgebung und des Debuggers, nicht aber auf den Zustand der zu bearbeitenden Quest. Es sind folgende Zustände möglich:
\begin{enumerate}
\item \textbf{Editormodus:} Der Benutzer kann den Source Code wie bei einem Texteditor bearbeiten. Auch Eingabedaten für den Debugger können eingegeben werden. Dies ist der Standardzustand, das Zustandspanel ist grau.
\item \textbf{Fehlermodus:} Es ist ein Fehler im Source Code des Benutzers aufgetreten. Der Fehler wurde entweder im Präprozessor, im Compiler oder im Interpreter erkannt. Das Zustandspanel ist rot und zeigt zusätzliche Informationen zum Fehler, wie etwa die Art des Fehlers und die Zeile, in der der Fehler aufgetreten ist\footnote{Unter Umständen wird nicht die richtige Zeile erkannt, sondern eine darauffolgende Zeile markiert.}. Im rechten Teil des Hauptfensters wird eine detaillierte Beschreibung des Fehlers gezeigt. In diesem Modus kann der Source Code bearbeitet werden, um den Fehler zu beheben.
%TODO ref right panel -> Fehleranzeige
%TODO ref compiler errors -> footnote 

Der Text im Zustandspanel wird im Fehlermodus aus folgenden Informationen zusammengesetzt:
\begin{enumerate}
\item \textbf{Präfix:} Beschreibender Text am Beginn der Nachricht.
\item \textbf{Dateiname:} Wird nur angegeben, wenn der Fehler in einer externen Datei oder einer Bibliothek aufgetreten ist.
\item \textbf{Zeilennummer:} Die Zeile, in der der Fehler aufgetreten ist.
\item \textbf{Postfix:} Beschreibender Textteil am Ende der Nachricht. Dies ist in manchen Sprachen notwendig, um grammatikalisch vollständige Sätze konstruieren zu können (Siehe Beispiel unten).
\end{enumerate}

Die Fehlernachricht wird mit folgendem Code zusammengesetzt. Dieser Teil stammt aus der Funktion \textbf{setErrorMode} in der Klasse \textbf{GUIleftPanel}:

\begin{lstlisting}[language=JAVA]
	// Setting error text for state panel
	this.jStateLabel.setText("<html>! ! ! " +
		// Error title prefix (if available)
		(title[0] == null ? _("error") : title[0]) +
		// File name (if available)
		(file == null || file != "main" ? "" : " in file " + file) + " " +
		// Error line (if available)
		(line >= 0 ? _("in line") + " " + (int)objLine[1] : "") +
		// Error title postfix (if available)
		(title[1] == null ? "" : " " + title[1]) +
	" ! ! !</html>");
\end{lstlisting}

\textbf{Beispiel:}
\[
\underbrace{Semikolon}_{Praefix} \underbrace{in\thinspace der\thinspace Datei\thinspace stdio.h}_{Datei} \underbrace{in\thinspace Zeile\thinspace 15}_{Zeilenangabe} \underbrace{(oder\thinspace vorher)\thinspace vergessen}_{Postfix}
\]

Präfix und Postfix können im Head der der Fehlerbeschreibungsdatei definiert werden. Für jeden bekannten Fehler gibt es ein eigenes HTLM-Dokument, das einen entsprechenden Beschreibungstext enthält. Diese Fehlerdateien sind im Ordner \textbf{error} im C Compact Programmordner zu finden.
%TODO ref guirightpanel -> error documents

\begin{lstlisting}[language=HTML]
<head>
	<prefix>Semikolon</prefix>
	<postfix>(oder vorher) vergessen</postfix>
</head>
\end{lstlisting}

\item \textbf{Schritt für Schritt debuggen:} Der Debugger kann grundsätzlich in zwei unterschiedlichen Modi ausgeführt werden. In diesem Fall wird das Programm so ausgeführt, dass der Benutzer jeden Schritt des Interpreters selbst initiieren muss (durch Druck auf den "Nächster Schritt"-Button oder die Taste F6). Hier ist das Zustandspanel gelb gefärbt. Das Bearbeiten des Source Code ist während der Laufzeit nicht möglich.
%TODO ist F6 korrekt???
%TODO refer to debugger

\item \textbf{Austomatisches debuggen:} Die andere Möglichkeit ist, den Interpreter in gewissen Zeitabständen selbst weiterspringen zu lassen. Auch hier ist das Bearbeiten des Source Code nicht möglich. Das Zustandspanel ist grün.
\end{enumerate}

Das Zustandspanel wird in der Methode \textbf{init()} der Klasse \textbf{GUIleftPanel} initialisiert.

\subsection{Textfeld für den Source Code}
\label{sec:gui-main-left-code}
Das Feld für den Source Code wurde als \textbf{RSyntaxTextArea}\footnote{http://bobbylight.github.io/RSyntaxTextArea/} implementiert. Dieses Projekt wurde von Robert Futrell veröffentlicht und steht unter einer modifizierten BSD-Lizenz\footnote{https://github.com/bobbylight/RSyntaxTextArea/blob/master/src/main/dist/RSyntaxTextArea.License.txt}.

Das verwendete Textfeld ermöglicht Syntax Highlighting für viele bekannte Programmiersprachen. Anfangs haben wir den Standardhihglighter für C-Code verwendet; seit der Version Alpha 1.2 wird ein eigens angepasster Syntax Highlighter verwendet. Dadurch werden auch Funktionen und Schlüsselwörter, die nur in C Compact vorkommen, richtig markiert.
%TODO ref Versionsgeschichte
%TODO ref JFlex/Syntax Highlighter [pointhi]
%TODO ref settings -> font size

RSyntaxTextArea stellt noch eine Reihe weiterer praktischer Funktionen zur Verfügung, wie etwa das einklappen (verstecken) von Kommentaren oder Funktionsrümpfen. Manche Features, wie etwa automatische Vervollständigung beim Tippen oder das Erkennen von gleichen Variablennamen haben wir vorerst deaktiviert, um Probleme zu vermeiden und die Benutzeroberfläche nicht zu überfüllen.
%TODO Letzten Satz nochmal lesen

\begin{figure}[htbp] 
  \centering
     \includegraphics[width=0.7\textwidth]{./media/images/gui/main/CCompact-gui-left-code.png}
  \caption{Das Textfeld für den Source Code}
  \label{fig:gui-main-left-code}
\end{figure}

Die Zeile, in der sich der Cursor befindet, wird durch einen Farbigen Balken markiert. Dieser Balken hat, je nach Modus des Debuggers, die selbe Farbe wie das Zustandspanel (siehe Kapitel \ref{sec:gui-main-left-zust}).

Dieses Textfeld wird mit der statischen Methode \textbf{initCodePane()} der Klasse \textbf{InitLeftPanel} initialisiert. Die Methode wird beim Erstellen des linken Teils der Benutzeroberfläche aufgerufen.

\subsection{Ein- und Ausgabetextfeld}
\label{sec:gui-main-left-io}
%TODO add picture of text pane while reading input data
%TODO ref settings -> font size

In diesem Textfeld werden Daten eingegeben, die vom Interpreter während der Laufeit eingelesen werden. Jedes Zeichen kann nur einmal eingelesen werden; bereits eingelesene Zeichen werden gelb hinterlegt. Die Ausgabedaten werden in einem separaten Textfeld dargestellt.

In C Compact gibt es für Ein- und Ausgabe keine Konsole, wie sie bei einfachen programmen meist verwendet wird. Bei unseren Versuchen hat sich gezeigt, dass dies die Schülerinnen und Schüler kaum stört. Besonders beim Entwicklen eines Algorithmus ist diese Methode der Ein- und Ausgabe besonders hilfreich, da die Eingabedaten nicht immer wieder eingegeben werden müssen.
%TODO ref Versuche

Dem Intrepreter muss immer ein Objekt, welches das Interface \textbf{StdInOut} aus dem Packet \textbf{at.jku.ssw.cmm.debugger} implementiert, übergeben werden. In diesem Interface befinden sich Methoden zur Ein- und Ausgabe. Diese Methoden werden vom Interpreter aufgerufen, wenn im Programm Anweisungen zur Ein- oder Ausgabe vorkommen.
%TODO ref interpreter, debugging listener, ...

\begin{lstlisting}[language=JAVA]
public interface StdInOut {
	public char in() throws RunTimeException;
	public void out(char arg0);
}
\end{lstlisting}

%TODO ref ????
Für den Debugger wurde die Ein- und Ausgabe in der Klasse \textbf{IOstream} im Packet \textbf{at.jku.ssw.cmm.debugger} implementiert. Die (gekürzte) Programmierung sieht wie folgt aus:

\begin{lstlisting}[language=JAVA]
@Override
public char in() throws RunTimeException {

	char c;
	try {
		// Read and remove next character
		c = this.inputStream.get(0);
		this.inputStream.remove(0);
	} catch (Exception e) {
		// Throw interpreter runtime error if no more input data available
		throw new RunTimeException("no input data", null, 0);
	}
	
	//Highlight the character read
	this.main.getLeftPanel().increaseInputHighlighter();
	
	return c;
}

@Override
public void out(final char arg0) {

	java.awt.EventQueue.invokeLater(new Runnable() {
		public void run() {
			// Add given character to output text area of the main GUI
			main.getLeftPanel().outputStream("" + arg0);
		}
	});
}
\end{lstlisting}

Da der Interpreter in einem separaten Thread ausgeführt wird, muss die Ausgabe der übergebenen Daten in den \textbf{Event Dispatcher Thread} eingereiht werden. Für die Eingabedaten inst das nicht notwendig, da der Eingabestring in einer lokalen Variable dieses Objektes gespeichert ist, auf den kein anderer Thread zugreifen kann.
%TODO ref EDT, thread problem

Um das Markieren der bereits eingelesenen Daten und das Aussehen des Eingabedatenfeldes richtig zu organisieren, wurde das Element \textbf{JInputDataPane} erstellt. Diese Klasse erbt von dem zuvor verwendeten \textbf{JTextPane}\footnote{http://docs.oracle.com/javase/7/docs/api/javax/swing/JTextPane.html}. Die Klasse JInputDataPane ist im Package \textbf{at.jku.ssw.cmm.gui.init} zu finden. 

Die verwendeten Textfelder werden mit den statischen Methoden \textbf{initInputPane()} und \textbf{initOutputPane()} der Klasse \textbf{InitLeftPanel} initialisiert.

\section{Rechter Teil der Benutzeroberfläche}
Der rechte Teil der Benutzeroberfläche besteht aus einem Zustandspanel für die aktuelle Aufgabe (Quest) und einem Panel mit mehreren Registerkarten, die unterschiedliche Bedienelemente enthalten, zum Beispiel für den Debugger.

Ähnlich wie der linke Teil der Benutzeroberfläche besitzt auch dieser Bereich eine eigene Klasse. Die Klasse \textbf{GUIrightPanel} im Paket \textbf{at.jku.ssw.cmm.gui} bildet eine Schnittstelle zwischen dem Hauptteil der Benutzeroberfläche (der Klasse \textbf{GUImain}, siehe Kapitel \ref{gui-main}) und den Komponenten, die in den Registerkarten des rechten Panels eingebettet sind.
%TODO letzten Satz nochmal lesen

Wie bei der Klasse \textbf{GUIleftPanel} wird die Benutzeroberfläche selbst mit der Methode \textbf{init} augerufen. Als Parameter wird eine Referenz auf \textbf{GUImain} übergeben, der Rückgabewert ist das fertig initialisierte rechte Panel.

In den Registerkarten können folgende Inhalte vorhanden sein:
\def\arraystretch{1.6}
\begin{table}
\begin{tabularx}{\columnwidth}{l|p{3cm}|p{6cm}|l}
\textbf{Name}&\textbf{Sichtbar}&\textbf{Beschreibung}&\textbf{Siehe Kapitel}\\
\hline
Debug&Immer&Enthält Elemente zum Bedienen des Debuggers&0.0.0\\%TODO ref Debugger GUI
Error&Fehler aufgetreten&Zeigt detaillierte Informationen zu einem beim Debuggen an&0.0.0\\%TODO ref compiler, interpreter errors
Quest&Questsystem aktiv&Zeigt die aktuelle Aufgabenstellung an und ermöglicht das Überprüfen der Aufgabe&0.0.0\\%TODO ref Questsystem
Profile&Questsystem aktiv&Zeigt das Profil des Benutzers: Benutzerbild, Name und Auszeichnungen&0.0.0%TODO ref Profile, profilePanel
\end{tabularx}
\caption{Registerkarten im rechten Teil des Hauptfensters}\label{tab:gui-main-right-reg}
\end{table}
%TODO add chapter references in last column

Da die Inhalte dieser Panele in den Bericht spezifischer Funktionen von C Compact fallen, werden Aufbau, Implementierung und Funktion dieser Komponenten in anderen Teilen dieser Dokumentation beschrieben.

\subsection{Zustandspanel des Questsystems}
Wie auch das Zustandspanel des Debuggers, das in Kapitel \ref{sec:gui-main-left-zust} beschrieben wurde, visualisiert dieses Panel einen Zustand. Allerdings bezieht sich diese Oberfläche nur auf das Questsystem und ist daher auch nur gemeinsam mit diesem Verfügbar (In Kapitel \ref{sec:gui-main-impl} wurde dargelegt, wie C Compact ohne den Funktionen des Questsystems gestartet werden kann). Es ist wichtig zu erwähnen, dass die Zustände des Questsystems vollkommen unabhängig von denen des Debuggers sind.
%TODO ref quest system, quest tester

Folgende Zustände können angezeigt werden:
\begin{enumerate}
\item \textbf{Keine Quest ausgewählt (Idle):} Zu Beginn hat der Benutzer noch keine Aufgabe ausgewählt, das Zustandspanel ist grau.
\item \textbf{Quest wird bearbeitet:} Während der Benutzer eine ausgewählte Aufgabe bearbeitet, ist das Panel blau.
\item \textbf{Quest wird getestet:} Wenn der Benutzer überprüfen will, ob er eine Aufgabe richtig erfüllt hat startet er einen Test. Während des Tests wird das Zustandspanel gelb, im Regelfall ist dieser Status aber sehr kurz. Wenn während des Tests ein Fehler auftritt (wie etwa eine Endlosschleife), kann der Test in diesem Zustand abgebrochen werden.
%TODO ref Quest test, quest test panel
\item \textbf{Test erfolgreich:} Verläuft der Test erfolgreich, wird das Zustandspanel grün.
\item \textbf{Test fehlgeschlagen:} In diesem Fall ist entweder ein Fehler im Source Code des Benutzers erkannt worden (eine Detaillierte Beschreibung wird im Fehlerpanel angezeigt, siehe Kapitel \ref{}) oder das Programm des Benutzers erfüllt die gestellte Aufgabe nicht. Der Benutzer hat die Möglichkeit, seinen Code zu überarbeiten.
%TODO finish ref
\end{enumerate}

Der angezeigte Zustand wird mit folgenden Methoden der Klasse \textbf{GUIrightPanel} initialisiert. Diese Methoden sind wirkungslos, wenn das Questsystem deaktiviert ist.
\begin{lstlisting}[language=JAVA]
public void setIdleMode();
public void setQuestMode(String title); // Name der Quest wird übergeben
public void setTestMode();
public void setSuccessMode();
public void setFailedMode();
\end{lstlisting}

%TODO add refs to the following four subsections
\subsection{Registerkarte ,,Debug''}
\ref{sec:gui-main-right-reg-deb}
Diese Registerkarte ist immer an der ersten Stelle und jederzeit verfügbar. Im oberen Teil befindet sich ein Panel mit Bedienelementen des Debuggers, darunter werden die Variablen im Programm des Benutzers während der Laufzeit des Interpreters dargestellt.

Eine Referenz auf das Panel in dieser Registerkarte wird von folgender Methode übergeben:
\begin{lstlisting}[language=JAVA]
public GUIdebugPanel getDebugPanel() {
	return this.debugPanel;
}
\end{lstlisting}

\subsection{Registerkarte ,,Error''}
Wenn beim Debuggen ein Fehler im Compiler, Präprozessor oder Interpreter auftritt, wird dieses Panel sichtbar. Außerdem werden Fehler angezeigt, die beim Überprüfen einer Quest auftreten, sofern der Fehler durch den Source Code des Benutzers aufgetreten ist. Ist der Fehler behoben und wurde das Programm erfolgreich ausgeführt, verschwindet diese Registerkarte automatisch.

Mit folgender Methode werden Fehlernachrichten eingeblendet. Als Parameter wird die Fehlernachricht des Compiler, Präprozessor oder Interpreter übergeben, der Rückgabewert ist der Name (Präfix und Postfix) des Fehlers, wie er im Zustandspanel des Debuggers angezeigt wird. Siehe dazu Kapitel \ref{sec:gui-main-left-zust}.
\begin{lstlisting}[language=JAVA]
public String[] showErrorPanel(String errorCode);
\end{lstlisting}

Eine Detaillierte Beschreibung dieser Methode und der Verwaltung von Fehlerdokumenten ist in Kapitel \ref{} zu finden.

\subsection{Registerkarte ,,Quest''}
In dieser Registerkarte werden Informationen und Bedienelemente zum Ausführen einer Aufgabe des Questsystems (siehe Kapitel \ref{}) angezeigt.

Eine Referenz auf das Panel in dieser Registerkarte wird von folgender Methode übergeben:
\begin{lstlisting}[language=JAVA]
public GUITestPanel getTestPanel(){
	return this.testPanel;
}
\end{lstlisting}

%TODO also mention this in chapter "quest panel"
Anmerkung: Dieses Panel wird - vor Allem im Source Code von C Compact - auch als ,,testPanel'' bezeichnet, da der Benutzer mit Elementen dieses Komponenten überprüfen kann, ob er die aktuelle Aufgabe richtig gelöst hat.

\subsection{Registerkarte ,,Profile''}
Hier werden Informationen zum aktuellen Spielerprofil angezeigt. Siehe Kapitel \ref{}.

Eine Referenz auf das Panel in dieser Registerkarte wird von folgender Methode übergeben:
\begin{lstlisting}[language=JAVA]
public ProfilePanel2 getProfilePanel() {
	return this.questPanel;
}
\end{lstlisting}

\section{Menüleiste}
\label{sec:gui-main-menu}
Menüs sind bei der Gestaltung einer Benutzeroberfläche nahezu unerlässlich. Eine Menüleiste, bzw. ein Menü erleichtert dem Benutzer die Orientierung, da alle Funktionen auf einen Blick zu finden sind. Die Menüleiste wird auch in den \emph{Common User Access}\footnote{http://de.wikipedia.org/wiki/Common\_User\_Access} Richtlinien für die Gestatlung von Benutzeroberflächen definiert; diese Standards wurden 1989 von der Firma IBM festgelegt. Diese Standards sind mittlerweile weit verbreitet und wurden in einer Reihe von Systemen umgesetzt.

Die Menüleiste wird in der statischen Methode \textbf{initFileM()} in der Klasse \textbf{at.jku.ssw.cmm.gui.init.InitMenuBar} angelegt. Die Tastaturkürzel - \emph{,,Keyboard Shortcuts''} - zum Beispiel \emph{Strg + S} zum Speichern, werden ebenfalls hier definiert. Beim Verwenden eines Tastaturkürzels wird der Event Listener des zugehörigen Menüeintrages aufgerufen.
%TODO ref to keyboard shortcuts in GUIcontrolPanel

Initialisieren eines Menüeintrages:
\begin{lstlisting}[language=JAVA]
// Menüeintrag wird initialisiert
JMenuItem openMI = new JMenuItem(_("Open"));

// Menüeintrag wird zum Menü hinzugefügt
fileM.add(openMI);

// Mit dem Action Listener verknüpfen, sodass dieser beim Anklicken des Menüs aufgerufen wird
openMI.addActionListener(listener.openHandler);

// Mit dem Keyboard Shortcut verknüpfen
openMI.setAccelerator(KeyStroke.getKeyStroke(KeyEvent.VK_O, ActionEvent.CTRL_MASK));

// Eintrag zur MenuBarControl hinzufügen
menuBarControl.add(newMI);
\end{lstlisting}

Die Events der Menüleiste werden in der Klasse \textbf{at.jku.ssw.cmm.gui.event.MenuBarEventListener} abgewickelt. Diese Klasse enthält Methoden, die bestimmte Aktionen abwickeln und innere Klassen, die einen Event Listener implementieren und diese Methoden aufrufen. Diese Aufteilung ist nötig, da bestimmte Aktionen auch von anderen instanzen aufgerufen werden können (zum Beispiel wird auch gespeichert, wenn der Benutzer C Compact schließt).

\subsection{Aktivieren und Deaktivieren von Menüeinträgen}
\label{sec:gui-main-menu-ctrl}
Da während der Debugger ausgeführt wird keine Aktionen erlaubt sind, die den Text im Source Code Textfeld ändern (Siehe Kapitel \ref{sec:gui-main-left-code}), werden bestimmte Menüeinträge aktiviert oder deaktiviert, wenn sich der Status des Debuggers ändert (Siehe Kapitel \ref{sec:gui-main-left-zust}). Diese Aktionen werden in der Klasse \textbf{at.jku.ssw.cmm.gui.MenuBarControl} zusammengefasst. Diese Klasse enthält eine Liste, in der alle Menüeinträge referenziert werden, die beim Ausführen des Debuggers deaktiviert werden sollen. Mit der Methode \textbf{add()} werden Objekte in diese Liste eingetragen.
\begin{lstlisting}[language=JAVA]
public void add(JMenuItem mi){
	this.list.add(mi);
}
\end{lstlisting}

Mit den Methoden \textbf{lockAll()} und \textbf{unlockAll()} werden diese Menüeinträge dann deaktiviert oder aktiviert.

\begin{lstlisting}[language=JAVA]
public void lockAll();
public void unlockAll();
\end{lstlisting}

Des Weiteren kann in dieser Kontrollklasse ein Menüeintrag definiert werden, der eine Liste der zuletzt geöffneten Dateien enthält. Mit der Methode \textbf{setRecentMenu} wird dieser Menüeintrag definiert:
\begin{lstlisting}[language=JAVA]
public void setRecentMenu( JMenu mi ){
	this.recentMI = mi;
}
\end{lstlisting}

Dieses Untermenü wird mit der Methode \textbf{updateRecentFiles} aktualisiert. Siehe dazu auch Kapitel \ref{}.
%TODO ref GUImainSettings
\begin{lstlisting}[language=JAVA]
public void updateRecentFiles( List<String> recentFiles, String currentFile );
\end{lstlisting}

Im Untermenü ,,Source Code'' (Siehe Kapitel \ref{}) gibt es Funktionen zum Aufheben bzw. Wiederholen der letzten Aktion (\emph{undo and redo}). Diese Menüeinträge sollen aber nur aktiviert sein, wenn eine solche Aktion tatsächlich möglich ist. Beispielsweise kann nach dem Start des Programmes nichts rückgängig gemacht werden, da noch keine Aktionen durchgeführt wurden. Außerdem dürfen diese Aktionen nicht ausgeführt werden, wenn der Debugger gerade läuft, da der Source Code sonst während der Laufzeit verändert werden würde.

Referenzen auf die beiden entsprechenden Menüeinträge werden in der Klasse MenuBarControl separat gespeichert.
\begin{lstlisting}[language=JAVA]
public void setUndo(JMenuItem undo){
	this.undo = undo;
}

public void setRedo(JMenuItem redo){
	this.redo = redo;
}
\end{lstlisting}

Mit der Methode \textbf{updateUndoRedo()} werden die Zustände dieser Menüeinträge aktualisiert. Als Parameter werden eine Referenz auf das Textfeld mit dem Source Code und eine boolean-Variable übergeben. Letztere ist \textbf{true} wenn der Debugger gerade läuft.
\begin{lstlisting}[language=JAVA]
public void updateUndoRedo( RSyntaxTextArea tArea, boolean running ){
	this.undo.setEnabled(!running & tArea.canUndo());
	this.redo.setEnabled(!running & tArea.canRedo());
}
\end{lstlisting}

\subsection{Menü ,,Datei''}
\label{sec:gui-main-menu-file}
Das wohl gewohnteste Menü in jedem Programm ist das Menü ,,Datei'' In C Compact enthält es neben den grundlegenden Funktionen zum Öffnen oder Speichern einer vorhandenen oder neuen Datei auch noch weitere Funktionen. Der Eintrag ,,Drucken'' verwendet eine Methode, die in der RSyntaxTextArea implementiert ist (Siehe Kapitel \ref{sec:gui-main-left-code}; allerdings treetn bei der Formatiering oft Fehler auf. Wesentlich wichtiger sind die Menüpunkte ,,Einstellungen'' (Siehe Kapitel \ref{}), ,,Sprache'' (Siehe Kapitel \ref{}) und ,,Über C Compact'' (Siehe Kapitel \ref{}). Schließlich gibt es noch dem Menüpunkt ,,Beenden'', C Compact kann aber auch mit der Tastenkombination \emph{Alt + F4} (diese Kombination ist in den meisten Betriebssystemen und Oberflächen bereits standardmäßig integriert) oder über entsprechende Bedienelemente im Fensterrahmen geschlossen werden.
%TODO ref sprache, einstellungen, about

\subsection{Menü ,,Source Code''}
Dieses Menü würde in den meisten Programmen dem Menü ,,Bearbeiten'' entsprechen. Allerdings beziehen sich die Menüeinträge wie ,,Kopieren'', ,,Einfügen'', ,,Rückgängig'', etc. immer auf das Textfeld für den Source Code - wir wollen mit dem Menünamen also Verwechslungen vermeiden - und andererseits enthält dieses Menü auch Einträge zum Steuern des Debuggers. Diese Einträge werden vom Benutzer kognitiv mit dem Source Code verknüpft.

\subsection{Menü ,,Fortschritt''}
Dieses Menü enthält Einträge, die das Questsystem (Siehe Kapitel \ref{}) betreffen. Mit ,,Profil Wählen'' Wird C Compact beendet und der Launcher (Siehe Kapitel \ref{}) gestartet, sodass der Benutzer ein anderes Profil wählen kann. Mit dem Befehl ,,Profil Exportortieren'' kann der Benutzer sein Profil an einem anderen Ort speichern (Siehe Kapitel \ref{}). Der Menüeintrag ,,Quest wählen'' öffnet ein Fenster zum Auswählen eines Questpaketes; im Anschluss kann der Benutzer eine Aufgabe zum Bearbeiten wählen (Siehe Kapitel \ref{}). Mit ,,Questpaket importieren'' kann ein Package importiert werden (Siehe Kapitel \ref{} und \ref{}).
%TODO ref Questsystem
%TODO ref launcher
%TODO ref profil exportieren
%TODO ref select quest window
%TODO ref package, import package



%!TEX root = "../../../DA_GUI.tex"

%	--------------------------------------------------------
% 	Funktionsweise des Debuggers
%	--------------------------------------------------------

\chapter{Der Debugger}
\label{sec:deb}

%   Einleitung
%!TEX root = "../../../DA_GUI.tex"

%	--------------------------------------------------------
% 	Debugger: Einleitung
%	--------------------------------------------------------

\section{Einleitung}

%   --------------------------------------------------------
%   Grundlegende Funktionen, Ideen
%   --------------------------------------------------------

\subsection{Die grundlegende Idee}
\label{sec:deb-idea}
Ein wichtiger Bestandteil von C Compact ist der integrierte Debugger, der speziell für die Bedürfnisse von Anfängern ausgelegt ist. Während das Programm Schritt für Schritt abgearbeitet wird, sind alle Variablen und der Call Stack in übersichtlicher Form dargestellt. In der Variablentabelle werden Wertänderungen farbig hervorgehoben. Dadurch soll ein fundiertes Verständnis für den Programmablauf und die Logik dahinter entstehen.

\begin{figure}[h]
\centering
\includegraphics[width=0.7\textwidth]{./media/images/gui/debugger/gui-debugger-marked.png}
\caption{Die vier Komponenten des Debuggers}
\label{fig:deb-intro-m1}
\end{figure}

Während die Debugger professioneller Entwicklungsumgebungen eine Hilfestellung für erfahrene Programmierer sind, ist der Debugmodus von C Compact besonders auf Anfänger ausgelegt. Damit Einsteiger und unerfahrene Programmierer von Beginn an mit dem Debugger in Berührung kommen, ist er ein fester Bestandteil der Benutezroberfläche.

Abbildung \ref{fig:deb-intro-m1} zeigt die vier Komponenten des Debuggers in der Benutzeroberfläche:
\begin{enumerate}
\item Im Quelltext wird die gerade abgearbeitete Zeile markiert
\item Mit den Kontrollelementen kann der Debugger gestartet und angehalten werden
\item Die Variablen des Programms werden übersichtlich angezeigt
\item Ein- und Ausgabe des laufenden Programms befinden sich in getrennten Textfeldern, eingelesene Zeichen werden gelb markiert.
\end{enumerate}

%   --------------------------------------------------------
%   Bedienung des Debuggers
%   --------------------------------------------------------

\subsection{Bedienung des Debuggers}
\label{sec:deb-use}
Mit den Kontrollelementen kann der Debugger intuitiv und einfach bedient werden. Die Bedienelemente (Abbildung \ref{fig:deb-gui-ctrl}) sind im rechten Teil des Hauptfensters im Tab \glqq{}Debug\grqq{} (siehe Kapitel \ref{sec:gui-main-right-reg-deb}) zu finden.

\def\arraystretch{1.4}
\begin{table}[h]
\begin{tabular}{|l|l|l|l|}
	\hline
	Element & \parbox{2cm}{Keyboard\\Shortcut}& Name & Funktion\\
	\hline
	\ding{228} / \ding{122}\thinspace \ding{122} & F5 & Play/Pause & Startet den Debugger/Hält den Debugger an\\
	\ding{122}\thinspace \ding{228} & F6 & Step & Springt zum nächsten Schritt weiter\\
	\ding{110} & F7 & Stop & Beendet den Debugger\\
	--- & --- & Schieberegler & \parbox{7cm}{Ändert die Zeitabstände beim automatischen Debuggen}\\
	\hline
\end{tabular}
\caption{Funktionen der Bedienelemente des Debuggers}\label{tab:deb-ctrl}
\end{table}



\begin{figure}[h]
\centering
\includegraphics[width=0.4\textwidth]{./media/images/gui/debugger/ctrl-elements.png}
\caption{Kontrollelemente des Debuggers}
\label{fig:deb-gui-ctrl}
\end{figure}

Der Debugger hat vier unterschiedliche Modi, die im linken Teil des Hauptfensters durch eine Zustandsanzeige visualisiert werden (siehe Kapitel \ref{sec:gui-main-left-zust}). Der Wechsel zwischen diesen Zuständen erfolgt durch Eingaben des Benutzers, entweder mit den Kontrollelementen (Buttons) in der Benutzeroberfläche, oder mit Keyboard Shortcuts.

Die vier Zustände sind:
\begin{enumerate}
\item \textbf{Text bearbeiten (ready mode)}
Dies ist der Standardzustand. Der Quelltext und Eingabedaten für den Debugger können bearbeitet werden. Außerdem sind Dateioperationen erlaubt.
\begin{figure}[h!]
\centering
\includegraphics[width=0.65\textwidth]{./media/images/gui/debugger/mode-idle.png}
\caption{Zustandsanzeige für den Editormodus}
\label{fig:deb-zust-1}
\end{figure}
\item \textbf{Fehler aufgetreten (error mode)}
Beim Compilieren oder während der Laufzeit des Debuggers ist ein Fehler aufgetreten. Um diesen zu beheben, ist das Bearbeiten des Quelltexts nach wie vor erlaubt.
\begin{figure}[h!]
\centering
\includegraphics[width=0.65\textwidth]{./media/images/gui/debugger/mode-error.png}
\caption{Zustandsanzeige für den Fehlermodus}
\label{fig:deb-zust-2}
\end{figure}
\item \textbf{Programm Schritt für Schritt abarbeiten oder Pause (pause mode)}
Der Debugger ist aktiv, wartet aber auf eine Eingabe des Benutzers, um den nächsten Befehl abzuarbeiten. In diesem Modus kann der Benutzer sein Programm Schritt für Schritt abarbeiten. Das Modifizieren des Quelltextes ist in diesem Modus nicht möglich.
\begin{figure}[h!]
\centering
\includegraphics[width=0.65\textwidth]{./media/images/gui/debugger/mode-step.png}
\caption{Zustandsanzeige für den Pausemodus}
\label{fig:deb-zust-3}
\end{figure}
\item \textbf{Programm automatisch abarbeiten (run mode)}
Das Programm des Benuters wird Schritt für Schritt abgearbeitet. Zwischen den Befehlen wird für eine bestimmte Zeit zwischen 0 und 5 Sekunden gewartet. Diese Verzögerung kann mit dem Schieberegler bei den Kontrollelementen eingestellt werden. Auch hier kann der Quelltext nicht verändert werden.
\begin{figure}[h!tp]
\centering
\includegraphics[width=0.65\textwidth]{./media/images/gui/debugger/mode-run.png}
\caption{Zustandsanzeige für den Automatischen Debugmodus}
\label{fig:deb-zust-4}
\end{figure}
\end{enumerate}

\begin{figure}[h!]
\centering
\includegraphics[width=0.65\textwidth]{./media/images/gui/debugger/RunModes_Simple.png}
\caption{Vereinfachtes Zustandsdiagramm des Debuggers}
\label{fig:deb-zust-simple}
\end{figure}

%Obwohl der Debugger selbst zwischen Modus Editormodus (1) und Fehlermodus (2) unterscheidet, gibt es in der Benutzeroberfläche nur mehr den Editormodus. Der Fehlermodus wird trotzdem im Zustandspanel des Debuggers im Hauptfenster angezeigt, damit der Benutzer auf den aufgetretenen Fehler aufmerksam gemacht wird. Wie in Abbildung \ref{fig:deb-zust-simple} ersichtlich ist, erfüllen die beiden Modi die selben Funktionen.

Der Pausemodus (3) hat zwei Funktionen: einerseits entspricht dieser Modus einer Pausierung (bezogen auf Zustand 4, in dem der Debugger automatisch durchläuft) und andererseits kann in diesem Modus mit dem \glqq{}Nächster Schritt\grqq{} Button auch direkt zum nächsten Befehl gesprungen werden.

Ein Sonderfall des run mode ist der \glqq{}quick run mode\grqq{}. Wenn der Regler für die Zeitabstände zwischen den Schritten des Debuggers auf null gestellt wird, werden die folgenden Schritte so schnell wie möglich abgearbeitet, bis entweder das Programm zu Ende ist oder der Debugger auf einen \glqq{}wait\grqq{}-Befehl stößt.

%   --------------------------------------------------------
%   Schlüsselwörter
%   --------------------------------------------------------

\subsection{Schlüsselwörter}
\label{sec:deb-keywords}
In der Sprache von C Compact gibt es zwei spezielle Schlüsselwörter, die Einfluss auf das Verhalten des Debuggers nehmen: \textbf{wait} und \textbf{library}.

\subsubsection*{wait}
In C Compact gibt es seit der Version Alpha 1.2 keine Breakpoints als Option in der Benutzeroberfläche mehr. Anstatt dessen wird der Befehl \textbf{wait} verwendet. Dieser Befehl kann in jeder Funktion verwendet werden. Stößt der Debugger auf diesen Befehl, geht er in den \textbf{Pausemodus}. An dieser Stelle kann der Debugger dann entweder Schritt für Schritt, oder wieder im automatischen Debugmodus fortgesetzt werden.

\begin{lstlisting}[language=CMM]
wait;
\end{lstlisting}

Durch die Umsetzung der Breakpoint-Funktion als Befehl in der Sprache besteht immer ein klarer Zusammenhang zwischen Aktionen des Debuggers und dem Quelltext.

\subsubsection*{library}
Funktionen und Variablen können vor dem Debugger \glqq{}versteckt\grqq{} werden. Das Attribut \textbf{library} kann sowohl für eine Variable, als auch für eine Funktion verwendet werden. Das Schlüsselwort muss immer zwischen Typ und Name der Variable bzw. Funktion stehen.

\begin{lstlisting}[language=CMM]
const float library M_PI = 3.141592654;
\end{lstlisting}
Eine Variable mit dem \textbf{library}-Attribut wird im Debugger nicht angezeigt. Dies wird zum Beispiel verwendet, um die zahlreichen Konstanten der Bibliothek \textbf{math.h} im Debugger auszublenden.

\begin{lstlisting}[language=CMM]
float library sin(float x);
\end{lstlisting}
Hat eine Funktion das \textbf{library}-Attribut, springt der Debugger nicht in diese Funktion, sondern arbeitet sie in einem Schritt ab. Dadurch können Bibliotheksfunktionen im Debugger als einzelner Befehl in einem Schritt abgearbeitet werden.


%   Implementierung
%!TEX root = "../../../DA_GUI.tex"

%	--------------------------------------------------------
% 	Debugger: Implementierung
%	--------------------------------------------------------

\section{Implementierung}

%   --------------------------------------------------------
%   Interfaces
%   --------------------------------------------------------

\subsection{Interfaces}
Für die Kommunikation mit dem Interpreter sind zwei Interfaces wichtig. Im Package \textbf{at.jku.ssw.cmm.debugger} befinden sich das Interface \textbf{Debugger} und das Interface \textbf{StdInOut}. Die Interfaces sind getrennt, da sie unterschiedliche Teile der Benutzeroberfläche betreffen. Bei einfacheren Anwendungen des Interpreters, wie etwa beim Testen von Quests, werden aber beide Interfaces in der selben Klasse implelemtiert.
%TODO ref quest test

Über das Interface \textbf{Debugger} übergibt der Interpreter jeweils den aktuellen Knoten im abstrakten Syntaxbaum. In der Methode \textbf{step} werden außerdem eine Liste der Variablen, die im letzten Schritt ausgelesen wurden und eine Liste der Variablen, die im letzten Schritt geändert wurden übergeben. Die geänderten Variablen werden in der Variablenliste im Hauptfenster gelb markiert. Es ist möglich, dass in einem Schritt mehrere Variablen geändert werden, da mit dem Befehl \textbf{library} eine ganze Funktion auf einmal abgearbeitet werden kann.

\begin{lstlisting}[language=JAVA]
public interface Debugger {
	boolean step(Node arg0, List<Integer> readVariables, List<Integer> changedVariables);
	}
\end{lstlisting}

Die Methode \textbf{step} muss \textbf{true} zurückgeben, ansonsten bricht der Interpreter ab. Auf diese Weise kann der Interpreter ohne Probleme angehalten werden.

Das Interface \textbf{StdInOut} enthält Methoden zur Ein- und Ausgabe im Programm. Die Methode \textbf{in} kann eine \textbf{RunTimeException} verursachen, um den Interpreter anzuhalten, wenn keine Eingabedaten vorhanden sind. Die Methode \textbf{out} übergibt einfach das Zeichen, das ausgegeben werden soll.

\begin{lstlisting}[language=JAVA]
public interface StdInOut {
	public char in() throws RunTimeException;
	public void out(char arg0);
}
\end{lstlisting}

%   --------------------------------------------------------
%   Starten des Threads
%   --------------------------------------------------------

\subsection{Der Thread des Debuggers}
Alle Aktionen der Benutzeroberfläche werden im sogenannten Event Dispatch Thread\footnote{https://docs.oracle.com/javase/tutorial/uiswing/concurrency/dispatch.html} ausgeführt. Wenn dieser Thread angehalten wird, kann die Benutzeroberfläche nicht mehr auf Eingaben des Benutzers reagieren. Der Debugger soll aber jederzeit verzögert werden können. Deshalb wird der Interpreter ine einem eigenen Thread ausgeführt.

Die Klasse \textbf{CMMwrapper} im Package \textbf{at.jku.ssw.cmm} enthält Methoden zum Starten des Threads und sorgt außerdem dafür, dass nicht mehrere Threads gleichzeitig ausgeführt werden.

Der Interpreter wird mit folgender Methode in einem eigenen trhead gestartet:
\begin{lstlisting}[language=JAVA]
public boolean runInterpreter(PanelRunListener listener, IOstream stream, Tab table) {

	// Check if another interpreter thread is already running
	if ( table != null && this.thread == null ) {
			
		this.table = table;

		// Reset the output text panel
		this.main.getLeftPanel().resetOutputTextPane();

		this.interpreter = new Interpreter(listener, stream);

		// Create new interpreter object
		this.thread = new CMMrun(table, interpreter, this, debug);

		// Run interpreter thread
		this.thread.start();

		return true;
	}
	// Another thread is already running
	else {
		// Error message
		DebugShell.out(State.ERROR, Area.INTERPRETER, "Already running or not compiled!");

		return false;
	}
}
\end{lstlisting}

Die Variable \textbf{thread} der Klasse \textbf{CMMwrapper} ist eine Referenz auf den aktuell laufenden Interpreter-Thread. Wenn diese Variable nicht \textbf{null} ist, bedeuted das, dass bereits ein anderer trhead läuft. Wenn der Thread des Interpreters richtig beendet wird, ruft er die Methode \textbf{setNotRunning()} auf. Die Referenz auf diesen Thread wird gelöscht und ein neuer Thread kann bei bedarf gestartet werden.
\begin{lstlisting}[language=JAVA]
public void setNotRunning() {
	this.thread = null;
}
\end{lstlisting}

%TODO keep code ref up to date
Die Klasse \textbf{CMMrun} erbt von der Klasse Thread\footnote{https://docs.oracle.com/javase/7/docs/api/java/lang/Thread.html}. Ein neuer Thread wird gestartet, indem die Methode \textbf{start()}, die von der Klasse Thread vererbt wurde, aufgerufen wird (Siehe erstes Codebeispiel dieses Abschnitts, Zeile 17). Der Thread führt dann die Methoden \textbf{run()} aus, die ebenfalls vererbt wird und in der Kindklasse überschrieben werden muss.
\begin{lstlisting}[language=JAVA]
@Override
public void run() {

	// Allocating memory for interpreter
	Memory.initialize();

	// Run main function
	try {
		interpreter.run(table);
	}
	// Thrown when runtime error occurs
	catch (final RunTimeException e) {
		// Laufzeitfehler abfangen
		... 
		reply.setNotRunning();
		return;
	} catch (Exception e) {
		// Unerwartete Fehler abfangen
		reply.setNotRunning();
		return;
	}

	// Thread freigeben, sodass ein neuer thread gestartet werden kann
	reply.setNotRunning();
}
\end{lstlisting}

Ein Thread wird beendet, indem die Methode \textbf{run} verlassen wird. Dies ist die einfachste und sicherste Möglichkeit einen Thread zu beenden. Wenn der Thread beendet wird, währen er läuft, kann nicht sichergestellt werden, dass er zu dieser Zeit nicht gerade auf eine Variable zugreift und einen Schreibprozess nicht abschließt.

\subsubsection*{Threadprobleme in Swing}
Das Verwenden mehrerer Threads erfordert generell Vorsicht beim Zugriff auf eine Variable von mehreren Thread aus. In Swing ist dieses Thema besonders wichtig, da die Benutzeroberfläche abstürzen kann, wenn ein solcher Konflikt zwischen Threads auftritt. Zum Vermeiden solcher Probleme, die häufig zum Ansturz der Benutzeronerfläche führen, sollten folgende Grundregeln beachtet werden.

\begin{enumerate}
\item Objekte, die Elemente der Benutzeroebrfläche sind, sollten niemals von einem anderen Thread aus verändert werden. Diese Objekte sind Instanzen von Klassen aus der AWT- oder Swing-Bibliothek oder abgeleitete Klassen.

Um von einem anderen Thread aus ein Swing-Objekt zu verändern, muss dieser Teil des Quellcodes in den Event Dispatch Thread eingereiht werden. Dafür gibt es zwei Möglichkeiten:
\begin{enumerate}
\item Meistens 
\end{enumerate}
\end{enumerate}


%   --------------------------------------------------------
%   Abarbeiten von Schritten
%   --------------------------------------------------------

\subsection{Abarbeiten von Knoten im Abstrakten Syntaxbaum}
Die Schnittstelle zwischen Benutzeroberfläche und



%   Darstellung der Variablen
%!TEX root = "../../../DA_GUI.tex"

%	--------------------------------------------------------
% 	Debugger: Darstellung der Variablen
%	--------------------------------------------------------

%	--------------------------------------------------------
% 	Einführung + Erklärung Variablendarstellung
%	--------------------------------------------------------
\section{Darstellungsform der Variablen}

Zur Darstellung wurden zwei Konzepte verfolgt:
\begin{enumerate}
\item Die Anzeige des Call Stacks als Liste mit separaten Tabellen für die globalen und lokalen Variablen, Abbildung \ref{fig:deb-var-m1}.
\item Die Anordnung aller Variablen in einem Baum, wobei lokale Variablen Funktionen untergeordnet sind, Abbildung \ref{fig:deb-var-m2}.
\end{enumerate}

\begin{figure}[h!]
\centering
	\begin{minipage}{0.50\textwidth}
		\centering
		\includegraphics[width=1.0\textwidth]{./media/images/gui/var/callstack.png}
		\caption{Darstellung der Variablen als Liste}\label{fig:deb-var-m1}
	\end{minipage}\hfill
	\begin{minipage}{0.45\textwidth}
		\centering
		\includegraphics[width=1.0\textwidth]{./media/images/gui/var/treetable2.png}
		\caption{Darstellung der Variablen als Baum}\label{fig:deb-var-m2}
	\end{minipage}
\end{figure}

In den ersten Versionen der Benutzeroberfläche wurden die Variablen wie in Punkt 1 beschrieben dargestellt. Bereits während unseres Praktikums am Institut für Systemsoftware der JKU wurde die zweite Form der Variablendarstellung theoretisch entworfen. Die erste grundlegende Implementierung erfolgete noch während der Sommerferien.

%TODO ref versionsgeschichte
In der ersten numerierten Version (Alpha 1.0) waren noch beide Darstellungsformen vorhanden und konnten vom Benutzer ausgewählt und gewechselt werden. Allerdings waren diese beiden Darstellungsformen sehr unterschiedlich implementiert, was Anfangs dazu führte, dass die Darstellung der Variablen als Baum nicht mehr als ein zusätzliches Feature ohne den vollen Funktionsumfang der Variablendarstellung als Call Stack war.

Für die Version Alpha 1.1 sollte ursprünglich das Variablendarstellungssystem so verändert werden, dass beide Anzeigemodi über ein gemeinsames Interface angesprochen werden können. Nach ausführlicher Besprechung und Evaluierung der unterschiedlichen Variablendarstellungen haben wir aber beschlossen, die Variablen nur als Baum darzustellen. Dadurch haben sich folgende Vorteile ergeben:
\begin{enumerate}
\item Die Implementierung einer neuen Funktion in der Variablenanzeige muss nur noch einmal erfolgen und nicht für jeden Anzeigemodus extra vorgenommen werden.
\item Daraus resultiert auch ein durchdachteres und einfacheres Bedienungskonzept der Benutezroberfläche, da dieses nur an eine Variablendarstellung angepasst werden muss.
\item Davon profitiert zuletzt der Benutzer, der eine komplette und sauber implementierte Anzeige und Erklärung der Variablen während des Programmablaufes vorfindet.
\end{enumerate}

%	--------------------------------------------------------
% 	Darstellung in getrennten Tabellen
%	--------------------------------------------------------
\section{Darstellung der Variablen in getrennten Tabellen}

Diese Form der Variablendarstellung wurde erstmals im Dokument Ausführungsumgebung für C-- von Herrn Professor Blachek beschrieben und nach dieser Vorlage implementiert\footnote{Ausführungsumgebung für C--, Günther Blaschek, V1.0, 2014-06-18}. Ein Vorteil dieser Darstellungsform war die Übersichtlichkeit und Einfachheit besonders beim Debuggen von einfachen Programmen. Leider verlor das Konzept an Übersichtlichkeit, wenn fortgeschrittene Programme mit komplexen Strukturen abgearbeitet wurden. Eine Tabelle konnte jeweils nur eine Ebene der Variablen, also beispielsweise den Inhalt einer Struktur oder die lokalen Variablen einer Funktion anzeigen.

%	--------------------------------------------------------
% 	Darstellung als Baum
%	--------------------------------------------------------
\section{Darstellung der Variablen als Baum}

\subsection{Darstellungsschema}

Die Darstellung der Variablen in Form einer Baumstruktur hat den Vorteil, dass alle Variablen in einem Feld untergrbracht sind und die Gültigkeitsbereiche der Variablen sofort ersichtlich sind.
Alle Elemente befinden sich in einem Ordner, der den Namen der Datei trägt. Dadurch wird symbolisiert, dass die Variablen ein Teil des Programmes sind. Dem Programm selbst untergeordnet sind globale Variablen und Funktionen. Lokale Variablen sind Funktionen untergeordnet, Variablen in Strukturen sind der Struktur untergeordnet.
Zusätzlich werden folgende Elemente in der Tabelle farbig markiert:
\begin{itemize}
\item Funktionen sind hellblau hinterlegt. Sie sollen dadurch optisch von Datenstrukturen und Variablen unterscheidbar sein.
\item Die Variablen, die zuletzt geändert wurden, sind gelb hinterlegt. Grundsätzlich kann pro Schritt des Debuggers nur eine Variable geändert werden. Werden aber einige Schritte übersprungen, zum Beispiel mit dem Schlüsselwort \glqq library\grqq, so werden mehrere Variablen in einem Debuggerschritt geändert.
% TODO refer to library
\item Variablen, die noch nicht initialisiert wurden, sind grau hinterlegt.
% TODO refer to undef variables
\end{itemize}

Bei Rechtsklick auf eine Variable im Baum wird ein Kontextmenü geöffnet. Mit der Option \glqq{}Gehe zur Deklaration\grqq{} kann der Benutzer zur Deklaration der Variable im Sourcecode springen. Die entsprechende Zeile wird markiert. Für Funktionen gibt es außerdem den Punkt \glqq{}Gehe zum Aufruf\grqq{}, der die Zeile markiert, in der diese Funktion aufgerufen wurde.

\subsection{Grundlegende Komponenten}
Die Variablen werden in einer Tabelle mit Baumstruktur, deshalb auch ,,TreeTable'' genannt, angezeigt. Die TreeTable ist eine Kombination aus einem JTree und einer JTable.
Im Folgenden werden diese Komponenten beschrieben und der Aufbau der TreeTable erläutert:

\subsubsection*{JTree - Baum}
Ein JTree\footnote{http://www.codejava.net/java-se/swing/jtree-basic-tutorial-and-examples} ist Swing-Element, das Daten hierarchisch untereinander darstellet. Die Daten sind allerdings nicht im JTree selbst gespeichert, sondern in externen Objekten. Ein JTree kann entweder mit einem Datenmodell oder mit dem Hauptknoten eines Baumes initialisiert werden\footnote{http://docs.oracle.com/javase/tutorial/uiswing/components/tree.html}. Da die Daten eines JTree einen Baum abbildet, kann durch diese Struktur traversiert werden. Solche Operationen\footnote{http://www.java-tutorial.ch/core-java-tutorial/expande-or-collapse-all-nodes-in-a-jtree} werden beispielsweise benötigt, um alle Knoten eines Baumes zu öffnen (expand) oder zu schließen (collapse).
Wird ein JTree direkt mit einem Baum initialisiert, müssen die Knoten das Interface TreeNode implementieren\footnote{http://docs.oracle.com/javase/7/docs/api/javax/swing/tree/TreeNode.html}. Für einfache Aufgaben können Standardklassen, wie etwa DefaultMutableTreeNode\footnote{https://docs.oracle.com/javase/7/docs/api/javax/swing/tree/DefaultMutableTreeNode.html} verwendet werden.
Mehr Möglichkeiten bietet aber die explizite Verwedung eines Datenmodells. Ein Datenmodell regelt die Interaktion des JTree mit seinem Datenbaum und stellt Methoden zum Einfügen, Löschen und Ändern von Knoten sowie zum Auslesen des Pfades eines Knotens zur Verfügung.
Der JTree in der TreeTable von C Compact verwendet ein eigenes Datenmodell, TreeTableDataModel. Der Baum besteht aus Objekten der Klasse DataNode. Da ein eigenes Datenmodell verwendet wird, muss DataNode das Interface TreeNode nicht implementieren.

\subsubsection*{JTable - Tabelle}
Tabellen können in Swing mit dem Komponenten JTable\footnote{http://docs.oracle.com/javase/tutorial/uiswing/components/table.html} dargestellt werden. Wie auch bei der Klasse JTree werden die Daten nicht im Objekt von JTable gespeichert. Im einfachsten Fall kann eine Tabelle durch ein zweidimensionales Array initialisiert werden\footnote{http://www.java2s.com/Tutorial/Java/0240\_\_Swing/CreatingaJTable.htm}. In der Regel wird aber ein Datenmodell wie etwa DefaultTableModel\footnote{http://docs.oracle.com/javase/8/docs/api/javax/swing/table/DefaultTableModel.html} verwendet. Ein für eine Tabelle verwendbares Datenmodell ist ein Objekt einer Klasse, die das Interface TableModel\footnote{http://docs.oracle.com/javase/8/docs/api/javax/swing/table/TableModel.html} implementiert.

\subsection{Implementierung}
Die grundlegende Implementierung erfolgte anhand eines Blogeintrages\footnote{http://www.hameister.org/JavaSwingTreeTable.html}, mittlerweile wurde der Code allerdings stark an die Anforderungen der Benutzeroberfläche angepasst und erweitert.

Im folgenden Klassendiagramm wird der Aufbau der TreeTable dargestellt. Swing-Komponenten sind Dunkelblau, von Java vorgegebene Klassen und Interfaces sind Hellblau dargestellt. Die meisten Klassen sind selbst implementiert und daher mit weiß gekennzeichnet.

\begin{figure}
\includegraphics[width=\textwidth]{./media/images/gui/var/TreeTableClasses.png}
\caption{Klassendiagramm der TreeTable}
\label{var_tt_class}
\end{figure}

\subsubsection*{TreeTable.java}
Diese Klasse ist die Hauptklasse der TreeTable und, da sie von JTable erbt, eine Swing-Komponente. Hier laufen alle Teile der TreeTable, wie etwa Datenmodelle und Renderer, zusammen.

\subsubsection*{VarTreeTable.java}
VarTreeTable ist eine Wrapperklasse für TreeTable. Sie übernimmt alle wichtigen Funktionen zur Kommunikation mit dem Variablenbaum und bildet so eine Schicht zwischen dem Debugger und der TreeTable. Auf diese Weise wird der Code übersichtlicher und ist besser strukturiert.
TreeTableView hat folgende Methoden:
\begin{itemize}
\item \textbf{init} Initialisiert die TreeTable mit einem Standard-Datenmodell. Dieses besteht nur aus einem Hauptknoten (root), der den Namen der aktuellen Datei trägt; es werden keine Variablen angezeigt. Das Standard-Datenmodell wird immer dann angezeigt, wenn der Debugger nicht aktiv ist, also wenn der Benutzer den Sourcecode editiert.
\item \textbf{standby} Löscht das aktuelle Datenmodell und übergibt ein Standard-Datenmodell an die TreeTable, sodass wie zu Beginn nur der aktuelle Dateiname angezeigt wird. Diese Methode wird immer aufgerufen, wenn der Debugger beendet wird.
\item \textbf{update} Aktualisiert das Datenmodell der TreeTable im Debugmodus und zeigt die Variablen im Speicher des Interpreters an.
\item \textbf{highlightVariable} Mit dieser Methode wird die Addresse einer zuletzt geänderten Variable an die TreeTable übergeben. Diese Variable wird in der Tabelle markiert. Es ist möglich, diese Funktion mehrmals aufzurufen und so mehrere Variablen zu markieren. Wenn das Datenmodell mit \textbf{update} aktualisiert wird, verfallen alle Addressen.
\item \textbf{updateFontSize} Bei Aufruf dieser Methode wird die Schriftgröße der Tabelle geändert und die gesamte TreeTable neu gezeichnet. Diese Methode wird aufgerufen, wenn die Einstellungen der Benutzeroberfläche geändert wurden.
\end{itemize}

\subsubsection*{TreeModel.java}
TreeModel ist ein Interface für alle Datenmodelle eines JTree.

\subsubsection*{TreeTableModel.java}
Dieses Interface erweitert das Interface TreeModel so, dass es mit einer TreeTable kompatibel wird. TreeTableModel definiert zusätzliche Methoden, die für das Datenmodell einer Tabelle benötigt werden, beispielsweise ,,getColumnName()''.

\subsubsection*{AbstractTreeTableModel.java}
Diese abstrakte Basisklasse implementiert einige konkrete Methoden für die Verwendung als Datenmodell eines JTree. In Objekten dieser Klasse werden der Hauptnoten (root) des Baumes gespeichert. Außerdem regelt AbstractTreeModel den Ablauf von Änderungsevents in der Datenstruktur des Baumes.

\subsubsection*{DataNode.java}
Ein Datenknoten (DataNode) ist ein Knoten des in der TreeTable abgebildeten Baumes. In diesen Knoten sind alle nötigen Informationen zu den angezeigten Variablen - auch jene, die nicht im Baum selbst sondern in der Tabelle angezeigt werden - gespeichert.
\newline
Die Klasse DataNode hat folgende Felder:
\begin{lstlisting}[language=JAVA]
// Informationen, die in der Tabelle angezeigt werden
private final String name;
private Object type;
private Object value;

// Zusätzliche Informationen
private int address;
private int declaration;
\end{lstlisting}
Die TreeTable besitzt drei Spalten: In der ersten befindet sich der Baum mit den Namen der Variablen. in der zweiten Spalte wird der Datentyp der Variable und in der dritten ihr Wert gezeigt. Die Felder für Datentyp und Wert haben den Typ Object, da sich in der Tabelle an der Stelle eines Wertes auch ein JButton befinden kann.
%TODO refer to JTableButtonRenderer.java
Zusätzlich werden die Addresse und die Deklarationszeile der Variable gespeichert. Die Adresse dient der eindeutigen Identifikation der Variable, zum Beispiel beim Traversieren durch den Variablenbaum. Dadurch können lokale Variablen auch bei mehrfachem Aufruf einer Funktion unterschieden werden.
Bei Rechtsklick auf eine Variable in der TreeTable wird ein Kontextmenü mit der Option ''Jump to declaration'' geöffnet. Im Sourcecode wird dann die Zeile markiert, die im Feld ''declaration'' des betreffenden Datenknoten gespeichert ist.

\subsubsection*{TreeTableDataModel.java}
TreeTableDataModel erbt von AbstractTreeTableModel beinhaltet einen weiteren Schritt zu einem Tabellen-Datenmodell. Beispielsweise sind Methoden implementiert, die im Datenmodell einer Tabelle benötigt werden. Außerdem wird in dieser Klasse das kontrete Aussehen der Tabelle festgelegt, indem die Tabellenspalten definiert sind:
\begin{lstlisting}[language=JAVA]
// Column names
static protected String[] columnNames = {_("Name"), _("Type"), _("Value")};
 
// Column types
static protected Class<?>[] columnTypes = {TreeTableModel.class, String.class, Object.class};
\end{lstlisting}

\subsubsection*{TableModel.java}
Dieses Interface bildet die Grundlage für alle Datenmodelle einer JTable\footnote{http://docs.oracle.com/javase/7/docs/api/javax/swing/table/TableModel.html}.

\subsubsection*{AbstractTableModel.java}
In dieser abstrakten Basisklasse sind bereits die meisten Methoden des Interface TreeModel implementiert\footnote{http://docs.oracle.com/javase/7/docs/api/javax/swing/table/AbstractTableModel.html}. Der Benutzer muss lediglich die folgenden Methoden selbst definieren:
\begin{lstlisting}[language=JAVA]
public int getRowCount();
public int getColumnCount();
public Object getValueAt(int row, int column);
\end{lstlisting}
Diese Methoden werden in der Klasse TreeTableModelAdapter implementiert.

\subsubsection*{TreeTableModelAdapter.java}
TreeTableModelAdapter bildet die Verbindung zwischen dem Datenmodell der Tabelle und dem des Baumes. Diese Klasse erbt von AbstractTreeTableModel und enthält eine Referenz auf TreeTableDataModel. Viele Methoden, wie etwa ,,getColumnName()'' oder ,,getValueAt()'' sind in TreeTableDataModel implementiert, da sich in dieser Klasse auch das eigentliche Datenmodell der TreeTable befindet, und werden in TreeTableModelAdapter übernommen. Beispielsweise greifen die folgenden Methoden auf die Implementierung in TreeTableDataModel zurück:
\begin{lstlisting}[language=JAVA]
public Object getValueAt(int row, int column) {
   return treeTableModel.getValueAt(nodeForRow(row), column);
}

public boolean isCellEditable(int row, int column) {
   return treeTableModel.isCellEditable(nodeForRow(row), column);
}
\end{lstlisting}

\subsubsection*{CellEditor.java}
Dieses Interface bildet die Grundlage für alle Klassen, die CellEditor einer JTable sind\footnote{https://docs.oracle.com/javase/7/docs/api/javax/swing/CellEditor.html}.
Ein CellEditor ist für die Dateneingabe in eine Tabelle verantwortlich\footnote{http://docs.oracle.com/javase/tutorial/uiswing/components/table.html\#editor}. Die einfachse Form eines CellEditors ist ein DefaultCellEditor\footnote{https://docs.oracle.com/javase/7/docs/api/javax/swing/DefaultCellEditor.html}. Dieser unterstützt die Eingabe von Daten in die Tabelle mithilfe eines Textfeldes, einer Checkbox oder einer Combobox.

\subsubsection*{AbstractCellEditor.java}
AbstractCellEditor ist eine abstrakte Basisklasse für alle Arten eines CellEditor in Swing\footnote{https://docs.oracle.com/javase/7/docs/api/javax/swing/AbstractCellEditor.html}. Einige grundlegende Methoden sind bereits implementiert.

\subsubsection*{TreeTableCellEditor.java}
Daten in der TreeTable sollen vom Benutzer zwar nicht verändert werden können, wenn das Editieren der Tabelle mit der Methode ,,setEditable()'' aber deaktiviert wird, werden auch Events (wie etwa Mausklicks) nicht mehr registriert. TreeTableCellEditor gibt also Events in der Tabelle bei Bedarf an den JTree weiter und verbietet Events in Zellen, die nur Text enthalten.

\subsubsection*{TableCellRenderer.java}
In diesem Interface sind alle Methoden definiert, die zum Rendern von Zellen einer JTable benötigt werden\footnote{https://docs.oracle.com/javase/7/docs/api/javax/swing/table/TableCellRenderer.html}.

\subsubsection*{TreeTableCellRenderer.java}
Diese Klasse erbt von JTree und wird verwendet, um einen Baum in die erste Spalte der Treetable zu rendern. Außerdem ist diese Klasse dafür verantwortlich, dass die Zeilen der Tabelle und des Baumes die selbe Höhe haben und dass die Zellen die richtige Farbe erhalten.
%TODO refer to table line colors

\subsubsection*{TableButtonRenderer.java}
Diese Klasse wird als Renderer für die zweite und dritte Spalte der TreeTable verwendet. Sie erlaubt das Verwenden von Swing-Komponenten wie etwa JButtons als Zellendaten und sorgt dafür, dass die Zellen die korrekte Hintergrundfarbe haben.

%TODO weitere Elemente der TreeTable: Popup, Kontextmenü

%   Popups
%!TEX root = "../../../DA_GUI.tex"

%	--------------------------------------------------------
% 	Debugger: Popups
%	--------------------------------------------------------


\section{Popups}
\label{sec:deb-popup}
Im Debugger von C Compact gibt es mehrere Anwendungsfälle für Popup-Fenster. Popups werden verwendet, um weitere Informationen zu einem Array oder einem String in der Variablentabelle zu liefern und um Rückgabewerte von Funktionen zu visualisieren (dies ist allerdings optional, siehe auch \ref{sec:win-set}). Dabei handelt es sich im Prinzip um kleine Anzeigeflächen, die ein Swing-Element enthält. Ein Popup wird ausgeblendet, sobald auf eine beliebige Fläche in einem Fenster von C Compact geklickt wird (der Event muss erfasst werden können).

\begin{figure}[htp]
\centering
\includegraphics[width=0.4\textwidth]{./media/images/gui/popup/popup-example.png}
\caption{Ein Popup zeigt den Rückgabewert einer Funktion.}
\label{fig:popup-example}
\end{figure}

\subsection{Initialisierung eines Popups}
Alle für das Popup relenanten Klassen befinden sich im Package \textbf{at.jku.ssw.cmm.gui.popup}. Die Popup-Funktionen selbst befinden sich in der Klasse \textbf{ImagePopup}. Zum Initialisieren werden aber die statischen Methoden in der Klasse \textbf{ComponentPopup} verwendet; diese Methoden übernehmen auch umständliche Positionsberechnungen für das Popup. Die Methode \textbf{createPopUp(...)} kann auch ohne das Parameter \glqq{}weight\grqq{} verwendet werden.
\begin{lstlisting}[language=JAVA]
public static void createPopUp( GUImain main, JComponent component, int x, int y, int w, int h, int orientation, double weight );
\end{lstlisting}

\begin{table}[h!]
\begin{tabular}{|ll|l|}
\hline 
GUImain & main  & Referenz auf das Hauptfenster, siehe \ref{sec:gui-main-impl} \\
\hline
JComponent & component & Das Swing-Element, das angezeigt werden soll \\
\hline
int & x & \multirow{2}{8cm}{Die Position, auf die das Popup zeigen soll} \\
int & y & \\
\hline 
int & w & \multirow{2}{8cm}{Höhe und Breite des Popups (Bezogen auf den äußeren Rand, \textbf{component} wird mit 5px Inset platziert)} \\
int & h & \\
\hline
int & orientation & Ausrichtung des Popups (siehe Unten) \\
\hline
double [optional] & weight & Position des Zeigers, Wert zwischen 0 und 1, Standard: 0.5\\
\hline
\end{tabular}
\caption{Parameter der Methode \textbf{createPopup(...)}}
\end{table}

\begin{figure}[h!]
\centering
\includegraphics[width=0.4\textwidth]{./media/images/gui/popup/popup-pos.png}
\caption{Positionierung eines Popups}
\end{figure}

Für das Parameter \textbf{orientation} stehen vier Möglichkeiten zur Verfügung. Die Konstanten sind in der Klasse \textbf{ImagePopup} definiert.
\begin{itemize}
\item \textbf{NORTH:} Zeiger befindet sich am oberen Rand
\item \textbf{SOUTH:} Zeiger befindet sich am unteren Rand
\item \textbf{WEST:} Zeiger befindet sich am linken Rand
\item \textbf{EAST:} Zeiger befindet sich am rechten Rand
\end{itemize}

Mit dem Parameter \textbf{weight} kann die Position des Zeigers am Popup bestimmt werden. Der Wert 0 bedeutet, dass der Zeiger --- je nach Ausrichtung --- ganz links oder ganz oben ist.

Unabhängig von Ausrichtung und Positionierung des Zeigers befindet sich dessen Spitze immer an dem mit den Parametern \textbf{x} und \textbf{y} angegebenen Punkt.

\subsection{Implementierung}
Die eigentliche Popup-Klasse, \textbf{ImagePopup}, muss mit weniger praktischen Parametern initialisiert werden: mit x und y ist beispielsweise die linke obere Ecke des Popups anzugeben. Deshalb werden zum Initialisieren die statischen Methoden in der Klasse ComponentPopup verwendet. Diese rechnen die angegebenen Parameter auf die Popup-Positionen um.

\subsubsection*{Das Popup in das Hauptfenster zeichnen}
In Swing besteht jedes Fenster aus mehreren Schichten\footnote{https://docs.oracle.com/javase/tutorial/uiswing/components/rootpane.html}, die unterschiedliche Aufgaben übernehmen. Die oberste Schicht ist das sogenannte \textbf{GlassPane}. Dieses ist standardmäßig deaktiviert. Wird es aktiviert, kann es verwendet werden, um über die Komponenten des Hauptfensters zu zeichnen und Events des Fensters abzufangen\footnote{https://docs.oracle.com/javase/tutorial/uiswing/components/rootpane.html\#glasspane}.

Die Klasse \textbf{ImagePopup} erbt von \textbf{JPanel} und ist daher eine einfache Swing-Komponente. 

Mit der Methode \textbf{invokePopup} wird eine beliebige Komponente zum GlassPane hinzugefügt und dort gezeichnet. Diese Methode befindet sich in der Klasse \textbf{GUImain} (siehe Kapitel \ref{sec:gui-main}), die das Hauptfenster von C Compact verwaltet.
\begin{lstlisting}[language=JAVA]
public void invokePopup(JPanel popup, int x, int y, int width, int height) {

	// Popup wird zum GlassPane hinzugefügt
	((JPanel) this.jFrame.getGlassPane()).add(popup);
	
	// Listener zum Schließen des Popuos wird initialisiert
	((JPanel) this.jFrame.getGlassPane()).addMouseListener(
		new PopupCloseListener(this.jFrame, ((JPanel) this.jFrame.getGlassPane()), popup, x, y, width, height));
		
	// GlassPane neu zeichnen
	((JPanel) this.jFrame.getGlassPane()).validate();
	((JPanel) this.jFrame.getGlassPane()).repaint();
}
\end{lstlisting}

In der Klasse \textbf{ImagePopup} wird die Methode \textbf{paintComponent}\footnote{http://www.oracle.com/technetwork/java/painting-140037.html\#callbacks} überschrieben. Dies ist eine der drei Methoden, die aufgerufen werden, wenn die Benutzeroberfläche neu gerendert werden muss. In gewisser Weise entspricht das der \textbf{paint}-Methode in ATW, allerdings spaltet Swing das Rendern auf drei unterschiedliche Methoden auf: \textbf{paintComponent} zeichnet das Element selbst, \textbf{paintBorder} zeichnet den Rand des Komponenten und \textbf{paintChildren} rendert die Komponenten innerhalb des zu zeichnenden Elementes. Durch diese Aufteilung können Manipulationen am Rendering eines Komponenten gezielter vorgenommen werden.

\begin{lstlisting}[language=JAVA]
@Override
public void paintComponent(Graphics g) {
        
    // Grenzen des Komponenten werden gesetzt
    switch( this.orientation ) {
    case NORTH:
    	g.setClip(g.getClipBounds().x, g.getClipBounds().y-10, g.getClipBounds().width, g.getClipBounds().height+10);
    	break;
    case SOUTH:
    	g.setClip(g.getClipBounds().x, g.getClipBounds().y, g.getClipBounds().width+10, g.getClipBounds().height+10);
    	break;
    case WEST:
    	g.setClip(g.getClipBounds().x-10, g.getClipBounds().y, g.getClipBounds().width+10, g.getClipBounds().height);
    	break;
    case EAST:
    	g.setClip(g.getClipBounds().x, g.getClipBounds().y, g.getClipBounds().width+10, g.getClipBounds().height);
    	break;
    }
    // Die Rendering-Methoden der Basisklasse werden aufgerufen
    super.paintComponent(g);
    super.paintBorder(g);
    	
    // Der Zeiger wird gezeichnet
    switch( this.orientation ) {
    case NORTH:
    	g.drawImage(image, centerX-10, centerY, image.getWidth(null), image.getHeight(null), this);
    	break;
    case SOUTH:
    	g.drawImage(image, centerX-10, centerY-15, image.getWidth(null), image.getHeight(null), this);
    	break;
    case WEST:
    	g.drawImage(image, centerX, centerY-10, image.getWidth(null), image.getHeight(null), this);
    	break;
    case EAST:
    	g.drawImage(image, centerX-15, centerY-10, image.getWidth(null), image.getHeight(null), this);
    	break;
    }
}
\end{lstlisting}

Mit der Methode \textbf{setClip} wird der Zeichenbereich des Popups so erweitert, dass der Zeiger gezeichnet werden kann, obwohl er außerhalb des JPanels liegt.

\subsubsection*{Popups mit Tabellen}
Um den Inhalt eines Arrays anschaulicher darzustellen, kann das Array in einem Popup geöffnet werden. Es wird dann als Tabelle dargestellt.

\begin{figure}[h!]
\centering
	\begin{minipage}{0.45\textwidth}
		\centering
		\includegraphics[width=1.0\textwidth]{./media/images/gui/popup/array1dim.png}
		\caption{Ein eindimensionales Array wird in einem Popup dargestellt}
		\label{fig:gui-popup-a1}
	\end{minipage}\hfill
	\begin{minipage}{0.46\textwidth}
		\centering
		\includegraphics[width=1.0\textwidth]{./media/images/gui/popup/array2dim.png}
		\caption{Ein zweidimensionales Array wird in einem Popup dargestellt}
		\label{fig:gui-popup-a2}
	\end{minipage}
\end{figure}

Tabellen können grundsätzlich wie alle anderen Komponenten in einem Popup dargestellt werden. Für die Implementierung im Debugger (Abbildung \ref{fig:gui-popup-a1} und Abbildung \ref{fig:gui-popup-a2}) wurden allerdings ein eigenes Datenmodell und ein eigener Renderer erstellt. \textbf{TablePopupModel} ist ein Datenmodell, das den Inhalt eines Arrays korrekt darstellen kann. Für zweidimensionale Arrays wird die erste Tabellenspalte mit \textbf{TablePopupRenderer} gezeichnet, sodass sie wie die Kopfspalte der Tabelle aussieht (Abbildung \ref{fig:gui-popup-a2}).



%   Fehlerbeschreibungen
%!TEX root = "../../../DA_GUI.tex"

%	--------------------------------------------------------
% 	Debugger: Fehleranzeige
%	--------------------------------------------------------

\section{Anzeige von Programmfehlern}
\label{sec:deb-error}
Im Quelltext des Benutzers können unterschiedliche Arten von Fehlern enthalten sein, die dazu führen, dass der Debugger entweder nicht gestartet werden kann (Probleme im Präprozessor oder im Compiler) oder dass der Debugger beendet wird (Laufzeitfehler). Allerdings wissen gerade unerfahrene Programmierer oft nicht, wodurch das Problem verursacht wurde und haben deshalb auch Schwierigkeiten, den Fehler zu finden und zu beheben.

Bei Versuchen mit Schülern wurde erneut besonders deutlich, dass die Fehlermeldungen des Compilers für Programmieranfänger keine große Hilfe sind (siehe Kapitel \ref{sec:sci-trial-gui}).

Beim Debuggen aufgetretene Fehler --- diese sollten nicht mit internen Fehlern verwechselt werden (siehe Kapitel \ref{sec:gui-int-error}) --- werden in C Compact im Sourcecode rot markiert, außerdem zeigt das Zustandspanel des Debuggers Fehlername und die Zeilennummer an (siehe Kapitel \ref{sec:gui-main-left-zust}). Im rechten Teil der Benutzeroberfläche wird eine detaillierte Beschreibung des Fehlers mit möglichen Ursachen und Hinweisen zur Fehlersuche gezeigt (siehe Kapitel \ref{sec:gui-main-right-error}).

\begin{figure}[h!]
\centering
\includegraphics[width=0.8\textwidth]{./media/images/gui/debugger/gui-error.png}
\caption{Ein Fehler im Quelltext wird angezeigt}
\end{figure}

Diese Fehlerbeschreibungen werden im Ordner \textbf{error} in einem Unterordner mit dem jeweiligen Sprachkennzeichen --- zum Beispiel \textbf{de} für Deutsch --- als HTML-Dokumente gespeichert. Im Ordner \textbf{error} befinden sich außerdem die Dateien \textbf{table.xml} und \textbf{style.css}. In der Datei \textbf{table.xml} werden Fehlermeldungen einzelner Komponenten von C Compact mit der zugehörigen Beschreibungsdatei verknüpft. Der Pfad wird unabhängig vom jeweiligen Sprachordner angegeben, sodass die Fehlermeldung in der eingestellten Sprache geladen werden kann.
\begin{lstlisting}[language=XML]
<error id='rbrac expected'>
	<file>compiler/parser/rbrac_expected.html</file>
</error>
\end{lstlisting}

Für unbekannte Fehlermeldungen ist die Datei \textbf{default.html} im jeweiligen Sprachordner vorhanden. Derzeit sind sind nur für die deutsche Übersetzung alle Fehlermeldungen ausgearbeitet. Auf Englisch sind also die meisten Fehlerbeschreibungen nicht verfügbar.

Die Datei \textbf{style.css} enthält die Formatierungsinformationen für die Fehlerdokumente





%!TEX root = "../../../DA_GUI.tex"

%	--------------------------------------------------------
% 		Wissenschaftliche Analyse
%	--------------------------------------------------------

\chapter{Wissenschaftliche Analyse}

%   Einleitung
%!TEX root = "../../../DA_GUI.tex"

%	--------------------------------------------------------
% 		Wissenschaftliche Analyse: Einleitung
%	--------------------------------------------------------

\section{Einleitung}

Bei der Entwicklung eines Programms und der Gestaltung der Benutzeroberfläche ist es wichtig, die Anforderungen beziehungsweise die Anwendung zu kennen und den Fortschritt des Programms regelmäßig zu überprüfen.

\section{Versuche mit SchülerInnen}
\label{sec:sci-trial-intro}
Um sicherzustellen, dass C Compact für den Unterricht tatsächlich geeignet ist und die Schülerinnen und Schüler auch gut mit der Entwicklungsumgebung umgehen können, haben wir beschlossen, Versuche mit Schülerinnen und Schülern der ersten und zweiten Klassen durchzuführen. Während unseres Projektes hatten wir insgesamt vier mal die Möglichkeit, unser Programm mit einer FSST-Gruppe (14-16 SchülerInnen) zu testen.

\def\arraystretch{1.6}
\begin{table}[h!]
\begin{tabular}{|l|l|l|l||l|l|}
\hline
Datum & Klasse & \parbox{1.3cm}{Anzahl Schüler} & Lehrer & Version & Testgegenstand \\
\hline
4. 11. 2014 & 2AHELS & 14 & Franz Matejka & Alpha 1.1 & Benutzeroberfläche \\
3. 12. 2014 & 2BHELS & 16 & Kurt Kreilinger & Alpha 1.2 & Benutzeroberfläche \\
18. 3. 2015 & 2BHELS & 15 & Christian Hanl & Alpha 1.4.2 & Benutzeroberfläche \\
22. 4. 2015 & 1AHELS & 16 & Reinhard Pfoser & Alpha 1.4.5 & Questsystem \\
\hline
\end{tabular}
\caption{Versuche mit Schülern der ersten und zweiten Klassen}
\end{table}
Siehe auch Kapitel \ref{sec:versions}, Versionsgeschichte.

Bei den ersten drei Versuchen sollte vor allem die Benutzeroberfläche und den Debugger verbessert werden, beim letzten Versuch sollte hingegen das Questsystem getestet und optimiert werden. Dieser Versuch wurde dementsprechend anders gestaltet (beispielsweise durch andere Aufgabenstellungen).

\subsection{Ziel der Versuche}
Die Versuche sollten nicht nur die grundsätzliche Anwendbarkeit von C Compact unter Beweis stellen, sondern auch Programmfehler, Bugs und Probleme bei der Bedienung aufzeigen. Als Entwickler ist es oft schwer, die Ansätze und Bedienungsvorgänge anderer Benutzer nachzuvollziehen. So entstehen zwangsläufig Fehler, die der Programmierer selbst nicht bemerkt, da er das Programm immer so bedient, wie er es für logisch erachtet.

%TODO kein Widerspruch mit Pointhi
Die Benutzeroberfläche kann, im Gegensatz zum Compiler oder Interpreter, nur beschränkt durch statische Codeanalysen und Testroutinen verbessert werden, da ein wichtiger Faktor für eine reibungslose Funktionalität der Benutzer ist. Wir konnten mit den Versuchen viele neue Ansätze sammeln und sahen, welche Funktionen und Features für die Schüler tatsächlich von Bedeutung sind.

\subsection{Ablauf eines Versuches}
\label{sec:sci-trials-gui-sched}
Die Versuche begannen immer mit einer Einführungsphase. Zuerst stellten wir uns vor und erklärten den Versuch und seinen Zweck. Dann folgte eine Einführung in C Compact, einerseits in die verwendete Sprache, andererseits auch in die Funktionsweise der Benutzeroberfläche. Im zweiten Versuch fanden wir heraus, dass es in der Einführungsphase besonders wichtig ist, auf den Debugger hinzuweisen. Besonders Schüler mit schlechteren Programmierkenntnissen waren mit der Funktionsweise eines Debuggers nicht vertraut und benutzten C Compact wie einen einfachen Texteditor. Im dritten und vierten Versuch führten wir zu Beginn eine einfache Übung zum Debugger mit der Klasse gemeinsam durch. Dafür konnten wir einen zuvor noch umfangreicheren Teil der Einführung zu Spracheigenheiten von C Compact kürzen, da Compiler und Präprozessor im Laufe des Projektes große Fortschritte erzielen konnten und die Sprache C immer ähnlicher wurde.

Den größte Teil der Versuchszeit arbeiteten die Schüler selbstständig an von uns gestellten Aufgaben. Bei den ersten drei Versuchen wurden Übungsblätter in Form eines PDF-Dokumentes gemeinsam mit C Compact verteilt; beim letzten Test wurden die Aufgaben mit dem Questsystem gestellt. In dieser Übungszeit beobachteten wir die Schüler, notierten Probleme und Programmfehler und halfen bei Schwierigkeiten. Auf diese Weise erhielten wir einen guten Eindruck von Problemen sowie Vor- und Nachteilen bei der Anwendung.

Im letztlich vergleichbare Ergebnisse zu erzielen und einen objektiven Gesamteindruck zu erhalten, wurden in den letzten 10 bis 15 Minuten der verwendeten Unterrichtszeit Fragebögen mit Fragen zum jeweiligen Versuchsthema ausgeteilt. Während die in der Übungszeit aufgenommenen Informationen vor allem einzelne Probleme aufzeigen, hatten wir durch die Fragebögen die Möglichkeit, die weitere Arbeit an C Compact nach einem allgemeinen Trend zu orientieren. Zum Beispiel zeigte der erste Versuch, als nächstes die Stabilität des Programms gegen Fehler erhöht werden sollte.
%TODO ref Anhang-> Versuchsunterlagen, FRAGEBÖGEN

\subsection{Persönliche Erfahrungen}
Bei den Versuchen mit Schülergruppen konnten wir auch außerhalb des Projekts viele hilfreiche Erfahrungen sammeln. Wir mussten die für den Versuch verwendeten Unterrichtsstunden selbst vorbereiten und ein Thema aus der Informatik, wie etwa den Sortieralgorithmus \glqq{}Bubblesort\grqq{} und die zugehörigen Übungen erklären. Anschließend unterstützten wir die Schülerinnen und Schüler der Unterrichtsgruppe beim Lösen der Aufgaben.

Dadurch konnten wir einerseits ein besseres Verständnis für den Programmierunterricht und die Aufgabe des Lehrers erreichen, was auch für weitere Entwicklungen des Projektes hilfreich war. Andererseits ist es auch eine interessante persönliche Erfahrung, nicht unter den Schülern zu sitzen, sondern vor der Klasse zu stehen.


%!TEX root=../Vorlage_DA.tex
%	########################################################
% 				Allgemeiner Teil (Theorie)
%	########################################################


%	--------------------------------------------------------
% 	Versionsgeschichte
%	--------------------------------------------------------
\chapter{Versionsgeschichte}
Dieses Kapitel enthält eine Auflistung der Versionen von C Compact. Die Versionen werden wie folgt unterteilt:
\begin{itemize}
\item \textbf{Kein Versionsname:} Die Versionsnummerierung begann erst im Herbst 2014. Alle Versionen, die zuvor entstanden sind, sind nicht nummeriert. Das betrifft im wesentlichen unsere Arbeit am Institut für Systemsoftware der Johannes Kepler Universität im Juli 2014. Zu dieser Zeit war eine Unterteilung in Versionen noch nicht unbedingt nötig.
\item \textbf{Alpha:} Versionen mit dem Namen Alpha umfassen die Entwicklungen vom Herbst 2014 bis Anfang 2015. Unter die Bezeichnung Alpha fallen alle Versionen, in denen das Questsystem noch nicht vollständig implementiert ist.
%%TODO Referenz auf Questsystem
\end{itemize}

Buildnummern wurden nicht nach der tatsächlichen Anzahl der Builds vergeben, sondern nach kleinen Erweiterungen, die noch in die aktuelle Version eingegliedert wurden.

Nachfolgend werden alle bisher benannten Versionen aufgelistet und beschrieben. Es sind nicht alle Builds vollständig aufgelistet, da die Änderungen teilweise minimal waren.

%	--------------------------------------------------------
% 	Kein Versionsname
%	--------------------------------------------------------
\subsubsection*{Pre-Alpha}
Entwicklungsstart: 7. Juli 2014
\begin{itemize}
\item Umfasst alle grundlegenden Funktionen: eine einfache, C-ähnliche Sprache und die Grundversion der aktuellen Benutzeroberfläche
\item Dateimanagement: Neue Datei - Öffnen - Speichern
\item String als Datentyp (ähnlich wie in Java)
\item Minimaler Präprozessor
\end{itemize}

%	--------------------------------------------------------
% 	Alpha 1.0
%	--------------------------------------------------------
\subsubsection*{Alpha 1.0}
Gültig ab: 1. Oktober 2014
\begin{itemize}
\item Variablen können sowohl als Baumstruktur, als auch in Tabellen dargestellt werden
\item Beim Schließen wird überprüft, ob die aktuelle Datei verändert wurde
\item Implementierung einer Standardbibliothek
\item Breakpoints
\end{itemize}

%	--------------------------------------------------------
% 	Alpha 1.1 (Build 3)
%	--------------------------------------------------------
\subsubsection*{Alpha 1.1}
Gültig ab: 4. November 2014,
Getestet: 5. November 2014 mit der 2AHELS
%%TODO Referenz auf 1. Versuch
\begin{itemize}
\item Variablenanzeige mit Tabellen wurde entfernt
\item Variablenanzeige als Baumstruktur wurde wesentlich optimiert
\item Step-over- und Step-out-buttons immer sichtbar
\end{itemize}

%	--------------------------------------------------------
% 	Alpha 1.2 (Build 1)
%	--------------------------------------------------------
\subsubsection*{Alpha 1.2 (Build 1)}
Gültig ab: 24. November 2014
\begin{itemize}
%%TODO Referenz auf Ausführungsmodus
\item Ausführungsmodus (Texteditor, Fehler, Automatisches Debuggen, Schritt-für-Schritt-Debuggen) wird sichtbar durch ein farbiges Panel über dem Textfeld gekennzeichnet.
\item Arrays können an Funktionen als Referenz übergeben werden
\item Entfernen der Step-over und Step-out Funktionen
\item Schlüsselwort "library" für Funktionen, die im Debugger übersprungen werden sollen
\item HTML-Dokumente für unterschiedliche Fehler im Debugmodus (Compiler, Preprozessor, Laufzeitfehler)
\item Tooltips für die wichtigsten Bedienelemente (Buttons, etc.)
\item Logarithmische Skalierung der Zeitschritte beim automatischen Debuggen
\item Die größten Elemente der Benutzeroberfläche sind verschiebbar und damit besser anzupassen
\end{itemize}

%	--------------------------------------------------------
% 	Alpha 1.2 (Build 4)
%	--------------------------------------------------------
\subsubsection*{Alpha 1.2 (build 4)}
Gültig ab: 2. Dezember 2014,
Getestet: 3. Dezember 2014 mit der 2BHELS
%%TODO Referenz auf 2. Versuch
\begin{itemize}
\item Verbessertes Design bei Fehlermeldungen
\item Eigener Syntax Highlighter für die Sprache von C Compact
\end{itemize}



\fi

%	--------------------------------------------------------
% 	Persönliche Erfahrungen
%	--------------------------------------------------------
%!TEX root=../Vorlage_DA.tex
%	########################################################
% 					Persönliche Erfahrungen
%	########################################################


%	--------------------------------------------------------
% 	Überschrift, Inhaltsverzeichnis
%	--------------------------------------------------------
\chapter{Persönliche Erfahrungen}

\section{Zusammenarbeit mit dem Institut für\\ Systemsoftware}
\label{sec:coop-jku}
Wir begannen bereits bei einem Pratkikum am Institut im Sommer 2014 mit unserem Projekt. Besonders in der Anfangsphase waren unsere Projektbetreuer an der Universität eine große Hilfe. Wir erlernten schnell die Grundlagen von Compiler- und Interpreterbau und entwarfen das grundsätzliche Aussehen der Benutzeroberfläche.

Während des Schuljahres schicken wir regelmäßig Fortschrittsberichte und aktuelle Ergebnisse an unsere Betreuer und erhalten dann Feedback. Wir führen unser Projekt weitgehend selbstständig durch, können uns bei Problemen aber jederzeit an unsere Ansprechpartner in der Universität wenden.

Kurz vor Weihnachten, am 22. 12. 2014, Verbrachten wir noch einmal einen Vormittag an der JKU, wo wir den Fortschritt unseres Projektes vorstellten und die aktuelle Version von C Compact präsentierten. Wir erhielten positive Rückmeldungen zu unserem Arbeitsfortschritt und hatten anschließend eine interessante Diskussion mit Herrn Professor Balschek, bei der wir neue Ideen zur Implementierung des Aufgabensystems sammelten.

\section{Zusammenarbeit mit unserem Projektbetreuer}
In der Schule wird unser Projekt von Herrn Dipl.-Ing. Franz Matejka betreut. Wir haben jede Woche acht Laboreinheiten, bei denen wir weitgehend selbstständig an unserem Projekt arbeiten. Herr Matejka unterstützt uns wenn notwendig mit Tips und Ideen zum Projekt.

\section{Teamarbeit bei der Softwareentwicklung}
Bei der gemeinsamen Entwicklung eines Softwareprojektes ist es wichtig, zusammenzuarbeiten und sich mit den Teamkollegen abzustimmen. Da wir uns schon vor Beginn des Projektes gut kannten, war die Zusammenarbeit an sich kein großes Problem. Da sich das manuelle Zusammenfügen unterschiedlicher Programmkomponenten jedoch als mühsam und umständlich erwies, begannen wir bald, die Versionsverwaltungssoftware git zu verwenden (siehe Kapitel \ref{sec:dev-git}).

Dadurch wurde die Teamarbeit hinsichtlich Softwareentwicklung wesentlich vereinfacht. Weil wir am Anfang Schwierigkeiten mit git und seinem weiten Umfang an Funktionen hatten, ist dieser Teil unserer Zusammenarbeit eine besonders prägende Erfahrung. Die Verwendung von Versionsverwaltungssoftware ist in der Softwareentwicklung unerlässlich.

Professionelle Zusammenarbeit und die richtige Kommunikation in einem Entwicklerteam sind mit einem Lernprozess verbunden, der in einem großen Projekt dauernd stattfindet.

\section{Versuche mit SchülerInnen}
Bei den Versuchen mit Schülergruppen konnten wir auch außerhalb des Projekts viele hilfreiche Erfahrungen sammeln. Wir mussten die für den Versuch verwendeten Unterrichtsstunden selbst vorbereiten und ein Thema aus der Informatik, wie etwa den Sortieralgorithmus \glqq{}Bubblesort\grqq{} und die zugehörigen Übungen erklären. Anschließend unterstützten wir die Schülerinnen und Schüler der Unterrichtsgruppe beim Lösen der Aufgaben.

Dadurch konnten wir einerseits ein besseres Verständnis für den Programmierunterricht und die Aufgabe des Lehrers erreichen, was auch für weitere Entwicklungen des Projektes hilfreich war. Andererseits ist es auch eine interessante persönliche Erfahrung, nicht unter den Schülern zu sitzen, sondern vor der Klasse zu stehen.

%	--------------------------------------------------------
% 	Autoren
%	--------------------------------------------------------
%!TEX root=../Vorlage_DA.tex
%	########################################################
% 					Autoren
%	########################################################


%	--------------------------------------------------------
% 	Überschrift, Inhaltsverzeichnis
%	--------------------------------------------------------
\chapter{Autoren}

%	--------------------------------------------------------
% 	Autor 1
%	--------------------------------------------------------
\htlParagraph{Fabian Hummer}

\renewcommand{\arraystretch}{1.2}
\begin{tabularx}{1\textwidth}{@{} l X l @{}}

\emph{Geburtstag, Geburtsort:} & 14.10.1995, Gmunden & 
\multirow{5}{2.5cm}{\includegraphics[width=2.5cm, angle=180]{./media/images/authors/fabian.jpg}
} 
\\
%%%\emph{Schulbildung:} & Volksschule \newline Hauptschule \newline HTL & \\
%%%\emph{Praktika:} & Firmenname, Zeit, Tätigkeit & \\
\emph{Anschrift:} & Im Gsperr 36\newline 4810 Gmunden\newline Österreich & \\
\emph{E-Mail:} & f.hummer@traunseenet.at & \\

\end{tabularx}
\\\\


%	########################################################
% 	Anhang		
%	########################################################
\appendix
\renewcommand{\thechapter}{\Alph{chapter}}

\setcounter{chapter}{0} 

\chapter{Dokumente}
Beschreibungstext...

Dieses Dokument wurde von Herrn Professor Mössenböck zu Beginn des Projektes erstellt. Hier werden die grundlegenden Anforderungen an Compiler und Interpreter festgelegt.

Dieses Dokument wurde von Herrn Professor Blaschek zu Beginn des Projektes erstellt. Hier werden die Funktionen und das Aussehen der ersten Benutzeroberfläche festgelegt.

\pagebreak
\section{Ursprüngliche Anforderungen an Compiler und Interpreter}
\label{app:anf-comp}

\begin{figure}[h!]
	\centering
	\includegraphics[width=1.0\textwidth]{./media/docs/Anforderung-compiler.pdf}
\end{figure}

\includepdf[pages=2-last]{./media/docs/Anforderung-compiler.pdf}

\pagebreak

\section{Ursprüngliche Anforderungen an die Benutzeroberfläche}
\label{app:anf-gui}

\begin{figure}[h!]
	\centering
	\includegraphics[width=1.0\textwidth]{./media/docs/Anforderung-gui.pdf}
\end{figure}

\pagebreak
\section{Fragebogen zur Entwicklungsumgebung}
Verwendet bei den ersten drei Versuchen (siehe Kapitel \ref{sec:sci-trial-gui}).

\begin{figure}[h!]
	\centering
	\includegraphics[width=1.0\textwidth]{./media/docs/Fragebogen-gui.pdf}
	\label{fig:app-doc-fb-gui}
\end{figure}

\pagebreak
\section{Fragebogen zum Questsystem}
Verwendet beim vierten Versuch (siehe Kapitel \ref{sec:sci-trial-quest}).

\begin{figure}[h!]
	\centering
	\includegraphics[width=1.0\textwidth]{./media/docs/Fragebogen-quest.pdf}
	\label{fig:app-doc-fb-quest}
\end{figure}

%\includepdf{./media/docs/Fragebogen-gui.pdf}


%	########################################################
% 	Quellenverzeichnis 		
%	########################################################
\addcontentsline{toc}{chapter}{Literaturverzeichnis}
\bibliographystyle{ieeetr}
\bibliography{./includes/literatur}


%	########################################################
% 	Abbildungsverzeichnis 		
%	########################################################
%\addcontentsline{toc}{chapter}{Abbildungsverzeichnis}
%\listoffigures

\end{document}  